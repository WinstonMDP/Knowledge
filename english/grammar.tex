\documentclass[oneside]{book}

\usepackage[utf8]{inputenc}
\usepackage[T2A]{fontenc}
\usepackage[russian]{babel}
\usepackage[left = 0.3\textwidth, right = 0.3\textwidth]{geometry}
\usepackage{parskip}
\usepackage[fleqn]{amsmath}
\usepackage{multirow}
\usepackage{textcomp}
\usepackage{hyperref}

\hypersetup{
    colorlinks,
    citecolor=black,
    filecolor=black,
    linkcolor=black,
    urlcolor=black
}

\title{Английский язык}
\date{\today}
\author{WinstonMDP}

\begin{document}
\maketitle

\tableofcontents

\chapter{Падежи}
\section{Subjective}
Слово является подлежащим.

Слово не изменяется.

\section{Objective}
Слова всех частей речи, кроме личных местоимений, не изменяются.

\section{Possessive.}
Слово используется для обозначения принадлежности.

Употребляется в существительных только одушевлённых.

К существительному или неопределённому местоимению добавляется  апостроф и окончание s.

Если окончание s не имени собственного уже присутствует, то добавляется
только апостроф после окончания.

Если у объекта, представляющего главное слово, несколько владельцев,
представляющих цепочку слов зависящих от этого главного слова, окончание s добавляется
только к последнему слову в цепочке.

Слово, от которого зависит слово в possessive падеже, может опускаться.

Неодушевлённые существительные,
с которыми употребляется
притяжательный падеж:
\begin{enumerate}
    \item Группы людей, организации.
    \item Временные отрезки, расстояния.
    \item Географические названия.
    \item Средства передвижения.
    \item Объекты природы уникальные явления.
    \item Праздники.
    \item В устойчивых выражениях.
\end{enumerate}

\chapter{Местоимения}
\section{Личные местоимения (и притяжательные)}
\begin{enumerate}
    \item \textbf{Subjective case.}
    \begin{center}
        \begin{tabular}{|c|c|c|}
            \hline

            & Ед. ч. & Мн. ч.
            \\
            \hline

            1 лицо & I & we
            \\
            \hline

            2 лицо & you & you
            \\
            \hline

            3 лицо & he, she, it & they
            \\
            \hline
        \end{tabular}
    \end{center}

    \item \textbf{Objective case.}
    \begin{center}
        \begin{tabular}{|c|c|c|}
            \hline

            & Ед. ч. & Мн. ч.
            \\
            \hline

            1 лицо & me & us
            \\
            \hline

            2 лицо & you & you
            \\
            \hline

            3 лицо & him, her, it & them
            \\
            \hline
        \end{tabular}
    \end{center}

    \item \textbf{Possessive case.}
    \begin{center}
        \begin{tabular}{|c|cc|cc|cc|}
            \hline
            \multirow{2}{*}{}& \multicolumn{2}{c|}{1 лицо}         & \multicolumn{2}{c|}{2 лицо}          & \multicolumn{2}{c|}{3 лицо}                  \\ \cline{2-7} 
                             & \multicolumn{1}{c|}{Ед. ч.}& Мн. ч. & \multicolumn{1}{c|}{Ед. ч.} & Мн. ч. & \multicolumn{1}{c|}{Ед. ч.}         & Мн. ч. \\ \hline
            Относительная ф. & \multicolumn{1}{c|}{my}    & our    & \multicolumn{1}{c|}{your}   & your   & \multicolumn{1}{c|}{his, her, its}  & their  \\ \hline
            Абсолютная ф.    & \multicolumn{1}{c|}{mine}  & ours   & \multicolumn{1}{c|}{yours}  & yours  & \multicolumn{1}{c|}{his, hers, its} & theirs \\ \hline
        \end{tabular}
    \end{center}
\end{enumerate}

he, him, his представляют личности мужского пола.

she, her, hers представляют личности женского пола.

it, its представляют не личности.

Личное (притяжательное) местоимение ставится в относительную форму,
если оно зависит от объекта принадлежности.

Личное (притяжательное) местоимение ставится в абсолютную форму,
если оно не зависит от объекта принадлежности.

\section{Возвратные местоимения}
\begin{center}
    \begin{tabular}{|c|c|c|}
        \hline

        & Ед. ч. & Мн. ч.
        \\
        \hline

        1 лицо & myself & ourselves
        \\
        \hline

        2 лицо & yourself & yourselves
        \\
        \hline

        3 лицо & himself, herself, itself & themselves
        \\
        \hline
    \end{tabular}
\end{center}

\section{Вопросительные местоимения}
what, which, who (whom), whose,
where, when, why, how.

\section{Относительные местоимения}
which, who (whom), whose,
where, when, why, how.

that может заменять which и who в неформальной речи.

Относительные местоимения, являющиеся вопросительными членами предложения, \\
можно опускать.

\section{Неопределённые местоимения}
\begin{center}
    \begin{tabular}{|l|l|l|l|}
    \hline
          & some      & any      & every      \\ \hline
    body  & somebody  & anybody  & everybody  \\ \hline
    one   & someone   & anyone   & everyone   \\ \hline
    thing & something & anything & everything \\ \hline
    where & somewhere & anywhere & everywhere \\ \hline
    \end{tabular}
\end{center}

\textbf{И:} all, one, each, other, others, another,
both, either,
\\
little, a little, much,
\\
few, a few, several, many,
\\
a lot of, plenty of, enough.

Производные от body и one эквивалентны.

Производные от some используются только в утвердительных предложениях, \\
просьбах и предложениях.

Производные от any, кроме any в значении любой, используется в отрицательных
и вопросительных предложениях.

little, a little, much используются только с неисчисляемым.

few, a few, several, many используются только с исчисляемым.

\section{Отрицательные местоимения}
no, none, nobody, no one, nothing, nowhere, neither.

no - относительная форма, none - абсолютная форма.

\section{Указательные местоимения}
this, these, that, those, such, the same.

\chapter{Синтаксис}
+ между словами  означает, что между этими словами нет других слов.

{}-{} между словами  означает, что между этими словами могут быть другие слова.

Дополнения ставятся после слов, от которых они зависят.

\section{Прямой и обратный порядки слов}
Прямой порядок слов: подлежащие - сказуемые.

Обратный порядок слов: вспомогательный глагол сказуемого - подлежащие -
смысловой глагол сказуемого.

\section{Типы вопросительных предложений}
Если есть в вопросе did, does, у (смыслового) глагола первый вид.

\subsection{Общий вопрос}
Обратный порядок слов.

\subsection{Специальный вопрос}
Вопросительное словосочетание, ответом на которое не является подлежащее
+ обратный порядок слов - предлог от сказуемого.

\subsection{Вопрос к подлежащему}
Вопросительное слово, ответом на которое является подлежащее,
заменяет в прямом порядке слов подлежащее.

\subsection{Альтернативный вопрос (вопрос с выбором)}
Строение, как у общего вопроса, но с акцентом на обязательную конструкцию,
выражающую альтернативные варианты.

\subsection{Вопрос с хвостиком (разделительный вопрос)}
Прямой порядок слов - хвостик.

Хвостик - это вспомогательный глагол + личное местоимение, заменяющее подлежащее.

Если до хвостика не отрицательное предложение, то
к вспомогательному глаголу добавляется not.

Если вспомогательным глаголом в части предложения до хвостика является am,
то вспомогательным глаголом в хвостике является are.

Если вспомогательным глагол в части предложения до хвостика является let's,
то вспомогательным глаголом в хвостике является shall.

Если подлежащим является относительное или отрицательное местоимение,
то личным местоимением, заменяющим подлежащее, в хвостике является they.

\section{Придаточные предложения}
В придаточном условия, времени, нет будущего. Вместо него настоящее.

Если и без придаточного определительного понятно, о чём идёт речь,
то запятая ставится, иначе нет.

Если придаточная часть стоит до главного, ставится запятая.

\chapter{Времена}
\section{Present simple}
\subsection{Образование}
Берётся первая форма глагола. Добавляется окончание s (es),
если глагол 3 лица единственного числа. Если нужно добавить окончание и
первая форма глагола оканчивается на согласная + y, y заменяется на i и
добавляется окончание es.

\subsection{Когда используется?}
Регулярные действия.

Действия, которые произойдут согласно расписания.

В инструкциях вместо повелительного наклонения.

Какие-то необычные случаи по типу заголовков газет и рассказов в интересной форме.

\section{Present continuous}
\subsection{Образование}
Берётся служебный глагол to be в present. После ставится первая форма глагола.
Добавляется окончание ing.

\subsection{Когда используется?}
Действие происходит прямо сейчас.

Действие развивается, постоянно меняется.

Запланированное действие, которое обязательно произойдёт.

Возмущение тем, что постоянно случается.

 Что-то новое противопоставляется (может и неявно) старому - новое в present continuous.

Придаточное предложение времени, когда в главной части present simple.

\section{Present perfect}
\subsection{Образование}
Берётся третья форма глагола. Добавляется служебный глагол have в present.

\subsection{Когда используется?}
Важен факт совершения действия, а не время его совершения, притом что действие связано с настоящим.

Период времени, который ещё не закончен.

Два глагола, в которых важны факт совершения, но второй дополняет первый -
первый в present perfect, второй в present simple.

\section{Past continuous}
\subsection{Образование}
То же, что и в present continuous, но в past.

\subsection{Когда употребляется?}
То же, что и в present continuous, но в past.

\section{Past perfect}
\subsection{Образование}
То же, что и в present perfect, но в past.

\subsection{Когда употребляется?}
Действие произошло до другого действия в прошлом.

\section{Future simple}
\subsection{Образование}
Берётся первая форма глагола. Перед глаголом добавляется will.

\subsection{Когда употребляется?}
Предсказание, не основанное на доказательстве.

Факт.

Решение, принятое в момент речи.

\section{Be going to}
\subsection{Образование}
Берётся первая форма глагола to be в present. После добавляется going to.
После добавляется смысловой глагол в неопределённой форме.

\subsection{Когда употребляется?}
Мы хотим сделать что-то в будущем.

Предсказание, основанное на доказательстве.

\section{Future continuous}
\subsection{Образование}
Берётся will be. После ставится глагол в первой форме. Добавляется окончание ing.

\subsection{Когда употребляется?}
Длительные действия в будущем.

Мы хотим, чтобы что-то произошло, и спрашиваем, произойдёт ли это.

Объяснение причины отказа.

\section{Future perfect}
\subsection{Образование}
То же, что и в present perfect, но в future.

\subsection{Когда употребляется?}
Важен факт совершения действия, а не его время.

\chapter{Орфография какая-то}
\section{Удвоение согласных}
Если изначальное слово заканчивается на гласную + согласную не x и w, конечная
согласная удваивается перед ing, ed, er, est,
в тех случаях когда:
\begin{enumerate}
    \item В изначальном слово представляет согласная + гласная + согласная.
    \item Изначальное слово оканчивается на l.
    \item Если последняя гласная ударная.
\end{enumerate}

\section{Добавление ing}
Если изначальный глагол заканчивается на согласная + e, e отбрасывается.

Если изначальный глагол заканчивается на ie, ie заменяется на y.

Удвоение согласных.

\chapter{Модальные глаголы и типо модальные глаголы}
\section{Could}
Способность в past.

Уверенность на 20\% в past и future.

В любом случае это general action.

\section{Might}
Официальное разрешение в past.

Уверенность на 30\% в past и в future.

\section{Might not}
Неспособность в past и future.

\section{May}
Официальное разрешение в present.

Уверенность на 40\% в present и future.

\section{Can}
Способность в present.

Уверенность на 50\% в present.

\section{Can't}
Неспособность в present.

Запрет.

\section{Be able to}
Способность.

\section{Was\textbackslash were able to}
Single action.

\section{Be allowed}
Разрешение.

\section{Must}
Долженствование от говорящего в present.

Уверенность на 90\% в present и future.

\section{Mustn't}
Запрет.

\section{Have to}
Долженствование от обстоятельств.

Уверенность на 90\% в past.

В любом случае have to не сам себе паровоз.

\section{Doesn't have to}
Отсутствие долженствования от обстоятельств.

\section{Should}
Совет.

\section{Ought to}
Формальный совет.

Совет, так как принято в обществе.

\section{Be to}
Долженствование по плану.

Приказ.

\section{Need}

\section{Need to}
Need to не сам себе паровоз.

\chapter{Other}
Наречия имеют суффикс ly.
\end{document}
