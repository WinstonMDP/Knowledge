\documentclass[oneside]{book}

\usepackage[utf8]{inputenc}
\usepackage[T2A]{fontenc}
\usepackage[russian]{babel}
\usepackage[left = 0.3\textwidth, right = 0.3\textwidth]{geometry}
\usepackage{parskip}
\usepackage[fleqn]{amsmath}
\usepackage{multirow}

\title{Английский язык}
\date{\today}
\author{Мы}

\begin{document}
	\maketitle

	\tableofcontents

	\chapter{Падежи}
	\begin{enumerate}
		\item \textbf{Subjective.}
		\begin{enumerate}
			\item Слово является подлежащим.
			\item Слово не изменяется.
		\end{enumerate}

		\item \textbf{Objective.}
		\begin{enumerate}
			\item Слова всех частей речи, кроме личных
			местоимений, не изменяются.
		\end{enumerate}

		\item \textbf{Possessive.}
		\begin{enumerate}
			\item Слово используется для обозначения
			принадлежности.

			\item Употребляется в существительных
			только одушевлённых.

			\item К существительному или неопределённому
			местоимению добавляется апостроф
			и окончание "s".

			\item Если окончание "s" не имени собственного
			уже присутствует, то добавляется
			только апостроф после окончания.

			\item Если у объекта, представляющего
			главное слово, несколько владельцев,
			представляющих цепочку слов зависящих от этого главного слова,
			окончание "s" добавляется
			только к последнему слову в цепочке.

			\item Слово, от которого зависит
			слово в прятяжательном падеже,
			может опускаться.
		\end{enumerate}

		Неодушевлённые существительные,
		с которыми употребляется
		притяжательный падеж:
		\begin{enumerate}
			\item Группы людей, организации.
			\item Временные отрезки, расстояния.
			\item Географические названия.
			\item Стредства передвижения.
			\item Объекты природы уникальные явления.
			\item Праздники.
			\item В устойчивых выражениях.
		\end{enumerate}
	\end{enumerate}

	\chapter{Местоимения}
	\section{Личные местоимения (и притяжательные).}
	\begin{enumerate}
		\item \textbf{Subjective case.}
		\begin{center}
			\begin{tabular}{|c|c|c|}
				\hline

				& Ед. ч. & Мн. ч.
				\\
				\hline

				1 лицо & I & we
				\\
				\hline

				2 лицо & you & you
				\\
				\hline

				3 лицо & he, she, it & they
				\\
				\hline
			\end{tabular}
		\end{center}

		\item \textbf{Objective case.}
		\begin{center}
			\begin{tabular}{|c|c|c|}
				\hline

				& Ед. ч. & Мн. ч.
				\\
				\hline

				1 лицо & me & us
				\\
				\hline

				2 лицо & you & you
				\\
				\hline

				3 лицо & him, her, it & them
				\\
				\hline
			\end{tabular}
		\end{center}

		\item \textbf{Possessive case.}
		\begin{center}
			\begin{tabular}{|c|cc|cc|cc|}
				\hline
				\multirow{2}{*}{}& \multicolumn{2}{c|}{1 лицо}         & \multicolumn{2}{c|}{2 лицо}          & \multicolumn{2}{c|}{3 лицо}                  \\ \cline{2-7} 
								 & \multicolumn{1}{c|}{Ед. ч.}& Мн. ч. & \multicolumn{1}{c|}{Ед. ч.} & Мн. ч. & \multicolumn{1}{c|}{Ед. ч.}         & Мн. ч. \\ \hline
				Относительная ф. & \multicolumn{1}{c|}{my}    & our    & \multicolumn{1}{c|}{your}   & your   & \multicolumn{1}{c|}{his, her, its}  & their  \\ \hline
				Абсолютная ф.    & \multicolumn{1}{c|}{mine}  & ours   & \multicolumn{1}{c|}{yours}  & yours  & \multicolumn{1}{c|}{his, hers, its} & theirs \\ \hline
			\end{tabular}
		\end{center}
	\end{enumerate}

	\begin{enumerate}
		\item he, him, his представляют личности мужского пола.
		
		\item she, her, hers представляют личности женского пола.
		
		\item it, its представляют не личности.
		
		\item Личное (притяжательное) местоимение
		ставится в относительную форму,
		если оно зависит от объекта
		принадлежности.

		\item Личное (притяжательное) местоимение
		ставится в абсолютную форму,
		если оно не зависит от объекта
		принадлежности.
	\end{enumerate}

	\section{Возвратные местоимения.}
	\begin{center}
		\begin{tabular}{|c|c|c|}
			\hline

			& Ед. ч. & Мн. ч.
			\\
			\hline

			1 лицо & myself & ourselves
			\\
			\hline

			2 лицо & yourself & yourselves
			\\
			\hline

			3 лицо & himself, herself, itself & themselves
			\\
			\hline
		\end{tabular}
	\end{center}

	\section{Вопросительные местоимения.}
	what, which, who (whom), whose,
	where, when, why, how.

	\section{Относительные местоимения.}
	which, who (whom), whose,
	where, when, why, how.

	\begin{enumerate}
		\item "that" может заменять "which" и "who"
		в неформальной речи.
		
		\item Относительные местоимения,
		являющиеся вопросительными членами
		предложения, можно опускать.
	\end{enumerate}

	\section{Неопределённые местоимения.}
	\begin{center}
		\begin{tabular}{|l|l|l|l|}
		\hline
			  & some      & any      & every      \\ \hline
		body  & somebody  & anybody  & everybody  \\ \hline
		one   & someone   & anyone   & everyone   \\ \hline
		thing & something & anything & everything \\ \hline
		where & somewhere & anywhere & everywhere \\ \hline
		\end{tabular}
	\end{center}

	\textbf{И:} all, one, each, other, others, another,
	both, either,
	\\
	little, a little, much,
	\\
	few, a few, several, many,
	\\
	a lot of, plenty of, enough.

	\begin{enumerate}
		\item Производные от "body" и "one" эквивалентны.
		
		\item Производные от "some" используется только в
		утвердительных предложениях, просьбах и предложениях.

		\item Производные от "any"{}, кроме "any" в значении
		"любой"{}, используется в отрицательных
		и вопросительных предложениях.

		\item "little"{}, "a little"{}, "much" используются только
		с неисчисляемым.

		\item "few"{}, "a few"{}, "several"{}, "many" используются
		только с исчисляемым.
	\end{enumerate}

	\section{Отрицательные местоимения.}
	no, none, nobody, no one, nothing, nowhere,
	neither.

	\begin{enumerate}
		\item no - относительная форма,
		none - абсолютная форма.
	\end{enumerate}

	\section{Указательные местоимения.}
	this, these, that, those, such, the same.

	\chapter{Синтаксис}
	\section{Порядок слов в предложении}
	\subsection{Прямой порядок слов}
	\begin{enumerate}
		\item Подлежащие.
		
		\item Сказуемые.
	\end{enumerate}

	Это скелет прямого порядка слов, который может произвольно
	наполняться обстоятельствами, определениями и дополнениями.
	Дополнения стявятся после слов, от которых они зависят.
\end{document}