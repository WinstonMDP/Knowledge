\documentclass[oneside]{book}

\usepackage[utf8]{inputenc}
\usepackage[T2A]{fontenc}
\usepackage[russian]{babel}
\usepackage[left = 0.3\textwidth, right = 0.3\textwidth]{geometry}
\usepackage{parskip}
\usepackage[fleqn]{amsmath}
\usepackage{multirow}

\title{Английский язык}
\date{\today}
\author{Мы}

\begin{document}
	\maketitle

	\tableofcontents

	\chapter{Падежи}
	\begin{enumerate}
		\item \textbf{Subjective.}
		\begin{enumerate}
			\item Слово является подлежащим.
			\item Слово не изменяется.
		\end{enumerate}

		\item \textbf{Objective.}
		\begin{enumerate}
			\item Слова всех частей речи, кроме личных
			местоимений, не изменяются.
		\end{enumerate}

		\item \textbf{Possessive.}
		\begin{enumerate}
			\item Слово используется для обозначения
			принадлежности.

			\item Употребляется в существительных
			только одушевлённых.

			\item К существительному или неопределённому
			местоимению добавляется апостроф
			и окончание "s".

			\item Если окончание "s" не имени собственного
			уже присутствует, то добавляется
			только апостроф после окончания.

			\item Если у объекта, представляющего
			главное слово, несколько владельцев,
			представляющих цепочку слов зависящих от этого главного слова,
			окончание "s" добавляется
			только к последнему слову в цепочке.

			\item Слово, от которого зависит
			слово в possessive падеже,
			может опускаться.
		\end{enumerate}

		Неодушевлённые существительные,
		с которыми употребляется
		притяжательный падеж:
		\begin{enumerate}
			\item Группы людей, организации.
			\item Временные отрезки, расстояния.
			\item Географические названия.
			\item Средства передвижения.
			\item Объекты природы уникальные явления.
			\item Праздники.
			\item В устойчивых выражениях.
		\end{enumerate}
	\end{enumerate}

	\chapter{Местоимения}
	\section{Личные местоимения (и притяжательные)}
	\begin{enumerate}
		\item \textbf{Subjective case.}
		\begin{center}
			\begin{tabular}{|c|c|c|}
				\hline

				& Ед. ч. & Мн. ч.
				\\
				\hline

				1 лицо & I & we
				\\
				\hline

				2 лицо & you & you
				\\
				\hline

				3 лицо & he, she, it & they
				\\
				\hline
			\end{tabular}
		\end{center}

		\item \textbf{Objective case.}
		\begin{center}
			\begin{tabular}{|c|c|c|}
				\hline

				& Ед. ч. & Мн. ч.
				\\
				\hline

				1 лицо & me & us
				\\
				\hline

				2 лицо & you & you
				\\
				\hline

				3 лицо & him, her, it & them
				\\
				\hline
			\end{tabular}
		\end{center}

		\item \textbf{Possessive case.}
		\begin{center}
			\begin{tabular}{|c|cc|cc|cc|}
				\hline
				\multirow{2}{*}{}& \multicolumn{2}{c|}{1 лицо}         & \multicolumn{2}{c|}{2 лицо}          & \multicolumn{2}{c|}{3 лицо}                  \\ \cline{2-7} 
								 & \multicolumn{1}{c|}{Ед. ч.}& Мн. ч. & \multicolumn{1}{c|}{Ед. ч.} & Мн. ч. & \multicolumn{1}{c|}{Ед. ч.}         & Мн. ч. \\ \hline
				Относительная ф. & \multicolumn{1}{c|}{my}    & our    & \multicolumn{1}{c|}{your}   & your   & \multicolumn{1}{c|}{his, her, its}  & their  \\ \hline
				Абсолютная ф.    & \multicolumn{1}{c|}{mine}  & ours   & \multicolumn{1}{c|}{yours}  & yours  & \multicolumn{1}{c|}{his, hers, its} & theirs \\ \hline
			\end{tabular}
		\end{center}
	\end{enumerate}

	\begin{enumerate}
		\item he, him, his представляют личности мужского пола.

		\item she, her, hers представляют личности женского пола.

		\item it, its представляют не личности.

		\item Личное (притяжательное) местоимение
		ставится в относительную форму,
		если оно зависит от объекта
		принадлежности.

		\item Личное (притяжательное) местоимение
		ставится в абсолютную форму,
		если оно не зависит от объекта
		принадлежности.
	\end{enumerate}

	\section{Возвратные местоимения}
	\begin{center}
		\begin{tabular}{|c|c|c|}
			\hline

			& Ед. ч. & Мн. ч.
			\\
			\hline

			1 лицо & myself & ourselves
			\\
			\hline

			2 лицо & yourself & yourselves
			\\
			\hline

			3 лицо & himself, herself, itself & themselves
			\\
			\hline
		\end{tabular}
	\end{center}

	\section{Вопросительные местоимения}
	what, which, who (whom), whose,
	where, when, why, how.

	\section{Относительные местоимения}
	which, who (whom), whose,
	where, when, why, how.

	\begin{enumerate}
		\item "that" может заменять "which" и "who"
		в неформальной речи.

		\item Относительные местоимения,
		являющиеся вопросительными членами
		предложения, можно опускать.
	\end{enumerate}

	\section{Неопределённые местоимения}
	\begin{center}
		\begin{tabular}{|l|l|l|l|}
		\hline
			  & some      & any      & every      \\ \hline
		body  & somebody  & anybody  & everybody  \\ \hline
		one   & someone   & anyone   & everyone   \\ \hline
		thing & something & anything & everything \\ \hline
		where & somewhere & anywhere & everywhere \\ \hline
		\end{tabular}
	\end{center}

	\textbf{И:} all, one, each, other, others, another,
	both, either,
	\\
	little, a little, much,
	\\
	few, a few, several, many,
	\\
	a lot of, plenty of, enough.

	\begin{enumerate}
		\item Производные от "body" и "one" эквивалентны.

		\item Производные от "some" используется только в
		утвердительных предложениях, просьбах и предложениях.

		\item Производные от "any"{}, кроме "any" в значении
		"любой"{}, используется в отрицательных
		и вопросительных предложениях.

		\item "little"{}, "a little"{}, "much" используются только
		с неисчисляемым.

		\item "few"{}, "a few"{}, "several"{}, "many" используются
		только с исчисляемым.
	\end{enumerate}

	\section{Отрицательные местоимения}
	no, none, nobody, no one, nothing, nowhere,
	neither.

	\begin{enumerate}
		\item no - относительная форма,
		none - абсолютная форма.
	\end{enumerate}

	\section{Указательные местоимения}
	this, these, that, those, such, the same.

	\chapter{Синтаксис}
	\begin{enumerate}
		\item "+" между словами в данном материале означает, что между этими словами нет
		других слов.

		\item "{}-{}"{} между словами в данном материале означает, что между этими словами могут быть
		другие слова.

		\item Дополнения ставятся после слов, от которых они зависят.
	\end{enumerate}

	\section{Прямой и обратный порядки слов}
	\begin{enumerate}
		\item Прямой порядок слов.
		\\
		Подлежащие - сказуемые.

		\item Обратный порядок слов.
		\\
		Вспомогательный глагол сказуемого - подлежащие - смысловой глагол сказуемого.
	\end{enumerate}

	\section{Типы вопросительных предложений}
	Если есть в вопросе "did"{}, "does"{}, у (смыслового) глагола первый вид. \\

	\subsection{Общий вопрос}
	Обратный порядок слов.

	\subsection{Специальный вопрос}
	Вопросительное словосочетание, ответом на которое не является подлежащее
	+ обратный порядок слов - предлог от сказуемого.

	\subsection{Вопрос к подлежащему}
	Вопросительное слово, ответом на которое является подлежащее,
	заменяет в прямом порядке слов подлежащее.

	\subsection{Альтернативный вопрос (вопрос с выбором)}
	Строение, как у общего вопроса, но с акцентом на обязательную конструкцию,
	выражающую альтернативные варианты.

	\subsection{Вопрос с хвостиком (разделительный вопрос)}
	Прямой порядок слов - хвостик.
	\\
	Хвостик - это вспомогательный глагол + личное местоимение, заменяющее подлежащее.

	\begin{enumerate}
		\item Если до хвостика не отрицательное предложение, то
		к вспомогательному глаголу добавляется "not".

		\item Если вспомогательным глаголом в части предложения до хвостика является "am"{},
		то вспомогательным глаголом в хвостике является "are".

		\item Если вспомогательным глагол в части предложения до хвостика является "let's"{},
		то вспомогательным глаголом в хвостике является "shall".

		\item Если подлежащим является относительное или отрицательное местоимение,
		то личным местоимением, заменяющим подлежащее, в хвостике является they.
	\end{enumerate}

	\chapter{Времена}
	\section{Present simple}
	\subsection{Образование}
	\begin{enumerate}
		\item Берётся первая форма глагола.

		\item Добавляется окончание "s" ("es"), если глагол 3 лица единственного числа.

		\item Если нужно добавить окончание и первая форма глагола оканчивается на согласная + "y"{},
		"y" \ заменяется на "i" и добавляется окончание "es".
	\end{enumerate}

	\subsection{Когда используется?}
	\begin{enumerate}
		\item Регулярные действия.
		\item Действия, которые произойдут согласно расписания.
		\item В инструкциях вместо повелительного наклонения.
		\item Какие-то необычные случаи по типу заголовков газет и рассказов в интересной форме.
	\end{enumerate}

	\section{Present continuous}
	\subsection{Образование}
	\begin{enumerate}
		\item Берётся служебный глагол "to be" в present.
		\item После ставится первая форма глагола.
		\item Добавляется окончание "ing".
	\end{enumerate}

	\subsection{Когда используется?}
	\begin{enumerate}
		\item Действие происходит прямо сейчас.
		\item Действие развивается, постоянно меняется.
		\item Запланированное действие, которое обязательно произойдёт.
		\item Возмущение тем, что постоянно случается.
		\item Что-то новое противопоставляется (может и неявно) старому - новое в present continuous.
		\item Придаточное предложение времени, когда в главной части present simple.
	\end{enumerate}

	\section{Present perfect}
	\subsection{Образование}
	\begin{enumerate}
		\item Берётся третья форма глагола.
		\item Добавляется служебный глагол "have" в present.
	\end{enumerate}

	\subsection{Когда используется?}
	\begin{enumerate}
		\item Важен факт совершения действия, а не время его совершения, притом что действие связано с настоящим.
		\item Период времени, который ещё не закончен.
		\item Два глагола, в которых важны факт совершения, но второй дополняет первый -
		первый в present perfect, второй в present simple.
	\end{enumerate}

	\section{Past continuous}
	\subsection{Образование}
	То же, что и в present continuous, но в past.
	\subsection{Когда употребляется?}
	То же, что и в present continuous, но в past.

	\section{Past perfect}
	\subsection{Образование}
	То же, что и в present perfect, но в past.

	\subsection{Когда употребляется?}
	\begin{enumerate}
		\item Действие произошло до другого действия в прошлом.
	\end{enumerate}

	\section{Future simple}
	\subsection{Образование}
	\begin{enumerate}
		\item Берётся первая форма глагола.
		\item Перед глаголом добавляется "will".
	\end{enumerate}

	\subsection{Когда употребляется?}
	\begin{enumerate}
		\item Предсказание, не основанное на доказательстве.
		\item Факт.
		\item Решение, принятое в момент речи.
	\end{enumerate}

	\section{Be going to}
	\subsection{Образование}
	\begin{enumerate}
		\item Берётся первая форма глагола "to be" в present.
		\item После добавляется "going to".
		\item После добавляется смысловой глагол в неопределённой форме.
	\end{enumerate}

	\subsection{Когда употребляется?}
	\begin{enumerate}
		\item Мы хотим сделать что-то в будущем.
		\item Предсказание, основанное на доказательстве.
	\end{enumerate}

	\section{Future continuous}
	\subsection{Образование}
	\begin{enumerate}
		\item Берётся "will be".
		\item После ставится глагол в первой форме.
		\item Добавляется окончание "ing".
	\end{enumerate}

	\subsection{Когда употребляется?}
	\begin{enumerate}
		\item Длительные действия в будущем.
		\item Мы хотим, чтобы что-то произошло, и спрашиваем, произойдёт ли это.
		\item Объяснение причины отказа.
	\end{enumerate}

	\section{Future perfect}
	\subsection{Образование}
	То же, что и в present perfect, но в future.

	\subsection{Когда употребляется?}
	\begin{enumerate}
		\item Важен факт совершения действия, а не его время.
	\end{enumerate}

	\chapter{Орфография какая-то}
	\section{Удвоение согласных}
	Если изначальное слово заканчивается на гласную + согласную не "x" и "w", конечная
	согласная удваивается перед "ing"{}, "ed"{}, "er"{}, "est"{},
	в тех случаях когда:
	\begin{enumerate}
		\item В изначальном слово представляет согласная + гласная + согласная.
		\item Изначальное слово оканчивается на "l".
		\item Если последняя гласная ударная.
	\end{enumerate}

	\section{Добавление ing}
	\begin{enumerate}
		\item Если изначальный глагол заканчивается на согласная + "e"{}, "e" отбрасывается.
		\item Если изначальный глагол заканчивается на "ie"{}, "ie" заменяется на "y".
		\item Удвоение согласных.
	\end{enumerate}

	\chapter{Модальные глаголы и типо модальные глаголы}
	\section{Might}
	\begin{enumerate}
		\item Официальное разрешение в past и в present.
		\item Possibility в future.
	\end{enumerate}

	\section{Might not}
	\begin{enumerate}
		\item Negative possibility.
	\end{enumerate}

	\section{Could}
	\begin{enumerate}
		\item Возможность сделать что-то в past.
		\item Способность в past.
		\item Уверенность на 30\%.
		\item Possibility в future.
	\end{enumerate}

	General action

	\section{May}
	\begin{enumerate}
		\item Официальное разрешение в present.
		\item Уверенность на 40\%.
		\item Possibility в future.
	\end{enumerate}

	\section{Can}
	\begin{enumerate}
		\item Возможность сделать что-то в present.
		\item Способность в present.
		\item Уверенность на 50\%.
	\end{enumerate}

	\section{Can't}
	\begin{enumerate}
		\item Логическая невозможность.
		\item Остутствие возможности.
		\item Запрет.
	\end{enumerate}

	\section{Be able to}
	\begin{enumerate}
		\item Ability в future.
	\end{enumerate}

	\section{Be allowed}
	\begin{enumerate}
		\item Разрешение.
	\end{enumerate}

	\section{Must}
	\begin{enumerate}
		\item Долженствование от говорящего в present.
		\item Уверенность на 90\%.
	\end{enumerate}

	\section{Mustn't}
	\begin{enumerate}
		\item Запрет.
	\end{enumerate}

	\section{Have to}
	\begin{enumerate}
		\item Долженствование от обстоятельств.
	\end{enumerate}

	Have to не сам себе паровоз.

	\section{Doesn't have to}
	\begin{enumerate}
		\item Остутствие долженствования от обстоятельств.
	\end{enumerate}

	\section{Should}
	\begin{enumerate}
		\item Совет.
	\end{enumerate}

	\section{Ought to:}
	\begin{enumerate}
		\item Формальный совет.
		\item Совет, так как принято в обществе.
	\end{enumerate}

	\section{Be to}
	\begin{enumerate}
		\item Долженствование по плану.
		\item Приказ.
	\end{enumerate}
\end{document}