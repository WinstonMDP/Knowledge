\documentclass[oneside]{book}

\usepackage[utf8]{inputenc}
\usepackage[T2A]{fontenc}
\usepackage[russian]{babel}
\usepackage{parskip}
\usepackage[fleqn]{amsmath}

\setlength{\parskip}{0.03\textheight}

\title{Математика}
\date{\today}
\author{Мы}

\begin{document}
	\maketitle

	\tableofcontents

	\chapter{Логика}
	\textbf{Из посылки A вытекает вывод B:}
	\begin{flalign*}
		A\longrightarrow B
	\end{flalign*}
	A - достаточное условие для B.
	\\
	B - необходимое условие для А.

	\textbf{Эквивалентные утверждения A и B} - это утверждения,
	при которых из посылки A
	вытекает вывод B и из посылки B вытекает
	вывод A.

	\textbf{Обратное утверждениe:}
	\begin{flalign*}
		B\longrightarrow A
	\end{flalign*}
	\begin{math}
		A\longrightarrow B
	\end{math}

	\textbf{Противоположное утверждениe:}
	\begin{flalign*}
		\overline{A}\longrightarrow \overline{B}
	\end{flalign*}
	\begin{math}
		A\longrightarrow B
	\end{math}

	\textbf{Утверждение, противоположное обратному:}
	\begin{flalign*}
		\overline{B}\longrightarrow \overline{A}
	\end{flalign*}
	\begin{math}
		A\longrightarrow B
	\end{math}

	Утверждение A и утверждение, обратное
	противоположному A, эквивалентны.

	\textbf{Доказательство от противного:}
	\\
	Чтобы доказать
	\begin{math}
		A\longrightarrow B
	\end{math}
	, надо доказать
	\begin{math}
		A\land\overline{\overline{B}}
	\end{math}

	\textbf{Метод математической индукции для натуральных чисел:}
	\\
	Чтобы доказать
	\begin{math}
		f(x) = g(x)
	\end{math}
	, надо
	доказать
	\begin{math}
		f(1) = g(1)\land f(n + 1) = g(n + 1),
	\end{math}
	приняв
	\begin{math}
		f(n) = g(n)
	\end{math}
	.

	\chapter{Числа}
	\textbf{Числовое кольцо} - это множество чисел,
	результат суммы, разности, произведения любых чисел которого
	принадлежит ему тоже.

	\textbf{Числовое поле} - это множество чисел,
	результат выполнения рациональных действий над любыми
	числами которого принадлежит ему тоже.

	\section{Натуральные числа (N)}
	\textbf{Свойства сложения и умножения:}
	\begin{enumerate}
		\item Переместительное.
		\begin{flalign*}
			a + b = b + a
		\end{flalign*}
		\begin{flalign*}
			ab = ba
		\end{flalign*}

		\item Сочетательное (ассоциативное).
		\begin{flalign*}
			(a + b) + c = a + (b + c)
		\end{flalign*}
		\begin{flalign*}
			(ab)c = a(bc)
		\end{flalign*}

		\item Распределительнoe.
		\begin{flalign*}
			c(a + b) = ac + cb
		\end{flalign*}
	\end{enumerate}

	\textbf{Делитель a} - это число, на которое
	a делится без остатка.
	
	\textbf{Кратное a} - это всякое число, которое
	делится на a без остатка.

	\textbf{Простое число} - это число, не имеющее
	никаких других делителей, кроме
	единицы и себя. (
	\begin{math}
		\overline{\textbf{\text{составное число}}}
	\end{math}
	)

	Простых чисел имеется бесконечное множество.

	Разложение числа на простые множители взаимно однозначно.

	\textbf{Взаимно простые числа} - это числа,
	не имеющие общих делителей.

	\textbf{Чётное число} - это число, кратное 2. (
	\begin{math}
		\overline{\textbf{\text{нечётное число}}}
	\end{math}
	)

	Число 2 - единственное чётное простое число.

	\textbf{Признаки делимости в 10-й системе счисления:}
	\begin{enumerate}
		\item Признак делимости на 2: последняя цифра в записи числа выражает чётное число.
		\item Признак делимости на 3: сумма цифр записи числа делится на 3.
		\item Признак делимости на 4: последние две цифры в записи числа выражают число, делящееся на 4.
		\item Признак делимости на 5: последняя цифра в записи числа является 0 или 5.
		\item Признак делимости на 9: сумма цифр записи числа делится на 9.
	\end{enumerate}

	\textbf{Наибольший общий делитель (НОД) a и b:}
	\begin{flalign*}
		( a, b )
	\end{flalign*}
	
	\textbf{Наименьшее общее кратное (НОК) a и b:}
	\begin{flalign*}
		[  a, b ] 
	\end{flalign*}

	\begin{flalign*}
		( a, b ) [ a, b ] = ab
	\end{flalign*}

	\section{Целые числа (Z)}
	\begin{flalign*}
		N \in Z
	\end{flalign*}

	\textbf{Положительное число} - это число,
	большее нуля.

	\textbf{Отрицательное число} - это число,
	меньшее нуля.

	\textbf{Противоположные числа} - это числа,
	отличающиеся знаком.

	\begin{flalign*}
		a - b = a + (-b)
	\end{flalign*}
	\begin{flalign*}
		a(-b) = -ab
	\end{flalign*}

	\textbf{Абсолютная величина (модуль) x:}
	\begin{flalign*}
		\left[
			\begin{aligned}
			\left\{
				\begin{aligned}
				\left\lvert x \right\rvert = x
				\\
				x \geq 0
				\end{aligned}
			\right.
			\\
			\left\{
				\begin{aligned}
				\left\lvert x \right\rvert = -x
				\\
				x < 0
				\end{aligned}
			\right.
			\end{aligned}
		\right.
	\end{flalign*}

	\section{Рациональные числа (Q)}
	\begin{flalign*}
		Z \in Q
	\end{flalign*}

	\textbf{Рациональное число} - это число,
	представимое в виде
	\begin{math}
		\frac{a}{b}
	\end{math}
	, где числитель
	\begin{math}
		a \in Z
	\end{math}
	, а знаменатель
	\begin{math}
		b \in N
	\end{math}
	.

	Рациональные числа образуют поле.

	\textbf{Дробное число} - это рациональное число,
	числитель которого не делится на знаменатель нацело.

	\textbf{Целая часть числа} - это наибольшее
	целое число, не превосходящее данного (
	\begin{math}
		\left[x\right] 	
	\end{math}
	)
	.

	\textbf{Дробная часть числа} - это разность
	между данным числом и его целой
	частью (
	\begin{math}
		(x)
	\end{math}
	)
	.

	\begin{flalign*}
		x - \left[x\right] \geq 0
	\end{flalign*}
	\begin{flalign*}
		x - \left[x\right] < 1
	\end{flalign*}

	Разложение рационального числа
	на сумму целой и дробной частей
	взаимно однозначно.

	\textbf{Десятичная дробь} - это дробь,
	у которой знаменатель представляет
	собой натуральную степень числа 10.

	Всякое рациональное число может
	быть представленно бесконечной
	десятичной периодической дробью
	взаимно однозначно.

	\section{Иррациональные числа (I)}
	Всякое иррациональное число может
	быть представленно бесконечной
	десятичной непериодической дробью
	взаимно однозначно.

	\section{Действительные числа (R)}
	\begin{flalign*}
		Q \in R
	\end{flalign*}
	\begin{flalign*}
		I \in R
	\end{flalign*}

	Действительные числа образуют поле.

	Множество действительных чисел упорядочено.

	Множество действительных чисел непрерывно.

	Всякое деятичное число
	определяет действительное число
	взаимно однозначно.
\end{document}