\documentclass[oneside]{book}

\usepackage[utf8]{inputenc}
\usepackage[T2A]{fontenc}
\usepackage[russian]{babel}
\usepackage[left = 0.3\textwidth, right = 0.3\textwidth]{geometry}
\usepackage{parskip}
\usepackage[fleqn]{amsmath}

\setlength{\parskip}{0.03\textheight}

\title{Математика}
\date{\today}
\author{Мы}

\begin{document}
	\maketitle

	\tableofcontents

	\chapter{Логика}
	\textbf{Из посылки A вытекает вывод B:}
	\begin{flalign*}
		A\longrightarrow B
	\end{flalign*}
	A - достаточное условие для B.
	\\
	B - необходимое условие для А.

	\textbf{Эквивалентные утверждения A и B} - это утверждения,
	при которых из посылки A
	вытекает вывод B и из посылки B вытекает
	вывод A.

	\textbf{Обратное утверждениe:}
	\begin{flalign*}
		B\longrightarrow A
	\end{flalign*}
	\begin{math}
		A\longrightarrow B
	\end{math}

	\textbf{Противоположное утверждениe:}
	\begin{flalign*}
		\overline{A}\longrightarrow \overline{B}
	\end{flalign*}
	\begin{math}
		A\longrightarrow B
	\end{math}

	\textbf{Утверждение, противоположное обратному:}
	\begin{flalign*}
		\overline{B}\longrightarrow \overline{A}
	\end{flalign*}
	\begin{math}
		A\longrightarrow B
	\end{math}

	Утверждение A и утверждение, обратное
	противоположному A, эквивалентны.

	\textbf{Доказательство от противного:}
	\\
	Чтобы доказать
	\begin{math}
		A\longrightarrow B
	\end{math}
	, надо доказать
	\begin{math}
		A\land\overline{\overline{B}}
	\end{math}

	\textbf{Метод математической индукции для натуральных чисел:}
	\\
	Чтобы доказать
	\begin{math}
		f(x) = g(x)
	\end{math}
	, надо
	доказать
	\begin{math}
		f(1) = g(1)\land f(n + 1) = g(n + 1),
	\end{math}
	приняв
	\begin{math}
		f(n) = g(n)
	\end{math}
	.

	\chapter{Алгебраические выражения}
	\textbf{Алгебраическое выражение} - это
	выражение, состоящее из чисел,
	буквенных величин и алгебраических
	операций над ними.
	
	\textbf{Область допустимых значений (ОДЗ)} - это
	множество всех наборов числовых
	значений букв, входящих
	в данное алгебраическое выражение.

	\textbf{Тождественно равные алгебраические выражения} - это
	алгебраические выражения, имеющие равные ОДЗ и равные
	числовые значения на этом ОДЗ.
	
	\textbf{Одночлен} - это
	алгебраическое выражение,
	состоящее из произведения числового коэффициента
	и буквенных величин.

	\textbf{Стандартный вид одночлена:}
	\begin{enumerate}
		\item Один числовой коэффициент.
		\item Нет повторяющихся буквенных величин.
	\end{enumerate}

	\textbf{Подобные одночлены} - это
	одночлены, отличающиеся только числовыми
	коэффициентами.

	\textbf{Многочлен (полином)} - это
	алгебраическое выражение,
	состоящее из суммы одночленов.

	\textbf{Стандартный вид многочлена:}
	\begin{enumerate}
		\item Все одночлены стандартного вида.
		\item Нет подобных одночленов.
	\end{enumerate}

    \textbf{Формулы сокращённого умножения:}
    \begin{enumerate}
        \item Квадрат суммы.
        \\
        \begin{math}
            (a + b)^2 = a^2 + 2ab + b^2
        \end{math}

        \item Разность квадратов.
        \\
        \begin{math}
            a^2 - b^2 = (a - b)(a + b)
        \end{math}
        
        \item Куб суммы.
        \\
        \begin{math}
            (a + b)^3 = a^3 + 3a^2 b + 3ab^2 + b^3
        \end{math}

        \item Сумма кубов.
        \\
        \begin{math}
            a^3 + b^3 = (a + b)(a^2 - ab + b^2)
        \end{math}

        \item Бином Ньютона.
        \\
        \begin{math}
            (a + b)^n = 
			\sum\limits_{i = 0}^n \frac{a^{n-i}b^in!}{i!(n - i)!} =
			\sum\limits_{i = 0}^n \frac{a^{n-i}b^i\prod\limits_{k = 0}^{i - 1}n - k}{i!}
        \end{math}
    \end{enumerate}

    \textbf{Неполный квадрат разности:}
    \begin{flalign*}
        a^2 - ab + b^2
    \end{flalign*}

	Многочлен
	\begin{math}
		Q(x)
	\end{math}
	является частным и многочлен
	\begin{math}
		R(x)
	\end{math}
	является остатком при делении многочлена 
	\begin{math}
		P_n(x)
	\end{math}
	на 
	\begin{math}
		S_m(x)
	\end{math}
	, если
	\begin{math}
		P_n(x) = S_m(x)Q(x) + R(x)
	\end{math}
	и степень
	\begin{math}
		R(x)
	\end{math}
	меньше степени
	\begin{math}
		S_m(x)
	\end{math}
	.

	Если
	степень
	\begin{math}
		P_n(x)
	\end{math}
	больше степени
	\begin{math}
		S_m(x)
	\end{math}
	, степень частного от деления
	\begin{math}
		P_n(x)
	\end{math}
	на
	\begin{math}
		S_m(x)
	\end{math}
	равна разности степеней
	\begin{math}
		P_n(x)
	\end{math}
	и
	\begin{math}
		S_m(x)
	\end{math}
	, иначе частное равно нулю.

	Остаток от деления многочлена
	\begin{math}
		P_n(x)
	\end{math}
	на двучлен вида
	\begin{math}
		x - \alpha 
	\end{math}
	равен значению многочлена при
	\begin{math}
		x = \alpha 
	\end{math}
	.

	\textbf{Теорема Безу:}
	\\
	Многочлен
	\begin{math}
		P_n(x)
	\end{math}
	делится без остатка на двучлен
	\begin{math}
		x - \alpha
	\end{math}
	, только если
	\begin{math}
		\alpha 
	\end{math}
	- корень многочлена.

	\chapter{Измерения}
	\textbf{Величина} - это
	объект, который может быть охарактеризован числом в результате
	измерения.

	\textbf{Постоянная величина} - это
	величина, множество значений которой
	состоит из одного элемента.

	\textbf{Переменная величина} - это
	величина, множество значений которой
	состоит более чем из одного элемента.

	\textbf{Область изменения} - это
	множество значений, принимаемых переменной
	величиной.
	
	\chapter{Функции}
	\textbf{Функция}
	\begin{math}
		\mathbf{y = f(x)}
	\end{math}
	- это переменная y,
	при которой каждому значению x
	из области изменения x поставлено
	в соответствие по определённому закону
	значение y.

	\begin{flalign*}
		y = f(x)
	\end{flalign*}
	y - функция (зависимая переменная).
	\\
	x - аргумент (независимая переменная).
	\\
	\begin{math}
		f(x) 
	\end{math}
	- закон соответствия между y и x.

	\textbf{Область определения функции, задаваемой законом соответствия}
	\begin{math}
		\mathbf{f(x)}
	\end{math}
	,
	\begin{math}
		\mathbf{D(f)}
	\end{math}
	- это область изменения независимой переменной функции.

	\textbf{Область значений функции, задаваемой законом соответствия}
	\begin{math}
		\mathbf{f(x)}
	\end{math}
	,
	\begin{math}
		\mathbf{E(f)}
	\end{math}- это
	область изменения функции.

	\textbf{Математический анализ} - это
	область математики, иследующая функции.

	\textbf{Равные (совпадающие) функции} - это
	функции, область определения которых равны
	и значения которых при любых одинаковых
	значениях аргумента равны.

	\textbf{Нуль (корень) функции} - это
	значение независимой переменной,
	при котором значение функции равно нулю.

	\textbf{Чётная функция:}
	\begin{flalign*}
		f(-x) = f(x)
	\end{flalign*}
	\begin{math}
		\left[
			\begin{aligned}
				&D(f) = (-a; a)
				\\
				&D(f) = [-a; a]
			\end{aligned}
		\right.
	\end{math}
	
	\textbf{Нечётная функция:}
	\begin{flalign*}
		f(-x) = -f(x)
	\end{flalign*}
	\begin{math}
		\left[
			\begin{aligned}
				&D(f) = (-a; a)
				\\
				&D(f) = [-a; a]
			\end{aligned}
		\right.
	\end{math}

	\textbf{Функция общего вида} - это
	функция, которая не является
	чётной или нечётной.

	\textbf{Возрастающая функция на некотором промежутке (интервале)} - это
	функция, для любых двух значений независимой переменной
	\begin{math}
		x_1
	\end{math}
	и
	\begin{math}
		x_2
	\end{math}
	которой на этом промежутке
	из неравенства
	\begin{math}
		x_1 < x_2
	\end{math}
	следует
	\begin{math}
		f(x_1) < f(x_2)
	\end{math}
	при том, что этот интервал
	лежит в её области определения. (
	\begin{math}
		\overline{\textbf{\text{невозрастающая функция на некотором промежутке (интервале)}}}
	\end{math}
	)

	\textbf{Убывающая функция на некотором промежутке (интервале)} - это
	функция, для любых двух значений независимой переменной
	\begin{math}
		x_1
	\end{math}
	и
	\begin{math}
		x_2
	\end{math}
	которой на этом промежутке
	из неравенства
	\begin{math}
		x_1 < x_2
	\end{math}
	следует
	\begin{math}
		f(x_1) > f(x_2)
	\end{math}
	при том, что этот интервал
	лежит в её области определения. (
	\begin{math}
		\overline{\textbf{\text{неубывающая функция на некотором промежутке (интервале)}}}
	\end{math}
	)

	\textbf{Интервал монотонности} - это
	интервал, на котором функция возрастает
	или убывает.

	\textbf{Точка минимума функции, задаваемой законом соответствия}
	\begin{math}
		\mathbf{f(x)}
	\end{math}
	,- это точка функции
	\begin{math}
		x_0
	\end{math}
	, определённой в самой этой точке и
	в некоторой окрестности точки
	\begin{math}
		x_0
	\end{math}
	при том, что множество значений этой окрестности
	равно области изменения
	\begin{math}
		x'
	\end{math}
	,
	для которой выполняется
	неравенство
	\begin{math}
		f(x_0) \leq f(x')
	\end{math}
	.

	\textbf{Точка максимума функции, задаваемой законом соответствия}
	\begin{math}
		\mathbf{f(x)}
	\end{math}
	,- это точка функции
	\begin{math}
		x_0
	\end{math}
	, определённой в самой этой точке и
	в некоторой окрестности точки
	\begin{math}
		x_0
	\end{math}
	при том, что множество значений этой окрестности
	равно области изменения
	\begin{math}
		x'
	\end{math}
	,
	для которой выполняется
	неравенство
	\begin{math}
		f(x_0) \geq f(x')
	\end{math}
	.

	\textbf{Экстремум} - это точка
	минимума или максимума функции.

	\textbf{Асимптота} - это прямая линия,
	к которой график функции неограниченно
	приближается при удалении точки
	графика в бесконечность.

	\textbf{Исследование функции:}
	\begin{enumerate}
		\item Область определения функции.
		\item Область значений функции.
		\item Нули функции.
		\item Чётная, или нечётная, или общего вида функция.
		\item Интервалы монотонности функции.
		\item Экстремумы функции.
		\item Асимптоты функции.
	\end{enumerate}

	\textbf{Сложная функция} - это
	функция, аргумент которой равен
	другой функции.

	\textbf{Обратная функция}
	\begin{math}
		\mathbf{y = f(x)}
	\end{math}
	\textbf{функции}
	\begin{math}
		\mathbf{u = g(x)}
	\end{math}
	\textbf{:}
	\begin{flalign*}
		f(g(a)) = a
	\end{flalign*}
	\begin{math}
		a \in D(g)
		\\
		g(a) \in D(f)
	\end{math}

	\textbf{Алгебраическая функция} - это
	функция, закон соответствия которой
	определяется алгебраическим выражением. (
	\begin{math}
		\overline{\textbf{трансцендентная функция}}
	\end{math}
	)

	\textbf{Элементарные функции} - это
	основные элементарные функции и
	сложные функции,
	образованные из основных элементарных.

	\textbf{Основные элементарные функции:}
	\begin{enumerate}
		\item Степенные.
		\\
		\begin{math}
			f(x) = x^k
			\\
			k \in R
		\end{math}

		\item Показательные.
		\\
		\begin{math}
			f(x) = a^x
			\\
			a \in R
			\\
			a > 0
			\\
			a \neq 1
		\end{math}

		\item Логарифмические.
		\\
		\begin{math}
			f(x) = \log_a^x
			\\
			a \in R
			\\
			a > 0
			\\
			a \neq 1
		\end{math}

		\item Тригонометрические.
		\\
		\begin{math}
			f(x) = \sin x
		\end{math}
		\\
		или
		\\
		\begin{math}
			f(x) = \cos x
		\end{math}
		\\
		или
		\\
		\begin{math}
			f(x) = \tan x
		\end{math}
		\\
		или
		\\
		\begin{math}
			f(x) = \cot x 
		\end{math}

		\item Обратные тригонометрические функции.
		\\
		\begin{math}
			f(x) = \arcsin x
		\end{math}
		\\
		или
		\\
		\begin{math}
			f(x) = \arccos x
		\end{math}
		\\
		или
		\\
		\begin{math}
			f(x) = \arctan x
		\end{math}
		\\
		или
		\\
		\begin{math}
			f(x) = \operatorname{arccot} x
		\end{math}
	\end{enumerate}

	\textbf{Целая рациональная функция (многочлен от переменной x) (ЦРФ) степени n}:
	\begin{flalign*}
		y = P_n(x) = \sum_{i = 0}^n a_ix^{n-i}
	\end{flalign*}
	\begin{math}
		a_0 \neq 0
	\end{math}

	y прямо пропорционально x и между x и y прямо пропорциональная зависимость, если
	\begin{math}
		y = ax
	\end{math}
	, при
	\begin{math}
		a \neq 0
	\end{math}
	.

	\textbf{Линейная ЦРФ (линейная функция):}
	\begin{flalign*}
		y = ax + b
	\end{flalign*}
	\begin{math}
		a \neq 0
	\end{math}

	\textbf{Квадратичная ЦРФ (квадратный (квадратичный) трёхчлен):}
	\begin{flalign*}
		y = ax^2 + bx + c
	\end{flalign*}
	\begin{math}
		a \neq 0
	\end{math}

	y обратно пропорционально x и между x и y обратно пропорциональная зависимость, если
	\begin{math}
		y = \frac{a}{x}
	\end{math}
	, при
	\begin{math}
		a \neq 0
	\end{math}
	.

	\textbf{Дробно-рациональная функция (ДРФ):}
	\begin{flalign*}
		y = \frac{P_n(x)}{Q_m(x)}
	\end{flalign*}
	\begin{math}
		P_n(x) \text{ - закон соответствия ЦРФ.}
		\\
		Q_m(x) \text{ - закон соответствия ЦРФ.}
	\end{math}

	\textbf{Дробно-линейная функция:}
	\begin{flalign*}
		y = \frac{ax + b}{cx + d} = \frac{a}{c} + \frac{\frac{bc - ad}{c^2}}{x + \frac{d}{c}}
	\end{flalign*}
	\begin{math}
		c \neq 0
		\\
		ad - bc \neq 0
	\end{math}

	\textbf{Алгебраическая иррациональная функция} - это
	функция, закон соответствия которой содержит
	извлечение корня целой степени из алгебраического выражения,
	содержащего аргумент.

	\chapter{Числа}
	\textbf{Числовое кольцо} - это множество чисел,
	результат суммы, разности, произведения любых чисел которого
	принадлежит ему тоже.

	\textbf{Числовое поле} - это множество чисел,
	результат выполнения рациональных действий над любыми
	числами которого принадлежит ему тоже.

	\section{Натуральные числа (N)}
	\textbf{Целочисленная переменная} - это
	величина, принимающая только
	натуральные значения.

	\textbf{Свойства сложения и умножения:}
	\begin{enumerate}
		\item Переместительное.
		\begin{flalign*}
			a + b = b + a
		\end{flalign*}
		\begin{flalign*}
			ab = ba
		\end{flalign*}

		\item Сочетательное (ассоциативное).
		\begin{flalign*}
			(a + b) + c = a + (b + c)
		\end{flalign*}
		\begin{flalign*}
			(ab)c = a(bc)
		\end{flalign*}

		\item Распределительнoe.
		\begin{flalign*}
			c(a + b) = ac + cb
		\end{flalign*}
	\end{enumerate}

	q является частным и r является остатком при делении а на b,
	если
	\begin{math}
		a = bq + r
	\end{math}
	и
	\begin{math}
		r < b
	\end{math}
	.

	\textbf{Делитель a} - это число, на которое
	a делится без остатка.
	
	\textbf{Кратное a} - это всякое число, которое
	делится на a без остатка.

	\textbf{Простое число} - это число, не имеющее
	никаких других делителей, кроме
	единицы и себя. (
	\begin{math}
		\overline{\textbf{\text{составное число}}}
	\end{math}
	)

	Простых чисел имеется бесконечное множество.

	Разложение числа на простые множители взаимно однозначно.

	\textbf{Взаимно простые числа} - это числа,
	не имеющие общих делителей.

	\textbf{Чётное число} - это число, кратное 2. (
	\begin{math}
		\overline{\textbf{\text{нечётное число}}}
	\end{math}
	)

	Число 2 - единственное чётное простое число.

	\textbf{Признаки делимости в 10-й системе счисления:}
	\begin{enumerate}
		\item Признак делимости на 2: последняя цифра в записи числа выражает чётное число.
		\item Признак делимости на 3: сумма цифр записи числа делится на 3.
		\item Признак делимости на 4: последние две цифры в записи числа выражают число, делящееся на 4.
		\item Признак делимости на 5: последняя цифра в записи числа является 0 или 5.
		\item Признак делимости на 9: сумма цифр записи числа делится на 9.
	\end{enumerate}

	\textbf{Наибольший общий делитель (НОД) a и b:}
	\begin{flalign*}
		( a, b )
	\end{flalign*}
	
	\textbf{Наименьшее общее кратное (НОК) a и b:}
	\begin{flalign*}
		[  a, b ] 
	\end{flalign*}

	\begin{flalign*}
		( a, b ) [ a, b ] = ab
	\end{flalign*}

	\section{Целые числа (Z)}
	\begin{flalign*}
		N \in Z
	\end{flalign*}

	\textbf{Целое алгебраическое выражение} - это
	алгебраическое выражение, в котором
	используют только сложение, вычитание, умножение.

	\textbf{Положительное число} - это число,
	большее нуля.

	\textbf{Отрицательное число} - это число,
	меньшее нуля.

	\textbf{Противоположные числа} - это числа,
	отличающиеся знаком.

	\begin{flalign*}
		a - b = a + (-b)
	\end{flalign*}
	\begin{flalign*}
		a(-b) = -ab
	\end{flalign*}

	
	\begin{flalign*}
		\left[
			\begin{aligned}
			&\left\{
				\begin{aligned}
				&\left\lvert x \right\rvert = x
				\\
				&x \geq 0
				\end{aligned}
			\right.
			\\
			&\left\{
				\begin{aligned}
				&\left\lvert x \right\rvert = -x
				\\
				&x < 0
				\end{aligned}
			\right.
			\end{aligned}
		\right.
	\end{flalign*}

	\section{Рациональные числа (Q)}
	\begin{flalign*}
		Z \in Q
	\end{flalign*}

	\textbf{Рациональное число} - это число,
	представимое в виде
	\begin{math}
		\frac{a}{b}
	\end{math}
	, где числитель
	\begin{math}
		a \in Z
	\end{math}
	, а знаменатель
	\begin{math}
		b \in N
	\end{math}
	.

	Рациональные числа образуют поле.

	\textbf{Арифметические (рациональные) действия:}
	сложение, вычитание, умножение, деление.

	\textbf{Рациональное алгебраическое выражение} - это
	алгебраическое выражение, в котором используют
	только рациональные действия.

	\textbf{Дробное алгебраическое выражение} - это
	рациональное алгебраическое выражение, в
	записи которого используют деление на буквенные
	выражения.

	\textbf{Алгебраическая дробь} - это
	это алгебраическое выражение, имеющее вид частного
	от деления двух целых
	алгебраических выражений.

	\textbf{Дробное число} - это рациональное число,
	числитель которого не делится на знаменатель нацело.

	\textbf{Целая часть числа} - это наибольшее
	целое число, не превосходящее данного (
	\begin{math}
		\left[x\right] 	
	\end{math}
	)
	.

	\textbf{Дробная часть числа} - это разность
	между данным числом и его целой
	частью (
	\begin{math}
		(x)
	\end{math}
	)
	.

	\begin{flalign*}
		x - \left[x\right] \geq 0
	\end{flalign*}
	\begin{flalign*}
		x - \left[x\right] < 1
	\end{flalign*}

	Разложение рационального числа
	на сумму целой и дробной частей
	взаимно однозначно.

	\textbf{Десятичная дробь} - это дробь,
	у которой знаменатель представляет
	собой натуральную степень числа 10.

	Всякое рациональное число может
	быть представленно бесконечной
	десятичной периодической дробью
	взаимно однозначно.

	\section{Иррациональные числа (I)}
	Всякое иррациональное число может
	быть представленно бесконечной
	десятичной непериодической дробью
	взаимно однозначно.

	\textbf{Иррациональные алгебраические выражения} - это
	алгебраическое выражение, в записи которого
	используются знаки радикала из буквенного выражения.

	\textbf{Корень находится в простейшей форме, если:}
	\begin{enumerate}
		\item Он не содержит иррациональности в знаменателе.
		\item Нельзя сократить его показатель с показателем подкоренного выражения.
		\item Все возможные множители вынесены из-под корня.
	\end{enumerate}

	\textbf{Подобные корни} - это корни,
	отличающиеся только коэффициентами.

	\section{Действительные числа (R)}
	\begin{flalign*}
		Q \in R
	\end{flalign*}
	\begin{flalign*}
		I \in R
	\end{flalign*}

	Действительные числа образуют поле.

	Множество действительных чисел упорядочено.

	Множество действительных чисел непрерывно.

	Всякое деятичное число
	определяет действительное число
	взаимно однозначно.

	Многочлен с действительными
	коэффициентами разлагается в произведение
	линейных двучленов вида
	\begin{math}
		x - a
	\end{math}
	и квадратных трёхчленов вида
	\begin{math}
		x^2 + px + q
	\end{math}
	.

	\textbf{n-ая степень числа a} - это произведение
	n сомножителей, равных а. (
	\begin{math}
		a^n
	\end{math}
	)
	\\
	a - основание степени.
	\\
	n - показатель степени.

	Возведение отрицательного числа
	в иррациональную степень не определено.

	Возведение нуля в не положительную степень не определено.

	\begin{flalign*}
		a^x = a^y \longrightarrow x = y
	\end{flalign*}

	\textbf{Корень n-ой степени из числа a} - это
	число, n-ая степень которого равна a. (
		\begin{math}
			\sqrt[n]{a} 
		\end{math}
	)

	\textbf{Извлечение корня степени из a} - это
	отыскание корня из a.

	\textbf{Арифметический корень (арифметическое значение корня)} - это
	положительный корень чётной степени из положительного числа.

	Корень чётной степени по умолчанию aрифметический.

	\begin{flalign*}
		\left[
			\begin{aligned}
			&\left\{
				\begin{aligned}
					&\sqrt[n]{a^n} = a
					\\
					&n \text{ - нечётное.}
				\end{aligned}
			\right.
			\\
			&\left\{
				\begin{aligned}
					&\sqrt[n]{a^n} = \left\lvert a\right\rvert  
					\\
					&n \text{ - чётное.}
				\end{aligned}
			\right.
			\end{aligned}
		\right.
	\end{flalign*}

	\textbf{Квадратный корень:}
	\begin{flalign*}
		\sqrt[2]{x} = \sqrt{x}
	\end{flalign*}

	\textbf{Кубический корень:}
	\begin{flalign*}
		\sqrt[3]{x}
	\end{flalign*}

	\begin{flalign*}
		\sqrt[b]{x^a} = x^\frac{a}{b}  
	\end{flalign*}

	\textbf{Логарифм числа N по основанию a} - это
	показатель степени, в которую
	нужно возвести a, чтобы получить N.
	
	\begin{flalign*}
		\left\{
			\begin{aligned}
				&a^{\log_a N} = N
				\\
				&N > 0
				\\
				&a > 0
				\\
				&a \neq 1
			\end{aligned}
		\right.
	\end{flalign*}

	Если число и основание логарифма
	лежат по одну сторону от
	единицы, то этот логарифм
	положителен, и наоборот.

	\textbf{Потенцирование} - это
	возведение числа, от которого взят
	логарифм, в этот логарифм.

	Если основание больше единицы,
	то большее число имеет больший
	логарифм.

	Если основание меньше единицы, то
	большее число имеет меньший логарифм.

	\textbf{Десятичный логарифм} - это
	логарифм по основанию 10.

	\begin{flalign*}
		\log_{10} N = \lg N
	\end{flalign*}

	\textbf{Характеристика} - это
	целая часть десятичного логарифма.

	\textbf{Мантиса} - это
	дробная часть десятичного логарифма.
	
	\textbf{Открытый интервал (a; b)} - это
	множество действительных чисел x, удовлетворяющих
	неравествам
	\begin{math}
		a < x \leq b
	\end{math}
	.

	\textbf{Окрестность точки}
	\begin{math}
		\mathbf{
			x_0 \
			(x_0 - h; x_0 + h)
		}
	\end{math}
	- это интервал длины 2h
	серединой
	\begin{math}
		x_0
	\end{math}
	.

	\textbf{Замкнутый интервал [a; b]} - это
	множество действительных чисел x,
	удовлетворяющих неравенствам
	\begin{math}
		a \leq x \leq b
	\end{math}
	.

	\textbf{Полуоткрытый интервал [a; b) или (a; b]} - это
	множество действительных чисел x,
	удовлетворяющих неравенствам
	\begin{math}
		a \leq x < b
	\end{math}
	или
	\begin{math}
		a < x \leq b
	\end{math}
	соответственно.

	\textbf{Бесконечный интервал}
	\begin{math}
		\mathbf{(a; \infty)}
	\end{math}
	, или
	\begin{math}
		\mathbf{[a; \infty)}
	\end{math}
	, или
	\begin{math}
		\mathbf{(\infty; b)}
	\end{math}
	, или
	\begin{math}
		\mathbf{(\infty; b]}
	\end{math}
	, или
	\begin{math}
		\mathbf{(\infty; \infty)}
	\end{math}
	- это множество действительных чисел x,
	удовлетворяющих
	\begin{math}
		a < x
	\end{math}
	, или
	\begin{math}
		a \leq x
	\end{math}
	, или
	\begin{math}
		x < b
	\end{math}
	, или
	\begin{math}
		x \leq b
	\end{math}
	, или
	\begin{math}
		x \in R
	\end{math}
	соответственно.
	
	\section{Комплексные числа (C)}
	\begin{flalign*}
		R \in C
	\end{flalign*}

	Комплексные числа образуют поле.

	\textbf{Комплексное число:}
	\begin{flalign*}
		z = a + bi
	\end{flalign*}
	a - действительная часть
	\\
	b - мнимая часть или коэффициент при мнимой единице.
	\\
	\begin{math}
		i^2 = -1
	\end{math}

	\textbf{Алгебраические действия:}
	рациональные действия и извлечение корня.

	\begin{math}
		z_1 = z_1
	\end{math}
	, если
	\begin{math}
		a_1 = a_2
	\end{math}
	и
	\begin{math}
		b_1 = b_2
	\end{math}
	.

	\begin{math}
		a_1 = a_2
	\end{math}
	и
	\begin{math}
		b_1 = b_2
	\end{math}
	, если
	\begin{math}
		z_1 = z_2
	\end{math}
	.

	\textbf{Чисто мнимое число} - это комплексное
	число, у которого действительная часть
	равна нулю.

	\textbf{Комплексно сопряжённые числа z и 
	\begin{math}
		\overline{\textbf{z}}
	\end{math}
	} - это
	два комплексных числа, действительные
	части которых равны, а мнимые
	противоположны.

	\begin{flalign*}
		z = \overline{\overline{z}}
	\end{flalign*}

	\begin{flalign*}
		z\overline{z} = a^2 + b^2
	\end{flalign*}

	\begin{flalign*}
		\overline{z_1} + \overline{z_2} = \overline{(z_1 + z_2)}
	\end{flalign*}
	\begin{flalign*}
		\overline{z_1}\overline{z_2} = \overline{(z_1z_2)}
	\end{flalign*}

	Значения многочлена при комплексного
	сопряжённых значениях
	комплексно сопряжены между собой.

	Если многочлен имеет комплексный корень,
	то и сопряжённое число является его корнем.

	Если
	\begin{math}
		P_n(z) = (z - \alpha)^k P_{n - k}(z)
	\end{math}
	,
	\begin{math}
		P_{n - k}
	\end{math}
	не делится на
	\begin{math}
		z - \alpha
	\end{math}
	нацело, то k - кратность корня
	\begin{math}
		\alpha
	\end{math}
	.

	Сумма кратности корней равна степени многочлена.

	Если
	\begin{math}
		P_n(z) = (z - \alpha)^k P_{n - k}(z)
	\end{math}
	,
	\begin{math}
		P_{n - k}
	\end{math}
	не делится на
	\begin{math}
		z - \alpha
	\end{math}
	нацело и
	\begin{math}
		k = 1
	\end{math}
	, то корень
	\begin{math}
		\alpha
	\end{math}
	однократный (простой).
	
	Если
	\begin{math}
		P_n(z) = (z - \alpha)^k P_{n - k}(z)
	\end{math}
	,
	\begin{math}
		P_{n - k}
	\end{math}
	не делится на
	\begin{math}
		z - \alpha
	\end{math}
	нацело и
	\begin{math}
		k > 1
	\end{math}
	, то корень
	\begin{math}
		\alpha
	\end{math}
	кратный.

	\textbf{Абсолютная величина (модуль) z:}
	\begin{flalign*}
		\left\lvert z\right\rvert = \sqrt{z\overline{z}}
	\end{flalign*}

	\textbf{Алгебраическая форма комплексного числа:}
	\begin{flalign*}
		z = a + bi
	\end{flalign*}

	\textbf{Тригонометрическая форма комплексного числа:}
	\begin{flalign*}
		z = r(\cos\phi + i\sin\phi)
	\end{flalign*}
	\\
	r - модуль.
	\\
	\begin{math}
		\phi
	\end{math}
	- аргумент.

	\textbf{Главное значение аргумента:}
	\begin{flalign*}
		arg z
	\end{flalign*}

	\begin{flalign*}
		\left\{
			\begin{aligned}
				&arg z \geq 0
				\\
				&arg z < 2\pi
			\end{aligned}
		\right.
	\end{flalign*}

	\begin{flalign*}
		z_1z_2 = r_1r_2(\cos(\phi_1 + \phi_2) + i\sin(\phi_1 + \phi_2))
	\end{flalign*}
	\begin{flalign*}
		\frac{z_1}{z_2} = \frac{r_1}{r_2}(\cos(\phi_1 - \phi_2) + i\sin(\phi_1 - \phi_2))
	\end{flalign*}

	\textbf{Формула Муавра:}
	\begin{flalign*}
		z^n = r^n(\cos n\phi + i\sin n\phi)
	\end{flalign*}
\end{document}