\documentclass[oneside]{book}

\usepackage[utf8]{inputenc}
\usepackage[T2A]{fontenc}
\usepackage[russian]{babel}
\usepackage[left = 0.3\textwidth, right = 0.3\textwidth]{geometry}
\usepackage{parskip}
\usepackage[fleqn]{amsmath}
\usepackage{mathtools}
\usepackage{amsfonts}
\usepackage{amssymb}

\setlength{\parskip}{0.03\textheight}

\newcommand{\meta}[1]{\text{<}#1\text{>}}
\newcommand{\sequence}[1]{\left\{#1\right\}}

\title{Математика}
\date{\today}
\author{Мы}

\begin{document}
	\maketitle

	\tableofcontents

	\chapter{Примечания}
	Чтобы понять, что означают '<', '>',
	попробуйте их убрать.

	\chapter{Логика}
	\begin{flalign*}
		\forall A, \ B \
		\left(
		A \longrightarrow B
		\Leftrightarrow
		\left\{
		\begin{aligned}
			&\text{Из посылки} \ A \ \text{вытекает вывод} \ B. \\
			&A - \text{достаточное условие для} \ B. \\
			&B - \text{необходимое условие для} \ A. \\
		\end{aligned}
		\right.
		\right)
	\end{flalign*}

	\begin{flalign*}
		\forall A, \ B \
		\left(
		\left\{
		\begin{aligned}
			&A \longrightarrow B \\
			&B \longrightarrow A
		\end{aligned}
		\right.
		\Leftrightarrow
		A \ \text{и} \ B - \text{логически эквивалентные утверждения.}
		\right)
	\end{flalign*}

	\begin{flalign*}
		\forall A, \ B \
		(
		(A \longrightarrow B) \ - \ \text{прямое утверждение}.
		\Leftrightarrow
		(B \longrightarrow A) \ - \ \text{обратное утверждение}.
		)
	\end{flalign*}

	\begin{flalign*}
		\forall A, \ B \
		(
		(A \longrightarrow B) \ - \ \text{прямое утверждение}.
		\Leftrightarrow
		(\overline{A} \longrightarrow \overline{B}) \ - \ \text{противоположное утверждение}.
		)
	\end{flalign*}

	\begin{flalign*}
		\forall A, \ B \
		(
		(A \longrightarrow B) \ - \ \text{прямое утверждение}.
		\Leftrightarrow
		(\overline{B} \longrightarrow \overline{A}) \ - \ \text{противоположное обратному утверждение}.
		)
	\end{flalign*}

	\begin{flalign*}
		\forall A, \ B \
		\left(
		\left\{
		\begin{aligned}
			&A - \text{прямое утверждение.} \\
			&B - \text{противоположное обратному утверждение.}
		\end{aligned}
		\right.
		\longrightarrow
		A \Leftrightarrow B
		\right)
	\end{flalign*}

	\textbf{Доказательство от противного:}
	\begin{flalign*}
		\forall A \ \exists B \ (
		B \wedge (\overline{A} \longrightarrow \overline{B})
		\longrightarrow
		A
		)
	\end{flalign*}

	\textbf{Метод математической индукции:}
	\begin{flalign*}
		\forall F \
		\left(
		\forall n \
		\left\{
		\begin{aligned}
			&n \in N \\
			&F(0) \\
			&F(n) \longrightarrow F(n + 1)
		\end{aligned}
		\right.
		\longrightarrow
		\forall n
		F(n)
		\right)
	\end{flalign*}

	\chapter{Алгебраические выражения}
	\textbf{Алгебраическое выражение} - это
	выражение, состоящее из чисел,
	буквенных величин и алгебраических
	операций над ними.

	\textbf{Область допустимых значений (ОДЗ)} - это
	множество всех наборов числовых
	значений букв, входящих
	в данное алгебраическое выражение.

	\textbf{Тождественно равные алгебраические выражения} - это
	алгебраические выражения, имеющие равные ОДЗ и равные
	числовые значения на этом ОДЗ.

	\textbf{Одночлен} - это
	алгебраическое выражение,
	состоящее из произведения числового коэффициента
	и буквенных величин.

	\textbf{Стандартный вид одночлена:}
	\begin{enumerate}
		\item Один числовой коэффициент.
		\item Нет повторяющихся буквенных величин.
	\end{enumerate}

	\textbf{Подобные одночлены} - это
	одночлены, отличающиеся только числовыми
	коэффициентами.

	\textbf{Многочлен (полином)} - это
	алгебраическое выражение,
	состоящее из суммы одночленов.

	\textbf{Стандартный вид многочлена:}
	\begin{enumerate}
		\item Все одночлены стандартного вида.
		\item Нет подобных одночленов.
	\end{enumerate}

    \textbf{Формулы сокращённого умножения:}
    \begin{enumerate}
        \item Квадрат суммы.
        \\
        \begin{math}
            (a + b)^2 = a^2 + 2ab + b^2
        \end{math}

        \item Разность квадратов.
        \\
        \begin{math}
            a^2 - b^2 = (a - b)(a + b)
        \end{math}

        \item Куб суммы.
        \\
        \begin{math}
            (a + b)^3 = a^3 + 3a^2 b + 3ab^2 + b^3
        \end{math}

        \item Сумма кубов.
        \\
        \begin{math}
            a^3 + b^3 = (a + b)(a^2 - ab + b^2)
        \end{math}

        \item Бином Ньютона.
        \\
        \begin{math}
            (a + b)^n =
			\sum\limits_{i = 0}^n \frac{a^{n-i}b^in!}{i!(n - i)!} =
			\sum\limits_{i = 0}^n \frac{a^{n-i}b^i\prod\limits_{k = 0}^{i - 1}n - k}{i!}
        \end{math}
    \end{enumerate}

    \textbf{Неполный квадрат разности:}
    \begin{flalign*}
        a^2 - ab + b^2
    \end{flalign*}

	Многочлен
	\begin{math}
		Q(x)
	\end{math}
	является частным и многочлен
	\begin{math}
		R(x)
	\end{math}
	является остатком при делении многочлена
	\begin{math}
		P_n(x)
	\end{math}
	на
	\begin{math}
		S_m(x)
	\end{math},
	если
	\begin{math}
		P_n(x) = S_m(x)Q(x) + R(x)
	\end{math}
	и степень
	\begin{math}
		R(x)
	\end{math}
	меньше степени
	\begin{math}
		S_m(x)
	\end{math}.

	Если
	степень
	\begin{math}
		P_n(x)
	\end{math}
	больше степени
	\begin{math}
		S_m(x)
	\end{math},
	степень частного от деления
	\begin{math}
		P_n(x)
	\end{math}
	на
	\begin{math}
		S_m(x)
	\end{math}
	равна разности степеней
	\begin{math}
		P_n(x)
	\end{math}
	и
	\begin{math}
		S_m(x)
	\end{math},
	иначе частное равно нулю.

	Остаток от деления многочлена
	\begin{math}
		P_n(x)
	\end{math}
	на двучлен вида
	\begin{math}
		x - \alpha
	\end{math}
	равен значению многочлена при
	\begin{math}
		x = \alpha
	\end{math}.

	\textbf{Теорема Безу:}
	\\
	Многочлен
	\begin{math}
		P_n(x)
	\end{math}
	делится без остатка на двучлен
	\begin{math}
		x - \alpha
	\end{math},
	только если
	\begin{math}
		\alpha
	\end{math}
	- корень многочлена.

	\chapter{Измерения}
	\textbf{Величина} - это
	объект, который может быть охарактеризован числом в результате
	измерения.

	\textbf{Постоянная величина} - это
	величина, множество значений которой
	состоит из одного элемента.

	\textbf{Переменная величина} - это
	величина, множество значений которой
	состоит более чем из одного элемента.

	\textbf{Область изменения} - это
	множество значений, принимаемых переменной
	величиной.

	\chapter{Последовательности}
	\begin{flalign*}
		\forall f \
		\left(
		\exists A \
		f: \mathbb{N} \longrightarrow A
		\Leftrightarrow
		f - \text{последовательность}.
		\right)
	\end{flalign*}

	\begin{flalign*}
		\forall \meta{n}, \meta{x_n}, \ f \
		\left(
		\left\{
		\begin{aligned}
			&f - \text{последовательность}. \\
			&f(n) = x_n
		\end{aligned}
		\right.
		\Leftrightarrow
		\sequence{x_n}
		\right)
	\end{flalign*}

	\begin{flalign*}
		\forall \meta{x_n} \
		\sequence{x_n} - \text{последовательность}.
	\end{flalign*}

	\begin{flalign*}
		\forall \sequence{x_n}
		\left(
		\forall k, \ l \
		(k < l \longrightarrow x_k < x_l)
		\Leftrightarrow
		\sequence{x_n} - \text{возрастающая последовательность}.
		\right)
	\end{flalign*}

	\begin{flalign*}
		\forall \sequence{x_n}
		\left(
		\forall k, \ l \
		(k < l \longrightarrow x_k \geq x_l)
		\Leftrightarrow
		\sequence{x_n} - \text{невозрастающая последовательность}.
		\right)
	\end{flalign*}

	\begin{flalign*}
		\forall \sequence{x_n}
		\left(
		\forall k, \ l \
		(k < l \longrightarrow x_k > x_l)
		\Leftrightarrow
		\sequence{x_n} - \text{убывающая последовательность}.
		\right)
	\end{flalign*}

	\begin{flalign*}
		\forall \sequence{x_n}
		\left(
		\forall k, \ l \
		(k < l \longrightarrow x_k \leq x_l)
		\Leftrightarrow
		\sequence{x_n} - \text{неубывающая последовательность}.
		\right)
	\end{flalign*}

	\begin{flalign*}
		\forall \sequence{x_n}, \ M \
		(
		\forall k \
		\left\lvert x_k \right\rvert \leq M
		\Leftrightarrow
		\sequence{x_n} - \text{ограниченная последовательность значением} \ M.
		)
	\end{flalign*}

	\begin{flalign*}
		\forall \sequence{x_n}, \ M \
		(
		\forall k \
		x_k \leq M
		\Leftrightarrow
		\sequence{x_n} - \text{ограниченная сверху последовательность значением} \ M.
		)
	\end{flalign*}

	\begin{flalign*}
		\forall \sequence{x_n}, \ M \
		(
		\forall k \
		x_k \geq M
		\Leftrightarrow
		\sequence{x_n} - \text{ограниченная снизу последовательность значением} \ M.
		)
	\end{flalign*}

	\begin{flalign*}
		\forall \sequence{x_n}, \ a \
		\left(
		\sequence{x_n} \rightrightarrows a
		\Leftrightarrow
		\sequence{x_n} - \text{последовательность, стабилизирующаяся к} \ a.
		\right)
	\end{flalign*}

	\begin{flalign*}
		\forall \sequence{x_n}, \ a \
		\left(
		\exists k \ \forall m, \ l \
		\left\{
		\begin{aligned}
			&x_m \in \mathbb{Z} \\
			&l > k \\
			&x_l = a
		\end{aligned}
		\right.
		\longrightarrow
		\sequence{x_n} \rightrightarrows a
		\right)
	\end{flalign*}

	\begin{flalign*}
		\forall \sequence{x_n}, \ M \
		\left(
		\forall m \
		\left\{
		\begin{aligned}
			&x_m \in \mathbb{Z} \\
			&\sequence{x_n} - \text{неубывающая последовательность}. \\
			&\sequence{x_n} - \text{ограниченная сверху последовательность значением} \ M.
		\end{aligned}
		\right.
		\longrightarrow
		\exists a \
		\left\{
		\begin{aligned}
			&a \in \mathbb{Z} \\
			&a \leq M \\
			&\sequence{x_n} \rightrightarrows a
		\end{aligned}
		\right.
		\right)
	\end{flalign*}

	\begin{flalign*}
		\forall \sequence{x_n}, \ M, \ a \
		\left(
		\forall m \
		\left\{
		\begin{aligned}
			&x_m \in \mathbb{R} \\
			&\sequence{x_n} - \text{ограниченная сверху последовательность значением} \ M. \\
			&\text{каждая соответствующая цифра} \ \sequence{x_n} \\
			&\rightrightarrows \\
			&\text{каждая соответствующая цифра} \ a \\
		\end{aligned}
		\longrightarrow
		\sequence{x_n} \rightrightarrows a
		\right.
		\right)
	\end{flalign*}

	\begin{flalign*}
		\forall \meta{n}, \meta{x_n}, \ a \
		\left(
		\left\{
		\begin{aligned}
			&\sequence{x_n} \\
			&x_n = a^{(n)}
		\end{aligned}
		\right.
		\Leftrightarrow
		\sequence{x_n} - \text{последовательность десятичных приближений} \ a.
		\right)
	\end{flalign*}

	\begin{flalign*}
		\forall \sequence{x_n}, \ a, \ b \
		\left(
		\left\{
		\begin{aligned}
			&a \in \mathbb{R} \\
			&b \in \mathbb{R} \\
			&\sequence{x_n} = \sequence{a^{(n)} + b^{(n)}}
		\end{aligned}
		\right.
		\longrightarrow
		\sequence{x_n} \rightrightarrows a + b
		\right)
	\end{flalign*}

	\begin{flalign*}
		\forall \sequence{x_n}, \ a, \ b \
		\left(
		\left\{
		\begin{aligned}
			&a \in \mathbb{R} \\
			&b \in \mathbb{R} \\
			&a > b > 0 \\
			&\sequence{x_n} = \sequence{a^{(n)} - (b^{(n)} + 10^{-n})}
		\end{aligned}
		\right.
		\longrightarrow
		\sequence{x_n} \rightrightarrows a - b
		\right)
	\end{flalign*}

	\begin{flalign*}
		\forall \sequence{x_n}, \ a, \ b \
		\left(
		\left\{
		\begin{aligned}
			&a \in \mathbb{R} \\
			&b \in \mathbb{R} \\
			&\sequence{x_n} = \sequence{{a^{(n)}b^{(n)}}^{(n)}}
		\end{aligned}
		\right.
		\longrightarrow
		\sequence{x_n} \rightrightarrows ab
		\right)
	\end{flalign*}

	\begin{flalign*}
		\forall a, \ b \
		\left(
		\left\{
		\begin{aligned}
			&a \in \mathbb{R} \\
			&b \in \mathbb{R} \\
			&\sequence{x_n} = \sequence{\left(\frac{a^{(n)}}{b^{(n)} + 10^{-n}}\right)^{(n)}}
		\end{aligned}
		\right.
		\longrightarrow
		\sequence{x_n} \rightrightarrows \frac{a}{b}
		\right)
	\end{flalign*}

	\begin{flalign*}
		\forall \sequence{x_n}, \ a \
		\left(
		\lim_{n \longrightarrow \infty} x_n = a
		\Leftrightarrow
		\sequence{x_n} \ \text{стремится к} \ a \ \text{как к своему пределу}.
		\right)
	\end{flalign*}

	\begin{flalign*}
		\forall \sequence{x_n}, \ a \
		\left(
		\lim_{n \longrightarrow \infty} x_n = a
		\Leftrightarrow
		\forall e \ \exists l \ \forall k \
		\left\{
		\begin{aligned}
			&\left\lvert a - x_k \right\rvert < e \\
			&k > l
		\end{aligned}
		\right.
		\right)
	\end{flalign*}

	\begin{flalign*}
		\forall \sequence{x_n}, \ M \
		\left(
		\forall m \
		\left\{
		\begin{aligned}
			&x_m \in \mathbb{R} \\
			&x_m > 0 \\
			&\sequence{x_n} - \text{неубывающая последовательность}. \\
			&\sequence{x_n} - \text{ограниченная сверху} \\
			&\text{последовательность значением} \ M.
		\end{aligned}
		\right.
		\longrightarrow
		\exists a \
		\left\{
		\begin{aligned}
			&a \leq M \\
			&\lim_{n \longrightarrow \infty} x_n = a \\
		\end{aligned}
		\right.
		\right)
	\end{flalign*}

	\begin{flalign*}
		\forall \sequence{x_n}, \ a \
		\left(
		\forall m \
		\left\{
		\begin{aligned}
			&x_m \in \mathbb{R} \\
			&\sequence{x_n} = \sequence{a^{(n)}}
		\end{aligned}
		\right.
		\longrightarrow
		\lim_{n \longrightarrow \infty} x_n = a
		\right)
	\end{flalign*}

	\begin{flalign*}
		\forall \sequence{x_n}
		\left(
		\exists a \
		\lim_{n \longrightarrow \infty} x_n = a
		\longrightarrow
		\sequence{x_n} - \text{ограниченная последовательность}.
		\right)
	\end{flalign*}

	\begin{flalign*}
		\forall \sequence{x_n}, \ a \
		\left(
		\lim_{n \longrightarrow \infty} x_n = a
		\longrightarrow
		\exists l \ \forall k \
		\left[
		\begin{aligned}
			&k > l \\
			&\left\{
			\begin{aligned}
				&a > 0 \\
				&x_k > \frac{a}{2}
			\end{aligned}
			\right. \\
			&\left\{
			\begin{aligned}
				&a < 0 \\
				&x_k < \frac{a}{2}
			\end{aligned}
			\right.
		\end{aligned}
		\right.
		\right)
	\end{flalign*}

	\begin{flalign*}
		\forall \sequence{x_n}, \ \sequence{y_n}, \ a, \ b \
		\left(
		\forall k \
		\left\{
		\begin{aligned}
			&\lim_{n \longrightarrow \infty} x_n = a \\
			&\lim_{n \longrightarrow \infty} y_n = b \\
			&x_k \leq y_k
		\end{aligned}
		\right.
		\longrightarrow
		a \leq b
		\right)
	\end{flalign*}

	\begin{flalign*}
		\forall \sequence{x_n}, \ \sequence{y_n}, \ \sequence{z_n}, \ a \
		\left(
		\forall k \
		\left\{
		\begin{aligned}
			&\lim_{n \longrightarrow \infty} x_n = a \\
			&\lim_{n \longrightarrow \infty} z_n = a \\
			&x_k \leq y_k \leq z_k
		\end{aligned}
		\right.
		\longrightarrow
		\lim_{n \longrightarrow \infty} y_n = a
		\right)
	\end{flalign*}

	\begin{flalign*}
		\forall \sequence{x_n}, \ a \
		\left(
		\lim_{n \longrightarrow \infty} x_n = a
		\longrightarrow
		\lim_{n \longrightarrow \infty} \left\lvert x_n \right\rvert = \left\lvert a \right\rvert
		\right)
	\end{flalign*}

	\begin{flalign*}
		\forall a, \ b \
		\left(
		\left\{
		\begin{aligned}
			&a \in \mathbb{R} \\
			&b \in \mathbb{R} \\
		\end{aligned}
		\longrightarrow
		\left\lvert a + b \right\rvert \leq \left\lvert a \right\rvert  + \left\lvert b \right\rvert
		\right.
		\right)
	\end{flalign*}

	\begin{flalign*}
		\forall a, \ b \
		\left(
		\left\{
		\begin{aligned}
			&a \in \mathbb{R} \\
			&b \in \mathbb{R} \\
		\end{aligned}
		\right.
		\longrightarrow
		\left\lvert a - b \right\rvert \geq \left\lvert \left\lvert a \right\rvert - \left\lvert b \right\rvert  \right\rvert
		\right)
	\end{flalign*}

	\begin{flalign*}
		\forall A, \ M \
		\left(
		M = \sup A
		\Leftrightarrow
		M - \text{точная верхняя граница} \ A.
		\right)
	\end{flalign*}

	\begin{flalign*}
		\forall A, \ M \
		\left(
		M = \inf A
		\Leftrightarrow
		M - \text{точная нижняя граница} \ A.
		\right)
	\end{flalign*}

	\begin{flalign*}
		\forall A, \ M \
		\left(
		\forall x, \ M' \ \exists y \
		\left\{
		\begin{aligned}
			&x \in A \\
			&x \leq M \\
			&y \in A \\
			&M' < y \leq M
		\end{aligned}
		\right.
		\Leftrightarrow
		M = \sup A
		\right)
	\end{flalign*}

	\begin{flalign*}
		\forall A, \ M \
		\left(
		\forall x, \ M' \ \exists y \
		\left\{
		\begin{aligned}
			&x \in A \\
			&x \geq M \\
			&y \in A \\
			&M' > y \geq M
		\end{aligned}
		\right.
		\Leftrightarrow
		M = \inf A
		\right)
	\end{flalign*}

	\begin{flalign*}
		\forall A \
		\left(
		\forall B, \ C \
		\left\{
		\begin{aligned}
			&B \in A \\
			&C \in A \\
			&\left[
			\begin{aligned}
				&B \subset C \\
				&C \subset B
			\end{aligned}
			\right.
		\end{aligned}
		\right.
		\Leftrightarrow
		A - \text{система вложенных отрезков}.
		\right)
	\end{flalign*}

	\begin{flalign*}
		\forall \sequence{A_n} \
		\left(
		\forall i, \ j \
		\left\{
		\begin{aligned}
			&i < j \\
			&A_j \subset A_i
		\end{aligned}
		\right.
		\Leftrightarrow
		\sequence{A_n} - \text{послeдовательность вложенных отрезков}.
		\right)
	\end{flalign*}

	\begin{flalign*}
		\forall \sequence{A_n}
		\left(
		\forall e \ \exists i \
		\left\{
		\begin{aligned}
			&e \in \mathbb{R} \\
			&e > 0 \\
			&\sequence{A_n} - \text{последовательность} \\
			&\text{вложенных отрезков}. \\
			&\left\lvert A_i \right\rvert < e
		\end{aligned}
		\right.
		\Leftrightarrow
		\begin{aligned}
			&\sequence{A_n} - \text{стягивающаяся последовательность}\\
			&\text{вложенных отрезков}.
		\end{aligned}
		\right)
	\end{flalign*}

	\begin{flalign*}
		\forall A \
		\left(
		A - \text{система вложенных отрезков}.
		\longrightarrow
		\exists x \ \forall B \
		\left\{
		\begin{aligned}
			&B \in A \\
			&x \in B
		\end{aligned}
		\right.
		\right)
	\end{flalign*}

	\textbf{Принцип полноты Кантора:}
	\begin{flalign*}
		\forall \sequence{A_n} \
		\left(
		\sequence{A_n} - \text{стягивающаяся послeдовательность вложенных отрезков}.
		\longrightarrow
		\exists x \ \forall i \
		\left\{
		\begin{aligned}
			&x \in A_i \\
			&x \ \text{единственен}.
		\end{aligned}
		\right.
		\right)
	\end{flalign*}

	\chapter{Функции}
	\begin{flalign*}
		\forall X, \ Y \
		X \times Y - \text{декартово произведение} \ X \ \text{и} \ Y.
	\end{flalign*}

	\begin{flalign*}
		\forall X, \ Y \
		\left(
		X \times Y
		\Leftrightarrow
		\forall \meta{x}, \ \meta{y} \
		\left\{
		\begin{aligned}
			&x \in X \\
			&y \in Y \\
			&X \times Y = \sequence{(x, y)}
		\end{aligned}
		\right.
		\right)
	\end{flalign*}

	\begin{flalign*}
		\forall \meta{x}, \ \meta{y}, \ f, \ X, \ Y \
		\left(
		\forall z, \ w \
		\left\{
		\begin{aligned}
			&X = \sequence{x} \\
			&Y = \sequence{y} \\
			&z \in X \\
			&w \in Y \\
			&x = z \longrightarrow y = w \\
			&f = X \times Y
		\end{aligned}
		\right.
		\Leftrightarrow
		y = f(x)
		\right)
	\end{flalign*}

	\begin{flalign*}
		\forall f \
		D(f) - \text{область определения} \ f.
	\end{flalign*}

	\begin{flalign*}
		\forall f \
		E(f) - \text{область значений} \ f.
	\end{flalign*}

	\begin{flalign*}
		\forall \meta{x}, \ f, \ X \
		\left(
		f(x)
		\Leftrightarrow
		\left\{
		\begin{aligned}
			&\meta{x} - \text{аргумент (независимая переменная)} \ f. \\
			&f - \text{функция от} \ \meta{x}.
		\end{aligned}
		\right.
		\right)
	\end{flalign*}

	\begin{flalign*}
		\forall \meta{x}, \ f, \ X \
		\left(
		f(x)
		\Leftrightarrow
		X = \sequence{x} \longrightarrow X = D(f)
		\right)
	\end{flalign*}

	\begin{flalign*}
		\forall \meta{y}, \ f, \ x \
		\left(
		y = f(x)
		\Leftrightarrow
		\meta{y} - \text{функция (зависимая переменная)} \ f.
		\right)
	\end{flalign*}

	\begin{flalign*}
		\forall \meta{y}, \ f, \ Y, \ x \
		\left(
		y = f(x)
		\Leftrightarrow
		Y = \sequence{y} \longrightarrow Y = E(f)
		\right)
	\end{flalign*}

	\begin{flalign*}
		\forall \meta{y}, \ \meta{x}, \ f \
		\left(
		y = f(x)
		\Leftrightarrow
		\left\{
		\begin{aligned}
			&y - \text{образ} \ x. \\
			&x - \text{прообраз} \ y.
		\end{aligned}
		\right.
		\right)
	\end{flalign*}

	\begin{flalign*}
		\forall f, \ X, \ Y \
		\left(
		f: X \longrightarrow Y
		\Leftrightarrow
		\left\{
		\begin{aligned}
			&X = D(f) \\
			&Y = E(f)
		\end{aligned}
		\right.
		\right)
	\end{flalign*}

	\begin{flalign*}
		\forall f, \ X, \ Y \
		\left(
		f: X \longrightarrow Y
		\Leftrightarrow
		\left\{
		\begin{aligned}
			&Y - \text{образ} \ X. \\
			&X - \text{прообраз} \ Y.
		\end{aligned}
		\right.
		\right)
	\end{flalign*}

	\begin{flalign*}
		\forall f, \ A \
		\left(
		\forall y
		\left\{
		\begin{aligned}
			&A = \sequence{y} \\
			&y \in E(f)
		\end{aligned}
		\right.
		\Leftrightarrow
		f - \text{сюръекция (накрытие)}.
		\right)
	\end{flalign*}

	\begin{flalign*}
		\forall f \
		\left(
		\forall x, \ y \
		(f(x) = f(y) \longrightarrow x = y)
		\Leftrightarrow
		f - \text{инъекция (вложение)}.
		\right)
	\end{flalign*}

	\begin{flalign*}
		\forall f \
		\left(
		\left\{
		\begin{aligned}
			&f - \text{сюръекция (накрытие)}. \\
			&f - \text{инъекция (вложение)}.
		\end{aligned}
		\right.
		\Leftrightarrow
		f - \text{биекция (взаимно-однозначное соответствие)}.
		\right)
	\end{flalign*}

	\begin{flalign*}
		\forall A, \ B \
		\left(
		A \thicksim B
		\Leftrightarrow
		A \ \text{и} \ B - \text{равномощные}.
		\right)
	\end{flalign*}

	\begin{flalign*}
		\forall f, \ A, \ B \
		\left(
		\left\{
		\begin{aligned}
			&f: A \longrightarrow B \\
			&f - \text{биекция}.
		\end{aligned}
		\right.
		\Leftrightarrow
		A \thicksim B
		\right)
	\end{flalign*}

	\begin{flalign*}
		\forall f \
		\left(
		\forall a, \ b \
		f(a, b) = f(b, a)
		\longrightarrow
		\begin{aligned}
			&f - \text{фунция, обладающая коммутативным} \\
			&\text{(переместительным) свойством}.
		\end{aligned}
		\right)
	\end{flalign*}

	\begin{flalign*}
		\forall f \
		\left(
		\forall a, \ b, \ c \
		f(f(a, b), c) = f(a, f(b, c))
		\longrightarrow
		\begin{aligned}
			&f - \text{фунция, обладающая ассоциативным} \\
			&\text{(сочетательным) свойством}.
		\end{aligned}
		\right)
	\end{flalign*}

	\begin{flalign*}
		\forall f, \ g \
		\left(
		\forall a, \ b, \ c \
		f(a, g(b, c)) = g(f(a, b), g(a, c))
		\longrightarrow
		\begin{aligned}
			&f - \text{фунция, обладающая дистрибутивным} \\
			&\text{(распределительным) свойством c} \ g.
		\end{aligned}
		\right)
	\end{flalign*}

	\begin{flalign*}
		\forall A \
		A \thicksim A
	\end{flalign*}

	\begin{flalign*}
		\forall A, \ B, \ C \
		\left(
		A \thicksim B
		\Leftrightarrow
		B \thicksim A
		\right)
	\end{flalign*}

	\begin{flalign*}
		\forall A, \ B \
		\left(
		\left\{
		\begin{aligned}
			&A \thicksim B \\
			&B \thicksim C
		\end{aligned}
		\right.
		\longrightarrow
		A \thicksim C
		\right)
	\end{flalign*}

	\begin{flalign*}
		\forall A \
		\left(
		A \thicksim \mathbb{N}
		\Leftrightarrow
		A - \text{счётное множество}.
		\right)
	\end{flalign*}

	\begin{flalign*}
		\forall A, \ B \
		\left(
		\forall a, \ b \
		\left\{
		\begin{aligned}
			&a \in A \\
			&b \in B \\
			&a \leq b \\
		\end{aligned}
		\right.
		\Leftrightarrow
		A \ \text{лежит левее} \ B.
		\right)
	\end{flalign*}

	\begin{flalign*}
		\forall A, \ B, \ c \
		\left(
		\forall a, \ b \
		\left\{
		\begin{aligned}
			&a \in A \\
			&b \in B \\
			&c \geq a \\
			&c \leq b
		\end{aligned}
		\right.
		\Leftrightarrow
		c \ \text{разделяет} \ A \ \text{и} \ B.
		\right)
	\end{flalign*}

	\begin{flalign*}
		\forall A \
		\left(
		\forall B, \ C \ \exists a \
		\left\{
		\begin{aligned}
			&B \subset A \\
			&C \subset A \\
			&a \in A \\
			&a \ \text{разделяет} \ B \ \text{и} \ C.
		\end{aligned}
		\right.
		\Leftrightarrow
		A - \text{полное}.
		\right)
	\end{flalign*}

	Если разделяющих элементов в полном множестве больше одного,
	то их бесконечно много.

	\begin{flalign*}
		\forall f, \ g \
		\left(
		\forall x \
		\left\{
		\begin{aligned}
			&D(f) = D(g) \\
			&f(x) = g(x)
		\end{aligned}
		\right.
		\Leftrightarrow
		f \ \text{и} \ g - \text{совпадающие функции.}
		\right)
	\end{flalign*}

	\begin{flalign*}
		\forall f, \ x \
		(
		f(x) = 0
		\Leftrightarrow
		x - \text{нуль (корень) функции f.}
		)
	\end{flalign*}

	\begin{flalign*}
		\forall f \
		\left(
		\forall x \
		\left\{
		\begin{aligned}
			&f(-x) = f(x) \\
			&\exists a \
			\left[
			\begin{aligned}
				&D(f) = (-a; a) \\
				&D(f) = [-a; a]
			\end{aligned}
			\right.
		\end{aligned}
		\right.
		\Leftrightarrow
		f - \text{чётная функция}.
		\right)
	\end{flalign*}

	\begin{flalign*}
		\forall f \
		\left(
		\forall x \
		\left\{
		\begin{aligned}
			&f(-x) = -f(x) \\
			&\exists a \
			\left[
			\begin{aligned}
				&D(f) = (-a; a) \\
				&D(f) = [-a; a]
			\end{aligned}
			\right.
		\end{aligned}
		\right.
		\Leftrightarrow
		f - \text{нечётная функция}.
		\right)
	\end{flalign*}

	\begin{flalign*}
		\forall f \
		\left(
		\left\{
		\begin{aligned}
			&\overline{f - \text{чётная функция}.} \\
			&\overline{f - \text{нечётная функция}.}
		\end{aligned}
		\right.
		\Leftrightarrow
		f - \text{общего вида функция}.
		\right)
	\end{flalign*}

	\begin{flalign*}
		\forall f, \ A \
		\left(
		\forall x_1, \ x_2 \
		\left\{
		\begin{aligned}
			&x_1 \in A \\
			&x_2 \in A \\
			&x_1 < x_2 \longrightarrow f(x_1) < f(x_2)
		\end{aligned}
		\right.
		\Leftrightarrow
		\begin{aligned}
			f - \text{возрастающая функция на} \ A.
		\end{aligned}
		\right)
	\end{flalign*}

	\begin{flalign*}
		\forall f, \ A \
		\left(
		\forall x_1, \ x_2 \
		\left\{
		\begin{aligned}
			&x_1 \in A \\
			&x_2 \in A \\
			&x_1 < x_2 \longrightarrow f(x_1) \geq f(x_2)
		\end{aligned}
		\right.
		\Leftrightarrow
		f - \text{невозрастающая функция на} \ A.
		\right)
	\end{flalign*}

	\begin{flalign*}
		\forall f, \ A \
		\left(
		\forall x_1, \ x_2 \
		\left\{
		\begin{aligned}
			&x_1 \in A \\
			&x_2 \in A \\
			&x_1 < x_2 \longrightarrow f(x_1) > f(x_2)
		\end{aligned}
		\right.
		\Leftrightarrow
		f - \text{убывающая функция на} \ A.
		\right)
	\end{flalign*}

	\begin{flalign*}
		\forall f, \ A \
		\left(
		\forall x_1, \ x_2 \
		\left\{
		\begin{aligned}
			&x_1 \in A \\
			&x_2 \in A \\
			&x_1 < x_2 \longrightarrow f(x_1) \leq f(x_2)
		\end{aligned}
		\right.
		\Leftrightarrow
		f - \text{неубывающая функция на} \ A.
		\right)
	\end{flalign*}

	\begin{flalign*}
		\forall f, \ A \
		\left(
		\left[
		\begin{aligned}
			&f - \text{функция убывающая на} \ A. \\
			&f - \text{функция возрастающая на} \ A.
		\end{aligned}
		\right.
		\Leftrightarrow
		A - \text{интервал монотонности} \ f.
		\right)
	\end{flalign*}

	\begin{flalign*}
		\forall f, \ x_0 \
		\left(
		\exists A \ \forall x \
		\left\{
		\begin{aligned}
			&x \in A \\
			&x_0 \in A \\
			&f(x_0) \leq f(x)
		\end{aligned}
		\right.
		\Leftrightarrow
		x_0 - \text{точка минимума} \ f.
		\right)
	\end{flalign*}

	\begin{flalign*}
		\forall f, \ x_0 \
		\left(
		\exists A \ \forall x \
		\left\{
		\begin{aligned}
			&x \in A \\
			&x_0 \in A \\
			&f(x_0) \geq f(x)
		\end{aligned}
		\right.
		\Leftrightarrow
		x_0 - \text{точка максимума} \ f.
		\right)
	\end{flalign*}

	\begin{flalign*}
		\forall f, \ x \
		\left(
		\left[
		\begin{aligned}
			&x - \text{точка минимума} \ f. \\
			&x - \text{точка минимума} \ f.
		\end{aligned}
		\right.
		\Leftrightarrow
		x - \text{экстремум} \ f.
		\right)
	\end{flalign*}

	\textbf{Асимптота} - это прямая линия,
	к которой график функции неограниченно
	приближается при удалении точки
	графика в бесконечность.

	\textbf{Исследование функции:}
	\begin{enumerate}
		\item Область определения функции.
		\item Область значений функции.
		\item Нули функции.
		\item Чётная, или нечётная, или общего вида функция.
		\item Интервалы монотонности функции.
		\item Экстремумы функции.
		\item Асимптоты функции.
	\end{enumerate}

	\begin{flalign*}
		\forall \meta{y}, \ f, \ g \
		(
		\forall x \
		y = f(g(x))
		\Leftrightarrow
		\meta{y} - \text{сложная функция}.
		)
	\end{flalign*}

	\begin{flalign*}
		\forall f, \ g \
		(
		\forall x \
		f(g(x)) = x
		\Leftrightarrow
		f - \text{обратная} \ g \ \text{функция}.
		)
	\end{flalign*}

	\textbf{Алгебраическая функция} - это
	функция, закон соответствия которой
	определяется алгебраическим выражением. (
	\begin{math}
		\overline{\textbf{трансцендентная функция}}
	\end{math}
	)

	\textbf{Элементарные функции} - это
	основные элементарные функции и
	сложные функции,
	образованные из основных элементарных.

	\textbf{Основные элементарные функции:}
	\begin{enumerate}
		\item
		\begin{flalign*}
		\forall \meta{x}, \ f, \ a \
		\left(
		\left\{
		\begin{aligned}
			&f(x) = x^a \\
			&a \in R
		\end{aligned}
		\right.
		\Leftrightarrow
		f - \text{степенная функция}.
		\right)
		\end{flalign*}

		\item
		\begin{flalign*}
		\forall \meta{x}, \ f, \ a \
		\left(
		\left\{
		\begin{aligned}
			&f(x) = a^x \\
			&a \in R \\
			&a > 0 \\
			&a \neq 1
		\end{aligned}
		\right.
		\Leftrightarrow
		f - \text{показательная функция}.
		\right)
		\end{flalign*}

		\item
		\begin{flalign*}
		\forall \meta{x}, \ f, \ a \
		\left(
		\left\{
		\begin{aligned}
			&f(x) = \log_a^x \\
			&a \in R \\
			&a > 0 \\
			&a \neq 1
		\end{aligned}
		\right.
		\Leftrightarrow
		f - \text{логарифмическая функция}.
		\right)
		\end{flalign*}

		\item
		\begin{flalign*}
		\forall \meta{x}, \ f \
		\left(
		\left[
		\begin{aligned}
			&f(x) = \sin x \\
			&f(x) = \cos x \\
			&f(x) = \tan x \\
			&f(x) = \cot x
		\end{aligned}
		\right.
		\Leftrightarrow
		f - \text{тригонометрическая функция}.
		\right)
		\end{flalign*}

		\item
		\begin{flalign*}
		\forall \meta{x}, \ f \
		\left(
		\left[
		\begin{aligned}
			&f(x) = \arcsin x \\
			&f(x) = \arccos x \\
			&f(x) = \arctan x \\
			&f(x) = \operatorname{arccot} x
		\end{aligned}
		\right.
		\Leftrightarrow
		f - \text{обратная тригонометрическая функция}.
		\right)
		\end{flalign*}
	\end{enumerate}

	\begin{flalign*}
		\begin{aligned}
			&\forall \meta{y}, \ \meta{x}, \ P_i, \ n, \\
			&a_0, \ a_1, \ \ldots \ , \ a_n
		\end{aligned} \
		\left(
		\left\{
		\begin{aligned}
			&y = P_n(x) = \sum_{i = 0}^n a_ix^{n-i} \\
			&a_0 \neq 0
		\end{aligned}
		\right.
		\Leftrightarrow
		\begin{aligned}
			&\meta{y} - \text{целая рациональная функция} \\
			&\text{(многочлен от переменной} \ \meta{x} \text{) (ЦРФ) степени} \ n.
		\end{aligned}
		\right)
	\end{flalign*}

	\begin{flalign*}
		\forall \meta{y}, \ \meta{x}, \ f, \ a \
		\left(
		\left\{
		\begin{aligned}
			&y = f(x) = ax \\
			&a \neq 0
		\end{aligned}
		\right.
		\Leftrightarrow
		\left\{
		\begin{aligned}
			&\meta{y} \ \text{прямо пропорционально} \ \meta{x}. \\
			&\text{между} \ \meta{y} \ \text{и} \ \meta{x} \ \text{прямо пропорциональная зависимость}.
		\end{aligned}
		\right.
		\right)
	\end{flalign*}

	\begin{flalign*}
		\forall \meta{y}, \ \meta{x}, \ f, \ a, \ b \
		\left(
		\left\{
		\begin{aligned}
			&y = f(x) = ax + b \\
			&a \neq 0
		\end{aligned}
		\right.
		\Leftrightarrow
		\meta{y} - \text{линейная ЦРФ (линейная функция)}.
		\right)
	\end{flalign*}

	\begin{flalign*}
		\forall \meta{y}, \ \meta{x}, \ f, \ a, \ b \
		\left(
		\left\{
		\begin{aligned}
			&y = f(x) = ax^2 + bx + c \\
			&a \neq 0
		\end{aligned}
		\right.
		\Leftrightarrow
		\begin{aligned}
			&\meta{y} - \text{квадратичная ЦРФ} \\
			&\text{(квадратный (квадратичный) трёхчлен)}.
		\end{aligned}
		\right)
	\end{flalign*}

	\begin{flalign*}
		\forall \meta{y}, \ \meta{x}, \ f, \ a \
		\left(
		\left\{
		\begin{aligned}
			&y = f(x) = \frac{a}{x} \\
			&a \neq 0
		\end{aligned}
		\right.
		\Leftrightarrow
		\left\{
		\begin{aligned}
			&\meta{y} \ \text{обратно пропорционально} \ \meta{x}. \\
			&\text{между} \ \meta{y} \ \text{и} \ \meta{x} \ \text{обратно пропорциональная зависимость}.
		\end{aligned}
		\right.
		\right)
	\end{flalign*}

	\begin{flalign*}
		\forall \meta{y}, \ x \
		\left(
		y = \frac{P_n(x)}{Q_m(x)}
		\Leftrightarrow
		\meta{y} - \text{дробно-рациональная функция (ДРФ)}.
		\right)
	\end{flalign*}

	\begin{flalign*}
		\forall \meta{y}, \ \meta{x}, \ a, \ b, \ c, \ d \
		\left(
		\left\{
		\begin{aligned}
			&y = f(x) = \frac{ax + b}{cx + d} = \frac{a}{c} + \frac{\frac{bc - ad}{c^2}}{x + \frac{d}{c}} \\
			&c \neq 0 \\
			&ad - bc \neq 0
		\end{aligned}
		\right.
		\Leftrightarrow
		\meta{y} - \text{дробно-линейная функция}.
		\right)
	\end{flalign*}

	\textbf{Алгебраическая иррациональная функция} - это
	функция, закон соответствия которой содержит
	извлечение корня целой степени из алгебраического выражения,
	содержащего аргумент.

	\chapter{Числа}
	\textbf{Числовое кольцо} - это множество чисел,
	результат суммы, разности, произведения любых чисел которого
	принадлежит ему тоже.

	\textbf{Числовое поле} - это множество чисел,
	результат выполнения рациональных действий над любыми
	числами которого принадлежит ему тоже.

	\section{Натуральные числа (N)}
	\textbf{Целочисленная переменная} - это
	величина, принимающая только
	натуральные значения.

	\textbf{Свойства сложения и умножения:}
	\begin{enumerate}
		\item Переместительное.
		\begin{flalign*}
			a + b = b + a
		\end{flalign*}
		\begin{flalign*}
			ab = ba
		\end{flalign*}

		\item Сочетательное (ассоциативное).
		\begin{flalign*}
			(a + b) + c = a + (b + c)
		\end{flalign*}
		\begin{flalign*}
			(ab)c = a(bc)
		\end{flalign*}

		\item Распределительнoe.
	\end{enumerate}

	\begin{flalign*}
		+ - \text{операция на числах, обладающая коммутативным свойством.}
	\end{flalign*}

	\begin{flalign*}
		+ - \text{операция на числах, обладающая ассоциативным свойством.}
	\end{flalign*}

	\begin{flalign*}
		* - \text{операция на числах, обладающая коммутативным свойством.}
	\end{flalign*}

	\begin{flalign*}
		* - \text{операция на числах, обладающая ассоциативным свойством.}
	\end{flalign*}

	\begin{flalign*}
		* - \text{операция на числах, обладающая дистрибутивным свойством c +.}
	\end{flalign*}

	\textbf{Делитель a} - это число, на которое
	a делится без остатка.

	\textbf{Кратное a} - это всякое число, которое
	делится на a без остатка.

	\begin{flalign*}
		\forall a \
		\left(
		\forall b \
		\left\{
		\begin{aligned}
			&a > 1 \\
			&b | a
			\longrightarrow
			\left[
			\begin{aligned}
				&b = \pm 1 \\
				&b = \pm a
			\end{aligned}
			\right.
		\end{aligned}
		\right.
		\Leftrightarrow
		a - \text{простое}.
		\right)
	\end{flalign*}

	\begin{flalign*}
		\forall a \
		\left(
		\overline{a - \text{простое}}.
		\Leftrightarrow
		a - \text{составное}.
		\right)
	\end{flalign*}

	Простых чисел имеется бесконечное множество.

	Разложение числа на простые множители взаимно однозначно.

	\textbf{Взаимно простые числа} - это числа,
	не имеющие общих делителей.

	\textbf{Чётное число} - это число, кратное 2. (
	\begin{math}
		\overline{\textbf{\text{нечётное число}}}
	\end{math}
	)

	Число 2 - единственное чётное простое число.

	\textbf{Признаки делимости в 10-й системе счисления:}
	\begin{enumerate}
		\item Признак делимости на 2: последняя цифра в записи числа выражает чётное число.
		\item Признак делимости на 3: сумма цифр записи числа делится на 3.
		\item Признак делимости на 4: последние две цифры в записи числа выражают число, делящееся на 4.
		\item Признак делимости на 5: последняя цифра в записи числа является 0 или 5.
		\item Признак делимости на 9: сумма цифр записи числа делится на 9.
	\end{enumerate}

	\textbf{Наибольший общий делитель (НОД) a и b:}
	\begin{flalign*}
		( a, b )
	\end{flalign*}

	\textbf{Наименьшее общее кратное (НОК) a и b:}
	\begin{flalign*}
		[ a, b ]
	\end{flalign*}

	\begin{flalign*}
		( a, b ) [ a, b ] = ab
	\end{flalign*}

	\section{Целые числа (Z)}
	\begin{flalign*}
		\mathbb{N} \subset \mathbb{Z}
	\end{flalign*}

	\textbf{Целое алгебраическое выражение} - это
	алгебраическое выражение, в котором
	используют только сложение, вычитание, умножение.

	\textbf{Положительное число} - это число,
	большее нуля.

	\textbf{Отрицательное число} - это число,
	меньшее нуля.

	\textbf{Противоположные числа} - это числа,
	отличающиеся знаком.

	\begin{flalign*}
		a - b = a + (-b)
	\end{flalign*}
	\begin{flalign*}
		a(-b) = -ab
	\end{flalign*}

	\begin{flalign*}
		\forall a, \ b \
		\left(
		a | b
		\Leftrightarrow
		a \ \text{делит} \ b.
		\right)
	\end{flalign*}

	\begin{flalign*}
		\forall a, \ b \
		\left(
		a | b
		\Leftrightarrow
		\exists c \
		\left\{
		\begin{aligned}
			&a \in \mathbb{Z} \\
			&b \in \mathbb{Z} \\
			&c \in \mathbb{Z} \\
			&b = ac
		\end{aligned}
		\right.
		\right)
	\end{flalign*}

	\begin{flalign*}
		\forall a | a
	\end{flalign*}

	\begin{flalign*}
		\forall a, \ b \
		\left(
		\left\{
		\begin{aligned}
			&a | b \\
			&b | a
		\end{aligned}
		\right.
		\longrightarrow
		a = \pm b
		\right)
	\end{flalign*}

	\begin{flalign*}
		\forall a, \ b, \ c \
		\left(
		\left\{
		\begin{aligned}
			&a | b \\
			&b | c
		\end{aligned}
		\right.
		\longrightarrow
		a | c
		\right)
	\end{flalign*}

	\begin{flalign*}
		\forall a, \ b, \ c \
		\left(
		\left\{
		\begin{aligned}
			&a | b \\
			&a | c
		\end{aligned}
		\right.
		\longrightarrow
		a | (b \pm c)
		\right)
	\end{flalign*}

	\begin{flalign*}
		\forall a, \ b, \ c \
		\left(
		\left\{
		\begin{aligned}
			&a | b \\
			&a | (b \pm c)
		\end{aligned}
		\right.
		\longrightarrow
		a | c
		\right)
	\end{flalign*}

	\begin{flalign*}
		\forall a, \ b, \
		\left(
		\left\{
		\begin{aligned}
			&a \in \mathbb{N} \\
			&b \in \mathbb{N} \\
			&b \neq 0
		\end{aligned}
		\right.
		\longrightarrow
		\exists! c, \ d \
		\left\{
		\begin{aligned}
			&d \in \mathbb{N} \\
			&c \in \mathbb{N} \\
			&a = bd + c \\
			&0 \leq c < \left\lvert b\right\rvert
		\end{aligned}
		\right.
		\right)
	\end{flalign*}

	\begin{flalign*}
		\forall a, \ b, \ c \
		\left(
		a \equiv b \ (mod \ c)
		\Leftrightarrow
		c | (a - b)
		\right)
	\end{flalign*}

	\begin{flalign*}
		\forall a, \ b, \ c \
		\left(
		a \equiv b \ (mod \ c)
		\Leftrightarrow
		a \% c = b \% c
		\right)
	\end{flalign*}

	\begin{flalign*}
		\forall a, \ b, \ c, \ d, \ i \
		\left(
		\left\{
		\begin{aligned}
			&a \equiv b \ (mod \ i) \\
			&c \equiv d \ (mod \ i)
		\end{aligned}
		\right.
		\longrightarrow
		\left\{
		\begin{aligned}
			&a + c \equiv b + d \ (mod \ i) \\
			&ac \equiv bd \ (mod \ i)
		\end{aligned}
		\right.
		\right)
	\end{flalign*}

	\begin{flalign*}
		\forall a, \ b, \ c, \ d \
		\left(
		a \equiv b \ (mod \ c)
		\longrightarrow
		a^d \equiv b^d \ (mod \ c)
		\right)
	\end{flalign*}

	\section{Рациональные числа (Q)}
	\begin{flalign*}
		\mathbb{Z} \subset \mathbb{Q}
	\end{flalign*}

	\textbf{Рациональное число} - это число,
	представимое в виде
	\begin{math}
		\frac{a}{b}
	\end{math}
	, где числитель
	\begin{math}
		a \in Z
	\end{math}
	, а знаменатель
	\begin{math}
		b \in N
	\end{math}
	.

	Рациональные числа образуют поле.

	\begin{flalign*}
		\mathbb{Q} \ \text{всюду плотно в} \ \mathbb{R}.
	\end{flalign*}

	\textbf{Арифметические (рациональные) действия:}
	сложение, вычитание, умножение, деление.

	\textbf{Рациональное алгебраическое выражение} - это
	алгебраическое выражение, в котором используют
	только рациональные действия.

	\textbf{Дробное алгебраическое выражение} - это
	рациональное алгебраическое выражение, в
	записи которого используют деление на буквенные
	выражения.

	\textbf{Алгебраическая дробь} - это
	это алгебраическое выражение, имеющее вид частного
	от деления двух целых
	алгебраических выражений.

	\textbf{Дробное число} - это рациональное число,
	числитель которого не делится на знаменатель нацело.

	\textbf{Целая часть числа} - это наибольшее
	целое число, не превосходящее данного (
	\begin{math}
		\left[ x \right]
	\end{math}
	)
	.

	\textbf{Дробная часть числа} - это разность
	между данным числом и его целой
	частью (
	\begin{math}
		(x)
	\end{math}
	)
	.

	\begin{flalign*}
		x - \left[x\right] \geq 0
	\end{flalign*}
	\begin{flalign*}
		x - \left[x\right] < 1
	\end{flalign*}

	Разложение рационального числа
	на сумму целой и дробной частей
	взаимно однозначно.

	\textbf{Десятичная дробь} - это дробь,
	у которой знаменатель представляет
	собой натуральную степень числа 10.

	Всякое рациональное число может
	быть представленно бесконечной
	десятичной периодической дробью
	взаимно однозначно.

	\section{Иррациональные числа (I)}
	Всякое иррациональное число может
	быть представленно бесконечной
	десятичной непериодической дробью
	взаимно однозначно.

	\textbf{Иррациональные алгебраические выражения} - это
	алгебраическое выражение, в записи которого
	используются знаки радикала из буквенного выражения.

	\textbf{Корень находится в простейшей форме, если:}
	\begin{enumerate}
		\item Он не содержит иррациональности в знаменателе.
		\item Нельзя сократить его показатель с показателем подкоренного выражения.
		\item Все возможные множители вынесены из-под корня.
	\end{enumerate}

	\textbf{Подобные корни} - это корни,
	отличающиеся только коэффициентами.

	\section{Действительные числа (R)}
	\begin{flalign*}
		Q \subset R
	\end{flalign*}
	\begin{flalign*}
		I \subset R
	\end{flalign*}

	Действительные числа образуют поле.

	Множество действительных чисел упорядочено.

	Множество действительных чисел непрерывно.

	\begin{flalign*}
		\mathbb{R} - \text{полное}.
	\end{flalign*}

	\textbf{Аксиома Архимеда:}
	\begin{flalign*}
		\forall a \exists n \
		\left\{
		\begin{aligned}
			&a \in \mathbb{R} \\
			&n \in \mathbb{N} \\
			&na \geq 1
		\end{aligned}
		\right.
	\end{flalign*}

	\begin{flalign*}
		\forall x, \ y \
		\left(
		\left\{
		\begin{aligned}
			&x < y \\
			&x \in \mathbb{R} \\
			&y \in \mathbb{R}
		\end{aligned}
		\right.
		\longrightarrow
		\exists z, \ w \
		\left\{
		\begin{aligned}
			&z \in \mathbb{Q} \\
			&w \in \mathbb{I} \\
			&x < z < y \\
			&x < w < y
		\end{aligned}
		\right.
		\right)
	\end{flalign*}

	Всякое десятичное число
	определяет действительное число
	взаимно однозначно.

	\begin{flalign*}
		\forall a \
		\left(
		\left\{
		\begin{aligned}
			&a \in \mathbb{R} \\
			&a \geq 0
		\end{aligned}
		\right.
		\longrightarrow
		\left\lvert a\right\rvert = a
		\right)
	\end{flalign*}

	\begin{flalign*}
		\forall a \
		\left(
		\left\{
		\begin{aligned}
			&a \in \mathbb{R} \\
			&a < 0
		\end{aligned}
		\right.
		\longrightarrow
		\left\lvert a\right\rvert = -a
		\right)
	\end{flalign*}

	Многочлен с действительными
	коэффициентами разлагается в произведение
	линейных двучленов вида
	\begin{math}
		x - a
	\end{math}
	и квадратных трёхчленов вида
	\begin{math}
		x^2 + px + q
	\end{math}
	.

	\textbf{n-ая степень числа a} - это произведение
	n сомножителей, равных а. (
	\begin{math}
		a^n
	\end{math}
	)
	\\
	a - основание степени.
	\\
	n - показатель степени.

	Возведение отрицательного числа
	в иррациональную степень не определено.

	Возведение нуля в не положительную степень не определено.

	\begin{flalign*}
		a^x = a^y \longrightarrow x = y
	\end{flalign*}

	\textbf{Корень n-ой степени из числа a} - это
	число, n-ая степень которого равна a. (
	\begin{math}
		\sqrt[n]{a}
	\end{math}
	)

	\textbf{Извлечение корня степени из a} - это
	отыскание корня из a.

	\textbf{Арифметический корень (арифметическое значение корня)} - это
	положительный корень чётной степени из положительного числа.

	Корень чётной степени по умолчанию aрифметический.

	\begin{flalign*}
		\left[
		\begin{aligned}
		&\left\{
		\begin{aligned}
			&\sqrt[n]{a^n} = a
			\\
			&n \text{ - нечётное.}
		\end{aligned}
		\right.
		\\
		&\left\{
		\begin{aligned}
			&\sqrt[n]{a^n} = \left\lvert a\right\rvert
			\\
			&n \text{ - чётное.}
		\end{aligned}
		\right.
		\end{aligned}
		\right.
	\end{flalign*}

	\textbf{Квадратный корень:}
	\begin{flalign*}
		\sqrt[2]{x} = \sqrt{x}
	\end{flalign*}

	\textbf{Кубический корень:}
	\begin{flalign*}
		\sqrt[3]{x}
	\end{flalign*}

	\begin{flalign*}
		\sqrt[b]{x^a} = x^\frac{a}{b}
	\end{flalign*}

	\textbf{Логарифм числа N по основанию a} - это
	показатель степени, в которую
	нужно возвести a, чтобы получить N.

	\begin{flalign*}
		\left\{
		\begin{aligned}
			&a^{\log_a N} = N
			\\
			&N > 0
			\\
			&a > 0
			\\
			&a \neq 1
		\end{aligned}
		\right.
	\end{flalign*}

	Если число и основание логарифма
	лежат по одну сторону от
	единицы, то этот логарифм
	положителен, и наоборот.

	\textbf{Потенцирование} - это
	возведение числа, от которого взят
	логарифм, в этот логарифм.

	Если основание больше единицы,
	то большее число имеет больший
	логарифм.

	Если основание меньше единицы, то
	большее число имеет меньший логарифм.

	\textbf{Десятичный логарифм} - это
	логарифм по основанию 10.

	\begin{flalign*}
		\log_{10} N = \lg N
	\end{flalign*}

	\textbf{Характеристика} - это
	целая часть десятичного логарифма.

	\textbf{Мантиса} - это
	дробная часть десятичного логарифма.

	\textbf{Открытый интервал (интервал) (a; b)} - это
	множество действительных чисел x, удовлетворяющих
	неравествам
	\begin{math}
		a < x \leq b
	\end{math}
	.

	\textbf{Окрестность точки}
	\begin{math}
		\mathbf{
			x_0 \
			(x_0 - h; x_0 + h)
		}
	\end{math}
	- это интервал длины 2h
	серединой
	\begin{math}
		x_0
	\end{math}
	.

	\begin{flalign*}
		\forall a, \ e \
		(a - e; a) \cup (a; a + e) - \text{проколотая} \ e\text{-окрестность точки} \ a.
	\end{flalign*}

	\textbf{Замкнутый интервал (отрезок) [a; b]} - это
	множество действительных чисел x,
	удовлетворяющих неравенствам
	\begin{math}
		a \leq x \leq b
	\end{math}
	.

	\textbf{Полуоткрытый интервал [a; b) или (a; b]} - это
	множество действительных чисел x,
	удовлетворяющих неравенствам
	\begin{math}
		a \leq x < b
	\end{math}
	или
	\begin{math}
		a < x \leq b
	\end{math}
	соответственно.

	\textbf{Бесконечный интервал}
	\begin{math}
		\mathbf{(a; \infty)}
	\end{math}
	, или
	\begin{math}
		\mathbf{[a; \infty)}
	\end{math}
	, или
	\begin{math}
		\mathbf{(\infty; b)}
	\end{math}
	, или
	\begin{math}
		\mathbf{(\infty; b]}
	\end{math}
	, или
	\begin{math}
		\mathbf{(\infty; \infty)}
	\end{math}
	- это множество действительных чисел x,
	удовлетворяющих
	\begin{math}
		a < x
	\end{math}
	, или
	\begin{math}
		a \leq x
	\end{math}
	, или
	\begin{math}
		x < b
	\end{math}
	, или
	\begin{math}
		x \leq b
	\end{math}
	, или
	\begin{math}
		x \in R
	\end{math}
	соответственно. (
	\begin{math}
		\overline{\textbf{\text{конечный интервал}}}
	\end{math}
	)

	\begin{flalign*}
		\forall a, \ b \
		\left\lvert [a; b] \right\rvert - \text{длина отрезка} \ [a; b].
	\end{flalign*}

	\begin{flalign*}
		\forall a, \ b \
		\left\lvert [a; b] \right\rvert = b - a
	\end{flalign*}

	\begin{flalign*}
		\forall A, \ a, \ b \
		\left(
		A = [a; b]
		\longrightarrow
		\overline{A - \text{счётное множество}.}
		\right)
	\end{flalign*}

	\begin{flalign*}
		\text{Мощность} \ [0; 1] - \text{мощность континуума}.
	\end{flalign*}

	\begin{flalign*}
		\forall C, \ B \
		\left(
		\left(
		\forall A, \ a \
		\left\{
		\begin{aligned}
			&A - \text{интервал}. \\
			&A \subset B \\
			&a \in A \\
		\end{aligned}
		\right.
		\longrightarrow
		a \in C
		\right)
		\Leftrightarrow
		C \ \text{всюду плотно в} \ B.
		\right)
	\end{flalign*}

	\section{Комплексные числа (C)}
	\begin{flalign*}
		\mathbb{R} \subset \mathbb{C}
	\end{flalign*}

	Комплексные числа образуют поле.

	\textbf{Комплексное число:}
	\begin{flalign*}
		z = a + bi
	\end{flalign*}
	a - действительная часть
	\\
	b - мнимая часть или коэффициент при мнимой единице.
	\\
	\begin{math}
		i^2 = -1
	\end{math}

	\textbf{Алгебраические действия:}
	рациональные действия и извлечение корня.

	\begin{math}
		z_1 = z_1
	\end{math}
	, если
	\begin{math}
		a_1 = a_2
	\end{math}
	и
	\begin{math}
		b_1 = b_2
	\end{math}
	.

	\begin{math}
		a_1 = a_2
	\end{math}
	и
	\begin{math}
		b_1 = b_2
	\end{math}
	, если
	\begin{math}
		z_1 = z_2
	\end{math}
	.

	\textbf{Чисто мнимое число} - это комплексное
	число, у которого действительная часть
	равна нулю.

	\textbf{Комплексно сопряжённые числа z и
	\begin{math}
		\overline{\textbf{z}}
	\end{math}
	} - это
	два комплексных числа, действительные
	части которых равны, а мнимые
	противоположны.

	\begin{flalign*}
		z = \overline{\overline{z}}
	\end{flalign*}

	\begin{flalign*}
		z\overline{z} = a^2 + b^2
	\end{flalign*}

	\begin{flalign*}
		\overline{z_1} + \overline{z_2} = \overline{(z_1 + z_2)}
	\end{flalign*}
	\begin{flalign*}
		\overline{z_1}\overline{z_2} = \overline{(z_1z_2)}
	\end{flalign*}

	Значения многочлена при комплексного
	сопряжённых значениях
	комплексно сопряжены между собой.

	Если многочлен имеет комплексный корень,
	то и сопряжённое число является его корнем.

	Если
	\begin{math}
		P_n(z) = (z - \alpha)^k P_{n - k}(z)
	\end{math}
	,
	\begin{math}
		P_{n - k}
	\end{math}
	не делится на
	\begin{math}
		z - \alpha
	\end{math}
	нацело, то k - кратность корня
	\begin{math}
		\alpha
	\end{math}
	.

	Сумма кратности корней равна степени многочлена.

	Если
	\begin{math}
		P_n(z) = (z - \alpha)^k P_{n - k}(z)
	\end{math}
	,
	\begin{math}
		P_{n - k}
	\end{math}
	не делится на
	\begin{math}
		z - \alpha
	\end{math}
	нацело и
	\begin{math}
		k = 1
	\end{math}
	, то корень
	\begin{math}
		\alpha
	\end{math}
	однократный (простой).

	Если
	\begin{math}
		P_n(z) = (z - \alpha)^k P_{n - k}(z)
	\end{math}
	,
	\begin{math}
		P_{n - k}
	\end{math}
	не делится на
	\begin{math}
		z - \alpha
	\end{math}
	нацело и
	\begin{math}
		k > 1
	\end{math}
	, то корень
	\begin{math}
		\alpha
	\end{math}
	кратный.

	\textbf{Абсолютная величина (модуль) z:}
	\begin{flalign*}
		\left\lvert z\right\rvert = \sqrt{z\overline{z}}
	\end{flalign*}

	\textbf{Алгебраическая форма комплексного числа:}
	\begin{flalign*}
		z = a + bi
	\end{flalign*}

	\textbf{Тригонометрическая форма комплексного числа:}
	\begin{flalign*}
		z = r(\cos\phi + i\sin\phi)
	\end{flalign*}
	\\
	r - модуль.
	\\
	\begin{math}
		\phi
	\end{math}
	- аргумент.

	\textbf{Главное значение аргумента:}
	\begin{flalign*}
		arg z
	\end{flalign*}

	\begin{flalign*}
		\left\{
			\begin{aligned}
				&arg z \geq 0
				\\
				&arg z < 2\pi
			\end{aligned}
		\right.
	\end{flalign*}

	\begin{flalign*}
		z_1z_2 = r_1r_2(\cos(\phi_1 + \phi_2) + i\sin(\phi_1 + \phi_2))
	\end{flalign*}
	\begin{flalign*}
		\frac{z_1}{z_2} = \frac{r_1}{r_2}(\cos(\phi_1 - \phi_2) + i\sin(\phi_1 - \phi_2))
	\end{flalign*}

	\textbf{Формула Муавра:}
	\begin{flalign*}
		z^n = r^n(\cos n\phi + i\sin n\phi)
	\end{flalign*}

	\chapter{Матрицы}
	\begin{flalign*}
		\forall A, \ m, \ n \
		A_{m \times n} - \text{матрица размера (порядка)} \ m \ \text{на} \ n.
	\end{flalign*}

	\begin{flalign*}
		\forall A, \ m, \ n \
		\left(
		A
		=
		\left(
		\begin{array}{rrrr}
			a_{1 1} & a_{1 2} & \ldots & a{1 n} \\
			a_{2 1} & a_{2 2} & \ldots & a{2 n} \\
			\vdots  & \vdots  & \ddots & \vdots \\
			a_{m 1} & a_{m 2} & \ldots & a_{m n}
		\end{array}
		\right)
		\Leftrightarrow
		A = A_{m \times n}
		\right)
	\end{flalign*}

	\begin{flalign*}
		\forall A, \ i, \ j \
		\left(
		\exists m, \ n \
		\left\{
		\begin{aligned}
			&A = A_{m \times n} \\
			&i \in \mathbb{N} \\
			&i \leq m \\
			&j \in \mathbb{N} \\
			&j \leq n
		\end{aligned}
		\right.
		\Leftrightarrow
		[A]_{i j} - \text{элемент матрицы с индексами i и j}.
		\right)
	\end{flalign*}

	\begin{flalign*}
		\forall A, \ n \
		\left[
		\begin{aligned}
			&A_n - \text{столбец} \ n \ \text{матрицы} \ A. \\
			&A_n - \text{строка} \ n \ \text{матрицы} \ A.
		\end{aligned}
		\right.
	\end{flalign*}

	\begin{flalign*}
		\forall A, \ m \
		A_{m \times 1} - \text{матрица-строка (вектор-строка)}.
	\end{flalign*}

	\begin{flalign*}
		\forall A, \ n \
		A_{n \times 1} - \text{матрица-столбец (вектор-столбец)}.
	\end{flalign*}

	\begin{flalign*}
		\forall A, \ m \
		A_{m \times m} - \text{квадратная}.
	\end{flalign*}

	\begin{flalign*}
		\forall A, \ m \
		A_{m \times m}
		\Leftrightarrow
		A - \text{порядка} \ m.
	\end{flalign*}

	\begin{flalign*}
		\forall A \
		\left(
		A = 0
		\Leftrightarrow
		A - \text{нулевая}.
		\right)
	\end{flalign*}

	\begin{flalign*}
		\forall A \
		\left(
		\forall i, \ j \
		[A]_{i j} = 0
		\Leftrightarrow
		A = 0
		\right)
	\end{flalign*}

	\begin{flalign*}
		\forall B \
		\left(
		\forall A \
		AB = BA = A
		\Leftrightarrow
		B - \text{единичная}.
		\right)
	\end{flalign*}

	\begin{flalign*}
		E - \text{единичная матрица}.
	\end{flalign*}

	\begin{flalign*}
		\forall i, \ j \
		\left(
		i \neq j \Leftrightarrow [E]_{i j} = 0
		\right)
	\end{flalign*}

	\begin{flalign*}
		\forall i, \ j \
		\left(
		i = j \Leftrightarrow [E]_{i j} = 1
		\right)
	\end{flalign*}

	\begin{flalign*}
		\forall i, \ j \
		E_{i j} - \text{матричная единичка}.
	\end{flalign*}

	\begin{flalign*}
		\forall i, \ j, \ k, \ l \
		\left(
		\left\{
		\begin{aligned}
			&k \neq i \\
			&l \neq j
		\end{aligned}
		\right.
		\Leftrightarrow
		[E_{i j}]_{k l} = 0
		\right)
	\end{flalign*}

	\begin{flalign*}
		\forall i, \ j, \ k, \ l \
		\left(
		\left\{
		\begin{aligned}
			&k = i \\
			&l = j
		\end{aligned}
		\right.
		\Leftrightarrow
		[E_{i j}]_{k l} = 1
		\right)
	\end{flalign*}

	\begin{flalign*}
		\forall A, \
		\left(
		\exists \lambda \
		A = \lambda E
		\Leftrightarrow
		A - \text{скалярная (диагональная)}.
		\right)
	\end{flalign*}

	\begin{flalign*}
		\forall A \
		\left(
		\exists n
		A^n = 0
		\Leftrightarrow
		A - \text{нильпотентная}.
		\right)
	\end{flalign*}

	\begin{flalign*}
		\forall A, \ B, \ m, \ n \
		\left(
		\forall i, \ j \
		\left\{
		\begin{aligned}
			&A_{m\times n} \\
			&B_{m \times n} \\
			&[A]_{i j} = [B]_{i j}
		\end{aligned}
		\right.
		\Leftrightarrow
		A = B
		\right)
	\end{flalign*}

	\begin{flalign*}
		\forall A \
		A^T - \text{транспонированная}.
	\end{flalign*}

	\begin{flalign*}
		\forall A, \ B \
		\left(
		\forall i, \ j \
		[A]_{i j} = [B]_{j i}
		\Leftrightarrow
		B = A^T
		\right)
	\end{flalign*}

	\begin{flalign*}
		\forall A, \ B \
		(A B)^T = B^T A^T
	\end{flalign*}

	\begin{flalign*}
		\forall A, \ B, \ i, \ j \
		[A + B]_{i j} = [A]_{i j} + [B]_{i j}
	\end{flalign*}

	\begin{flalign*}
		+ - \text{операция на матрицах, обладающая коммутативным свойством.}
	\end{flalign*}

	\begin{flalign*}
		+ - \text{операция на матрицах, обладающая ассоциативным свойством.}
	\end{flalign*}

	\begin{flalign*}
		\forall A, \ \lambda, \ i, \ j \
		\left(
		\lambda \in \mathbb{F}
		\longrightarrow
		[\lambda A]_{i j} = \lambda [A]_{i j}
		\right)
	\end{flalign*}

	\begin{flalign*}
		* - \text{операция на матрице и числе, обладающая ассоциативным свойством.}
	\end{flalign*}

	\begin{flalign*}
		&* - \text{операция на матрице и числе, обладающая дистрибутивным свойством} \\
		&\text{(относительно матриц и относительно чисел) c +.}
	\end{flalign*}

	\begin{flalign*}
		\forall A, \ B, \ m, \ r, \ n, \ i, \ j \
		\left(
		\left\{
		\begin{aligned}
			&A = A_{m \times r} \\
			&B = B_{r \times n}
		\end{aligned}
		\right.
		\longrightarrow
		[A B]_{i j} = \sum_{k = 1}^r [A]_{i k} [B]_{k j}
		\right)
	\end{flalign*}

	\begin{flalign*}
		* - \text{операция на матрицах, обладающая ассоциативным свойством.}
	\end{flalign*}

	\begin{flalign*}
		* - \text{операция на матрицах, обладающая дистрибутивным свойством c +.}
	\end{flalign*}

	\textbf{Элементарные преобразования матриц:}
	\begin{enumerate}
		\item Перестановка местами любых двух строк или столбцов матрицы.
		\begin{enumerate}
			\item Переставить местами i и j строки - это единичную матрицу,
			где переставили местами i и j строки или столбцы,
			умножить на матрицу.

			\item Переставить местами i и j столбцы - это умножить матрицу на единичную матрицу,
			где переставили местами i и j строки или столбцы.
		\end{enumerate}

		\item Умножение любой строки или столбца матрицы на константу,
		отличную от нуля.
		\begin{enumerate}
			\item Умножить строку i на константу, отличную от нуля, - это
			единичную матрицу,
			где i строку или столбец умножили на эту константу,
			умножить на матрицу.

			\item Умножить столбец i на константу, отличную от нуля, - это
			умножить матрицу на единичную матрицу,
			где i строку или столбец умножили на эту константу.
		\end{enumerate}

		\item Прибавление к любой строке или столбцу матрицы этой матрицы
		другой строки или столбца, умноженной на некоторую
		константу.
		\begin{enumerate}
			\item Прибавить к строке i строку j, умноженную на константу, - это
			единичную матрицу,
			где прибавили к i строке строку j, умноженную на константу,
			умножить на матрицу.

			\item Прибавить к столбцу i столбец j, умноженный на константу, - это
			умножить матрицу на единичную матрицу,
			где прибавили к i столбцу столбец j, умноженный на константу.
		\end{enumerate}
	\end{enumerate}

	\textbf{Ведущий элемент} - это
	первый ненулевой элемент строки.

	\textbf{Ступенчатый вид матрицы} - это
	матрица, номера столбцов ведущих элементов которой
	возрастают, а нулевые строки, если они есть,
	расположены внизу.

	\textbf{Улучшенный (приведённый, канонический) ступенчатый вид матрицы} - это
	ступенчатый вид матрицы, в котором все ведущие
	элементы - единицы, над которыми в столбце все
	элементы - нули.

	Любую матрицу элементарными преобразованиями
	можно привести к улучшенному ступенчатому виду.

	\begin{flalign*}
		\forall \sequence{a_r}, \ a, \ n \
		\left(
		\forall k, \ l \
		\left\{
		\begin{aligned}
			&a_k \in \mathbb{N} \\
			&a_k \neq n \\
			&k \neq l \longrightarrow a_k \neq a_l \\
			&a = a_1, a_2, \ldots, a_n
		\end{aligned}
		\right.
		\Leftrightarrow
		a - \text{перестановка}.
		\right)
	\end{flalign*}

	\begin{flalign*}
		\forall f, \ n \
		\left(
		\left\{
		\begin{aligned}
			&(f(1), f(2), \ldots, f(n)) - \text{перестановка}. \\
			&f
			=
			\left(
			\begin{array}{rrrr}
				1    & 2    & \ldots & n    \\
				f(1) & f(2) & \ldots & f(n)
			\end{array}
			\right)
		\end{aligned}
		\right.
		\Leftrightarrow
		f - \text{подстановка}.
		\right)
	\end{flalign*}

	\begin{flalign*}
		\forall f, \ n \
		\left(
		n - \text{сумма инверсий первой и второй строки подстановки} \ f.
		\longrightarrow
		sgn f = (-1)^n
		\right)
	\end{flalign*}

	\begin{flalign*}
		\forall f \
		\left(
		\left\{
		\begin{aligned}
			&f - \text{подстановка}. \\
			&\text{Сумма инверсий первой и второй строки} \ f \ \text{чётна}.
		\end{aligned}
		\right.
		\Leftrightarrow
		f - \text{чётная}.
		\right)
	\end{flalign*}

	\begin{flalign*}
		\forall f, \ g \
		\left(
		\left\{
		\begin{aligned}
			&f - \text{подстановка}. \\
			&g - \text{подстановка}.
		\end{aligned}
		\right.
		\longrightarrow
		fg - \text{умножение (композиция) подстановок}.
		\right)
	\end{flalign*}

	\begin{flalign*}
		\forall f, \ g \
		\left(
		\left\{
		\begin{aligned}
			&f - \text{подстановка}. \\
			&g - \text{подстановка}.
		\end{aligned}
		\right.
		\longrightarrow
		fg
		=
		\left(
		\begin{array}{rrrr}
			1       & 2       & \ldots & n       \\
			g(f(1)) & g(f(2)) & \ldots & g(f(n))
		\end{array}
		\right)
		\right)
	\end{flalign*}

	\begin{flalign*}
		\forall f, \ g \
		\left(
		\left\{
		\begin{aligned}
			&f - \text{подстановка}. \\
			&g - \text{подстановка}.
		\end{aligned}
		\right.
		\longrightarrow
		sgn(fg) = sgn f * sgn g
		\right)
	\end{flalign*}

	\begin{flalign*}
		\forall f \
		\left(
		f - \text{подстановка}.
		\longrightarrow
		\left\{
		\begin{aligned}
			&d(f) - \text{декремент}. \\
			&d(f) = \text{длина} \ f - \\
			&(\text{число независимых циклов} \ f + \\
			&\text{количество символов, оставляемых на месте}). \\
			&d(f) = \text{количество действительно перемещаемых символов} - \\
			&\text{количество независимых циклов}. \\
			&d(f) = \text{сумма длин циклов} - \text{количество циклов}.
		\end{aligned}
		\right.
		\right)
	\end{flalign*}

	\begin{flalign*}
		\forall f \
		\left(
		f - \text{подстановка}.
		\longrightarrow
		sgn f = (-1)^{d(f)}
		\right)
	\end{flalign*}

	\begin{flalign*}
		id - \text{подстановка}.
	\end{flalign*}

	\begin{flalign*}
		id - \text{тождественная}.
	\end{flalign*}

	\begin{flalign*}
		id
		=
		\left(
		\begin{array}{rrrr}
			1 & 2 & \ldots & n \\
			1 & 2 & \ldots & n
		\end{array}
		\right)
	\end{flalign*}

	\begin{flalign*}
		\forall f \
		f^{-1}
		=
		\left(
		\begin{array}{rrrr}
			f(1) & f(2) & \ldots & f(n) \\
			1    & 2    & \ldots & n
		\end{array}
		\right)
	\end{flalign*}

	\begin{flalign*}
		\forall n, \ k \
		\left(
		k \in \mathbb{N}
		\longrightarrow
		(a_1, a_2, \ldots, a_n)^{nk} = id
		\right)
	\end{flalign*}

	При возведении подстановки в степень,
	кратную НОКу длин всех её циклов,
	будет получаться тождественная
	подстановка.

	\begin{flalign*}
		\forall A \
		det A - \text{определитель} \ A.
	\end{flalign*}

	\begin{flalign*}
		\forall A, \ n \
		\left(
		A_{n \times n}
		\longrightarrow
		det A = \sum_{f \in S_n} sgn f * [A]_{1 f(1)} * [A]_{2 f(2)} * \ldots * [A]_{n f(n)}
		\right)
	\end{flalign*}

	\begin{flalign*}
		\forall A \
		det A^T = det A
	\end{flalign*}
\end{document}