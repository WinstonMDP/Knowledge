\documentclass[oneside]{book}

\usepackage[utf8]{inputenc}
\usepackage[T2A]{fontenc}
\usepackage[russian]{babel}
\usepackage[left = 0.3\textwidth, right = 0.3\textwidth]{geometry}
\usepackage{parskip}
\usepackage[fleqn]{amsmath}
\usepackage{mathtools}
\usepackage{amsfonts}
\usepackage{amssymb}
\usepackage{graphicx}
\usepackage{hyperref}
\usepackage{bookmark}
\usepackage{textcomp}

\setlength{\parskip}{0.03\textheight}

\graphicspath{{images/}}

\hypersetup{
    colorlinks,
    citecolor=black,
    filecolor=black,
    linkcolor=black,
    urlcolor=black
}

\newcommand{\meta}[1]{\text{<}#1\text{>}}
\newcommand{\set}[1]{\left\{#1\right\}}

\title{Трансляторы}
\date{\today}
\author{WinstonMDP}

\begin{document}
    \maketitle

    \tableofcontents

    \chapter{}
    \textbf{Компилятор} - это программа,
    переводящая текст программы
    с одного языка на другой.

    \textbf{Интерпретатор} - это программа,
    выполняющая код программы,
    не переводя её на другой язык.

    \textbf{Компоновщик (линкер)} - это программа,
    выполняющая разрешение внешних
    адресов памяти, по которым код из одного
    файла может обращаться к информации из другого файла.

    \textbf{Загрузчик} - это программа,
    которая помещает все выполнимые
    объектные файлы в память для
    выполнения.

    Компиляция состоит из анализа и синтеза.

    В течение компиляции код может
    переводиться по цепочке в несколько
    промежуточных представлений.

    Таблица символов содержит в себе информацию,
    которая накапливается на протяжении компиляции.

    \textbf{Проход (pass)} - это этапы компиляции,
    преобразующие один файл в другой
    (необязательно в файл с целевым кодом).

    \chapter{Лексический анализ (сканирование)}
    \textbf{Лексема} - это значащая последовательность символов кода.

    \textbf{Токен} - это значение <имя токена, значение атрибута>,
    представляющее лексему, где значение атрибута
    указывает на запись в таблице символов.

    \chapter{Синтаксический анализ (парсинг/разбор)}
    Синтаксический анализатор структурирует токены в синтаксическое дерево.

    \textbf{Контекстно-свободная грамматика (КС-грамматика)}.

    \textbf{Терминальный символ} - это элементарный символ языка,
    определяемый грамматикой.

    \textbf{Нетерминальный символ} - это множество строк терминалов,
    заданное продукцией.

    \textbf{Продукция} - это определение конкретного нетерминального символа.
    Записывается как $ a \rightarrow b $,
    где $ a $ - нетерминал, называемый заголовком
    (левой частью) продукции, $ b $ - последовательность
    (декартово произведение, если про множества)
    нетерминалов и/или (объединение, если про множества) терминалов
    (последовательность может быть пустой, что соответствует пустой строке или
    пустому множеству), называемая телом (правой частью) продукции.

    Контекстно-свободная грамматика имеет четыре компонента:
    \begin{enumerate}
        \item Множество терминальных символов.
        \item Множество нетерминнальных символов.
        \item Множество продукций.
        \item Стартовый нетерминальный символ.
    \end{enumerate}

    Грамматика выводит (порождает) строки, начиная со стартового символа.

    \textbf{Язык} - это множество строк терминалов, определяемые грамматикой.

    \textbf{Синтаксический анализ} - это выяснение для строки терминалов
    способа её вывода из стартового символа грамматики.

    \textbf{Дерево разбора} - это дерево, представляющее
    порождение конкретной строки языка.

    \textbf{Неоднозначная грамматика} - это грамматика,
    имеющая более одного дерева разбора для какой-то строки.

    \textbf{Форма Бэкуса-Наура (BNF)} - это другая форма записи КС грамматики.

    Для любой КС грамматики существует парсер, который требует для разбора
    строки из $ n $ терминалов время, не превышающее $ O\left(n^3\right) $.

    Вручную обычно используют нисходящий разбор.
    Восходящий разбор работает в большем количестве случаем, чем нисходящий,
    поэтому его используют в автоматическом построении парсеров.

    \textbf{Синтаксически управляемая трансляция} - это трансляция,
    выполняемая путём присоединения правил или программных фрагментов
    к продукциям грамматики.

    \textbf{Атрибут} - это некоторая величина, связанная с программной конструкцией.

    \textbf{Синтаксически управляемое определение} связывается:
    \begin{enumerate}
        \item С каждым грамматическим символом множеством атрибутов.
        \item С каждой продукцией множеством семантических правил для
            вычисления значений атрибутов, связанных с символами продукции.
    \end{enumerate}

    \textbf{Простое синтаксически управляемое определение} - это
    синтаксически управляемое определение, в котором атрибуты идут
    в том же порядке, что и соответствующие терминалы и нетерминалы.

    \textbf{Синтезированный атрибут} - это атрибут узла дерева разбора,
    значение которого определяется на основании атрибутов дочерних узлов
    этого узла и атрибутов самого этого узла.

    \textbf{Аннотированное дерево разбора} - это дерево разбора с указанием значений
    атрибутов в каждом узле.

    \textbf{(Синтаксически управляемая) схема трансляции} - это запись присоединённых
    к продукциям грамматики программных фрагментов.

    \textbf{Синтаксический анализ методом рекурсивного спуска} - это способ
    нисходящего синтаксического анализа, при котором для обработки входной
    строки используется множество рекурсивных процедур, где с каждым нетерминалом
    грамматики связана своя процедура.

    \textbf{Предикативный (предсказывающий) синтаксический анализ} - это
    синтаксический анализ методом рекурсивного спуска, при котором
    сканируемый символ однозначно определяет поток управления в теле
    процедуры для каждого нетерминала.

    Предикативный анализ делает выбор продукции на основе первых терминалов продукции.
    Такой анализ возможен, когда первые нетерминал у двух любых продукций
    различен.

    \textbf{Леворекурсивная продукция} - это продукция, в которой
    первый символ продукции есть заголовок продукции.

    Синтаксический анализатор методом рекурсивного спуска зацикливается
    на леворекурсивных продукциях.

    \chapter{Семантический анализ}
    Семантический анализатор проверяет синтаксическое дерево на корректность.

    \chapter{Генерация промежуточного кода}
    Генерация кода для абстрактной вычислительной машины.

    \chapter{Оптимизация кода}
    Оптимизация промежуточного кода.

    \chapter{Генерация кода}
    Генерация кода на целевом языке.
\end{document}
