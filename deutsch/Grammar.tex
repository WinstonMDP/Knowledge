\documentclass[oneside]{book}

\usepackage[utf8]{inputenc}
\usepackage[T2A]{fontenc}
\usepackage[german, russian]{babel}
\usepackage[left = 0.3\textwidth, right = 0.3\textwidth]{geometry}
\usepackage{parskip}
\usepackage[fleqn]{amsmath}
\usepackage{multirow}
\usepackage{hyperref}

\hypersetup{
    colorlinks,
    citecolor=black,
    filecolor=black,
    linkcolor=black,
    urlcolor=black
}

\title{Немецкий язык}
\date{\today}
\author{Мы}

\begin{document}
    \maketitle

    \tableofcontents

    \chapter{Фонетика}
    \section{Произношение буквосочетаний}
    ei, ey, ai, ay
    \begin{math}
        \longrightarrow
    \end{math}
    [ai]

    ui
    \begin{math}
        \longrightarrow
    \end{math}
    [ui]

    eu, \"au
    \begin{math}
        \longrightarrow
    \end{math}
    [oi]

    au
    \begin{math}
        \longrightarrow
    \end{math}
    [au]

    ie, ih, ieh
    \begin{math}
        \longrightarrow
    \end{math}
    [i:]

    sch
    \begin{math}
        \longrightarrow [\varsigma]
    \end{math}

    tsch
    \begin{math}
        \longrightarrow [t\varsigma]
    \end{math}

    st
    \begin{math}
        \longrightarrow [\varsigma t]
    \end{math}

    sp
    \begin{math}
        \longrightarrow [\varsigma p]
    \end{math}

    rh
    \begin{math}
        \longrightarrow
    \end{math}
    [r]

    ph
    \begin{math}
        \longrightarrow
    \end{math}
    [f]

    th, dt
    \begin{math}
        \longrightarrow
    \end{math}
    [t]

    tz, ts
    \begin{math}
        \longrightarrow
    \end{math}
    [ts]

    ss
    \begin{math}
        \longrightarrow
    \end{math}
    [s]

    ng
    \begin{math}
        \longrightarrow [\eta]
    \end{math}

    nk
    \begin{math}
        \longrightarrow [\eta k]
    \end{math}

    ck
    \begin{math}
        \longrightarrow
    \end{math}
    [k]

    \subsection{Произношение ch}
    [x]:
    \begin{enumerate}
        \item После a, o, u, au.
    \end{enumerate}

    \begin{math}
        [\zeta]
    \end{math}:
    \begin{enumerate}
        \item После i, e, \"a, \"o, \"u, ei, eu, \"au, l, n, r.
        \item В суффиксе chen.
    \end{enumerate}

    [k]:
    \begin{enumerate}
        \item Перед s, но не всегда.
        \item Перед r.
    \end{enumerate}

    Примечания:
    \begin{enumerate}
        \item
        \begin{math}
            [x]
        \end{math}
        иногда может переходить в
        \begin{math}
            [\zeta]
        \end{math}
        в результате грамматических преобразований.

        \item В некоторых словах встречаются
        \begin{math}
            [\varsigma], [t\varsigma]
        \end{math}.
    \end{enumerate}

    \section{Долгий и краткий гласный}
    Краткий:
    \begin{enumerate}
        \item Перед двойной согласной.

        \item Перед \\
        x, \\
        pf, tz, nk, \\
        ck, ch, \\
        sch.

        \item В безударном закрытом слоге.
    \end{enumerate}

    Долгий:
    \begin{enumerate}
        \item Двойная гласная.

        \item Перед \\
        hl, hm, hn, hr, h, \\
        \ss, \\
        bl, dl, gl, \\
        br, dr, gr, \\
        gd, gs, \\
        ks, ts.

        \item В ударном открытом слоге.

        \item В ударном закрытом слоге, если
        есть форма с открытым слогом.
    \end{enumerate}

    Полудолгий:
    \begin{enumerate}
        \item В безударном открытом слоге.
    \end{enumerate}

    \section{Произношение r}
    [r]:
    \begin{enumerate}
        \item В начале слога.
        \item После звонкого согласного звука.
    \end{enumerate}

    [x]:
    \begin{enumerate}
        \item После глухого согласного звука.
    \end{enumerate}

    [a]:
    \begin{enumerate}
        \item После долгого гласного звука внутри слога.
    \end{enumerate}

    \begin{math}
        [\mu]
    \end{math}
    \begin{enumerate}
        \item После краткого гласного звука внутри слога.
    \end{enumerate}

    \chapter{Артикли}
    \section{Неопределённые}
    \begin{center}
        \begin{tabular}{|c|c|c|}
            \hline
                   & Ед. ч. & Мн. ч.                \\ \hline
            М. р.  & ein    & \multirow{3}{*}{-}    \\ \cline{1-2}
            Ж. р.  & eine   &                       \\ \cline{1-2}
            Ср. р. & ein    &                       \\ \hline
        \end{tabular}
    \end{center}

    \section{Определённые}
    \begin{center}
        \begin{tabular}{|c|c|c|}
            \hline
                   & Ед. ч. & Мн. ч.                \\ \hline
            М. р.  & der    & \multirow{3}{*}{die}  \\ \cline{1-2}
            Ж. р.  & die    &                       \\ \cline{1-2}
            Ср. р. & das    &                       \\ \hline
        \end{tabular}
    \end{center}

    \chapter{Местоимения}
    \section{Личные}
    \begin{center}
        \begin{tabular}{|c|c|c|}
            \hline
                   & Ед. ч.      & Мн. ч.   \\ \hline
            1 лицо & ich         & wir      \\ \hline
            2 лицо & du          & ihr      \\ \hline
            3 лицо & er, sie, es & sie, Sie \\ \hline
        \end{tabular}
    \end{center}

    \section{Притяжательные}
    \begin{center}
        \begin{tabular}{|c|c|c|}
            \hline
                   & Ед. ч.          & Мн. ч.   \\ \hline
            1 лицо & mein            & unser    \\ \hline
            2 лицо & dein            & euer     \\ \hline
            3 лицо & sein, ihr, sein & ihr, Ihr \\ \hline
        \end{tabular}
    \end{center}

    К основе местоимения присоединяется окончание, соответствующее
    заменяемому артиклю.

    \chapter{Времена}
    \section{Pr\"asens}
    Окончание отпадает от инфинитива и основа глагола
    получает новое, соответствующее лицу и числу.

    \begin{center}
        \begin{tabular}{|c|c|c|}
            \hline
            Лицо               & Число & Окончание \\ \hline
            \multirow{2}{*}{1} & eд.   & e         \\ \cline{2-3}
                               & мн.   & en        \\ \hline
            \multirow{2}{*}{2} & eд.   & st        \\ \cline{2-3}
                               & мн.   & t         \\ \hline
            \multirow{2}{*}{3} & eд.   & t         \\ \cline{2-3}
                               & мн.   & en        \\ \hline
        \end{tabular}
    \end{center}

    Общее:
    \begin{enumerate}
        \item 2 л. ед. ч.: Если инфинитив оканчивается на sen, \ss en, zen, xen,
        то окончание t.

        \item 2 л. ед. ч. и мн. ч., 3 л. ед. ч.: Если инфинитив оканчивается на
        ten, den, dnen, knen, gnen, pnen, bnen, fnen, chnen, то в формах
        с неизменяемой корневой гласной между основой и окончанием e.
    \end{enumerate}

    Слабые:
    \begin{enumerate}
        \item Если инфинитив глагола оканчивается на rn, ln, то отпадает
        только последняя буква, а в 1 и 3 л. мн. ч. окончание n.
    \end{enumerate}

    Сильные:
    \begin{enumerate}
        \item Инфинитив оканчивается только на en.
        \item Изменяемые корневые гласные в 2 л. и 3 л. ед. ч.
        \begin{enumerate}
            \item е - i, если краткий, e - ie, если долгий.
            \item a - \"a, au - \"au, o - \"o.
        \end{enumerate}
    \end{enumerate}

    \chapter{Падежи}
    \section{Nominativ}
    Вопросы: wer? was?

    Единственное число:
    \begin{enumerate}
        \item Если по артиклю или заменяющему его слову нельзя определить род
        существительного, то прилагательное получает окончание, соответствующее
        роду (сильное склонение).

        \item Если по артиклю или заменяющему его слову можно определить
        род существительного, то прилагательное получает окончание e.
    \end{enumerate}

    Множественное число:
    \begin{enumerate}
        \item Если нет артикля или заменяющего его слова,
        то прилагательное получает окончание e.

        \item Если есть артикль или заменяющее его слово,
        то прилагательное получает окончание en.
    \end{enumerate}

    \section{Akkusativ}
    Вопросы: wen? was?

    Прилагательные ед. ч. м. р. получают окончание en.
\end{document}
