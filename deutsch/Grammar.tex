\documentclass[oneside]{book}

\usepackage[utf8]{inputenc}
\usepackage[T2A]{fontenc}
\usepackage[german, russian]{babel}
\usepackage[left = 0.3\textwidth, right = 0.3\textwidth]{geometry}
\usepackage{parskip}
\usepackage[fleqn]{amsmath}
\usepackage{multirow}

\title{Немецкий язык}
\date{\today}
\author{Мы}

\begin{document}
	\maketitle

	\chapter{Фонетика}
	\section{Произношение буквосочетаний}
	ei, ey, ai, ay
	\begin{math}
		\longrightarrow
	\end{math}
	[ai]

	ui
	\begin{math}
		\longrightarrow
	\end{math}
	[ui]

	eu, "au
	\begin{math}
		\longrightarrow
	\end{math}
	[oi]

	au
	\begin{math}
		\longrightarrow
	\end{math}
	[au]

	ie, ih, ieh
	\begin{math}
		\longrightarrow
	\end{math}
	[i:]

	sch
	\begin{math}
		\longrightarrow [\varsigma]
	\end{math}

	tsch
	\begin{math}
		\longrightarrow [t\varsigma]
	\end{math}

	st
	\begin{math}
		\longrightarrow [\varsigma t]
	\end{math}

	sp
	\begin{math}
		\longrightarrow [\varsigma p]
	\end{math}

	rh
	\begin{math}
		\longrightarrow
	\end{math}
	[r]

	ph
	\begin{math}
		\longrightarrow
	\end{math}
	[f]

	th, dt
	\begin{math}
		\longrightarrow
	\end{math}
	[t]

	tz, ts
	\begin{math}
		\longrightarrow
	\end{math}
	[ts]

	ss
	\begin{math}
		\longrightarrow
	\end{math}
	[s]

	ng
	\begin{math}
		\longrightarrow [\eta]
	\end{math}

	nk
	\begin{math}
		\longrightarrow [\eta k]
	\end{math}

	ck
	\begin{math}
		\longrightarrow
	\end{math}
	[k]

	\subsection{Произношение ch}
	[x]:
	\begin{enumerate}
		\item После a, o, u, au.
	\end{enumerate}

	\begin{math}
		[\zeta]
	\end{math}:
	\begin{enumerate}
		\item После i, e, "a, "o, "u, ei, eu, "au, l, n, r.
		\item В суффиксе chen.
	\end{enumerate}

	[k]:
	\begin{enumerate}
		\item Перед s, но не всегда.
		\item Перед r.
	\end{enumerate}

	Примечания:
	\begin{enumerate}
		\item
		\begin{math}
			[x]
		\end{math}
		иногда может переходить в
		\begin{math}
			[\zeta]
		\end{math}
		в результате грамматических преобразований.

		\item В некоторых словах встречаются
		\begin{math}
			[\varsigma], [t\varsigma]
		\end{math}.
	\end{enumerate}

	\section{Долгий и краткий гласный}
	Краткий:
	\begin{enumerate}
		\item Перед двойной согласной.

		\item Перед x, pf, tz, nk, \\
		ck, ch, \\
		sch.

		\item В безударном закрытом слоге.
	\end{enumerate}

	Долгий:
	\begin{enumerate}
		\item Двойная гласная.

		\item Перед hl, hm, hn, hr, h, \\
		"s, \\
		bl, dl, gl, \\
		br, dr, gr, \\
		gd, gs, \\
		ks, ts.

		\item В ударном открытом слоге.

		\item В ударном закрытом слоге, если
		есть форма с открытым слогом.
	\end{enumerate}

	Полудолгий:
	\begin{enumerate}
		\item В безударном открытом слоге.
	\end{enumerate}

	\section{Произношение r}
	[r]:
	\begin{enumerate}
		\item В начале слога.
		\item После звонкого согласного звука.
	\end{enumerate}

	[x]:
	\begin{enumerate}
		\item После глухого согласного звука.
	\end{enumerate}

	[a]:
	\begin{enumerate}
		\item После долгого гласного звука внутри слога.
	\end{enumerate}

	\begin{math}
		[\mu]
	\end{math}
	\begin{enumerate}
		\item После краткого гласного звука внутри слога.
	\end{enumerate}
\end{document}