\documentclass[main.tex]{subfiles}

\begin{document}
\chapter{Аксиомы}
\section{Равенства}
\begin{flalign*}
    \ax{x \equiv y \to \left(x \in z \tot y \in z\right)}
\end{flalign*}

\section{Пары}
\begin{flalign*}
    &\ax {
    x \in \set{y, z}
    =
    \left[
    \begin{aligned}
        &x \equiv y \\
        &x \equiv z
    \end{aligned}
    \right.
    } \\
    &\ntnt{\set{x, y}}{\tx{множество из $ x, y $}}
\end{flalign*}

\section{Объединения}
\begin{flalign*}
    &\ax {
    x \in \cup \ y
    =
    \exists z \ x \in z \in y
    } \\
    &\ntnt{\cup x}{\tx{объединение $ x $}}
\end{flalign*}

\section{Степени}
\begin{flalign*}
    &\ax{x \in \mathcal{P} \ y = x \subseteq y} \\
    &\ntnt{\mathcal{P} \ x}{\tx{множество всех подмножеств $ x $}}
\end{flalign*}

\section{Выделения}
\begin{flalign*}
    \set{\alpha \in x \mid y \ \alpha}
    =
    \exel z \
    w
    \to
    \left(
    w \in z
    \tot
    \begin{cases}
        w \in x \\
        y \ w
    \end{cases}
    \right)
\end{flalign*}
Выводится из аксиомы подстановки.

\section{Бесконечности}
\begin{flalign*}
    \ax{\exists x - \tx{индуктивное}}
\end{flalign*}

\section{Выбора}
В этой системе, которая вроде является системой Мартина-Лёфа Пера, вроде доказуема.
\begin{flalign*}
    \varnothing \not\in x
    \to
    \exists y
    \begin{cases}
        y - \tx{функция из $ x $} \\
        y \subseteq x \times \cup x \\
        \forall z \in x \ y_z \in z
    \end{cases}
\end{flalign*}

\section{Регулярности (фундированности)}
\begin{flalign*}
    \ax{x \neq \varnothing \to \exists y \in x \ \forall z \in x \ z \not\in y}
\end{flalign*}

\section{Подстановки}
\begin{flalign*}
    \ax {
    \argtype{y}{\bb{S}}
    \to
    z - \tx{функциональное свойство}
    \to
    \exists z \ w
    \to
    \left(w \in z \tot \exists i \in y \ x \ i \ w\right)
    }
\end{flalign*}
\end{document}
