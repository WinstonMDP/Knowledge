\documentclass[main.tex]{subfiles}

\begin{document}
\chapter{Центр}
\section{Конспект}
\subsection{Логика высказываний}
\subsubsection{Синтаксис}
В языке есть переменные $ \tx{Var} = \set{p_1, p_2, \ldots, p_n} $.

Есть логические связки: $ \wedge, \vee, \to, \neg, \top, \bot $.

Пропозициональные константы: $ \top, \bot $.

Формулы логики высказываний (пропозициональные формулы):
\begin{enumerate}
    \item Переменные и константы - это формулы.
    \item Если $ x $ и $ y $ - формулы, то $ (x \wedge y), (x \vee y), (\neg x) $ -
        формулы.
    \item Любая формула получается по правилам 1 и 2.
\end{enumerate}

Соглашение об опускании скобок:
\begin{enumerate}
    \item Опускаем внешние скобки.
    \item Если приоритет позволяет, скобки опускаем.
\end{enumerate}

$ \tx{Fm} $ - это множество всех формул.

\subsubsection{Семантика}
\begin{flalign*}
    &\bb{B} = \set{0, 1} \\
    &x - \tx{булева функция} = \exists y \in \bb{N} \ x \in \bb{B}^{\bb{B}^y} \\
    &x - \tx{оценка} = x \in \bb{B}^{\tx{Fm}} \\
    &\forall z - \tx{оценка} \ z_{x \wedge y} \equiv \tx{est}_x * \tx{est}_y \\
    &\forall z - \tx{оценка} \ z_{x \vee y} \equiv \tx{max} \ \tx{est}_x \ \tx{est}_y \\
    &\forall z - \tx{оценка} \ z_{\neg x} \equiv 1 - \tx{est}_x \\
    &x - \tx{тавтология} = \forall y - \tx{оценка} \ y_x \equiv 1 \\
    &x - \tx{выполнимое} = \exists y - \tx{оценка} \ y_x \equiv 1 \\
    &x \doteqdot y = \forall z - \tx{оценка} \ z_x \equiv z_y
\end{flalign*}

Таблица истинности несёт информацию о значении формулы на всех оценках.
Функцию, которая кодирует таблицу истинности формулы $ x $, обозначают $ \varphi \ x $.

Теорема о функциональной полноте:
\begin{flalign*}
    y \in \bb{N} \to x \in \bb{B}^{\bb{B}^y} \to \exists z \ x \equiv \varphi \ z
\end{flalign*}

Для любой формулы достаточно только $ \wedge, \neg $ или $ \vee, \neg $, как удобнее.

Элементарная конъюнкция - это конъюнкция некоторых переменных и отрицания некоторых
переменных.

Дизъюнктивная нормальная форма (д.н.ф.) - это дизъюнкция элементарных конъюнкций.

Всякая формула эквивалентна некоторой д.н.ф.

Если в эквивалентных формулах сменить все конъюнкции на дизъюнкции, а дизъюнкции на
конъюнкции, то формулы останутся эквивалентными.

Результаты подстановки в эквивалентные формулы эквивалентные формулы будут
эквивалентны.

Существует алгоритм преобразования любой формулы в д.н.ф. с помощью
последовательного применения основных тождеств булевой алгебры.

\subsection{Логика предикатов (логика первого порядка)}
Индивидные переменные - это переменные, не обозначающие высказывания, например множества.

Появляются предикаты.

Появляются кванторы.

\section*{Семантика}
\begin{flalign*}
    &z - \tx{предикат на множестве $ x \not\equiv \varnothing $ от $ y $ переменных}
    =
    z \in \bb{B}^{x^y} \\
    &x - \tx{функция на множестве $ y $ от $ z $ переменных}
    =
    x \in y^{y^z}
\end{flalign*}

Модель - это непустое множество, на котором фиксирован некоторый, может быть, бесконечный
набор предикатов, функций и констант.

Сигнатура - это совокупность имён предикатов, функций и констант.

\section*{Синтаксис}
Алфавит: символы сигнатуры, переменные.

Свободные переменные $ \tx{FrVar} $.

Связанные переменные $ \tx{BdVar} $.

Все переменные индивидные.

Логические связки: $ \wedge, \vee, \neg, \top, \bot $.

Кванторы: $ \forall, \exists $.

Вспомогательные символы: $ ( \ ) \ , $.

Термы ($ \tx{Tm} \ \Sigma $):
\begin{enumerate}
    \item Свободные переменные.
    \item Константы.
    \item $ f \in \tx{Func} \ \Sigma  \to f(t_1, \ldots, t_n) $
        (это текст, а не действительное применение функции).
    \item Только эти 3 правила.
\end{enumerate}

Формулы ($ \tx{Fm} \ \Sigma $):
\begin{enumerate}
    \item $ t_1, \ldots, t_n \in \tx{Tm} \ \Sigma \to P \in \tx{Pred} \ \Sigma \to
        P(t_0, \ldots, t_n) $. Такие формулы называются атомарными.
    \item $ A \andc B \in \tx{Fm} \ \Sigma \to (A \wedge B), (A \vee B), (\neg A) $.
    \item $ \top, \bot $. Тоже являются атомарными.
    \item Связанная переменная $ x $ не входит в формулу $ A \to A[a/x] $,
        где $ a \in \tx{FrVar} $.
\end{enumerate}

Замкнутая формула (предложение) - формула без свободных переменных.

Значение терма вычисляется рекурсией по построению. Обозначается $ x_y $ (((.

\begin{flalign*}
    &M \models P(t_1, \ldots, t_n) \tot P_M ((t_1)_M, \ldots, (t_n)_M) \equiv 1,
    \tx{где $ P(t_1, \ldots, t_n) $ - атомарная} \\
    &\tx{$ P(t_1, \ldots, t_n) $ истинно в модели $ M $} \\
    &M \models B \wedge C
    \tot
    \begin{cases}
        M \models B \\
        M \models C
    \end{cases} \\
    &M \models \neg B \tot M \not\models B \\
    &M \models B \vee C
    \tot
    \left[
    \begin{aligned}
        &M \models B \\
        &M \models C
    \end{aligned}
    \right. \\
    &M \models \left(\forall x \ B[a/x]\right)
    \tot
    \tx{для любого $ c \in M \ M \models B[a/\underline{c}] $} \\
    &M \models \left(\exists x \ B[a/x]\right)
    \tot
    \tx{существует $ c \in M \ M \models B[a/\underline{c}] $}
\end{flalign*}

Выразимость (определимость) предиката - это если он соответствует какой-то формуле.

Гомоморфизм моделей - это функция, у которой dom и rng имеют одну сигнатуру.

Подмодель - это подмножество, замкнутое относительно всех функций модели.

Выразимые предикаты сохраняются под действием автоморфизма.

Общезначимость и выполнимость формулы. Общезначимость - это тавтология, но не логике
высказываний. Эквивалентность.

Предварённая нормальная форма.

Теория (аксиоматическая) - это множество замкнутых формул, называемых не логическими аксиомами.

Модель теории - это модель, в которой выполняются все не логические аксиомы теории.

Нормальная модель - это модель, где равны только одинаковые символы
(не символы, не помню, как называется).

Мощные теории (первого порядка, насколько я понял) (мощные, так как через них многие другие теории выражаются):
\begin{enumerate}
    \item $ \in $ \ ZFC.
    \item $ +, \ *, \ S, \ 0, \ = $ \ PA.
\end{enumerate}
Хотя было сказано, что PA более слабый.

Аксиомы равенства (классические аксиомы, связанные с равенством в любой модели):
\begin{enumerate}
    \item $ = $ есть отношение эквивалентности.
    \item $ x_i = x_i' \to f(x_1, \ldots, x_i, \ldots, x_n) =
        f(x_1, \ldots, x_i', \ldots, x_n) $.
    \item То же для предикатов.
\end{enumerate}

Нормальная модель - это модель, где равны только одинаковые символы
(не символы, не помню, как называется).

Исчисление предикатов.

Аксиомы:
\begin{enumerate}
    \item Тавтологии исчисления высказываний.
    \item $\left(\forall x \ A[a/x]\right) \to A[a/t] $
    \item $ A[a/t] \to \exists x \ A[a/x] $
\end{enumerate}

Правила вывода:
\begin{enumerate}
    \item Modus ponens. $ \frac{A \ A \to B}{B} $
    \item $ \frac{A \to B}{A \to B[a/x]} $
    \item $ \frac{B \to A}{\exists x B[a/x] \to A} $
\end{enumerate}

Вывод из множества гипотез $ \Gamma $ - это конечная последовательность формул,
каждая из которых либо аксиома, либо гипотеза, либо получается из предыдущих по правилам
вывода.

\textbf{Теорема о дедукции:}
\begin{flalign*}
    \Gamma \andc A - \tx{замкнутые}
    \to
    \left(\Gamma, A \vdash B \tot \Gamma \vdash A \to B\right)
\end{flalign*}

\textbf{Теорема Гёделя о полноте исчисления предикатов:}
\begin{flalign*}
    T \vdash A \tot T \models A
\end{flalign*}
\end{document}
