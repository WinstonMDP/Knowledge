\documentclass[oneside]{book}

\usepackage[utf8]{inputenc}
\usepackage[T2A]{fontenc}
\usepackage[russian]{babel}
\usepackage[left = 0.3\textwidth, right = 0.3\textwidth]{geometry}
\usepackage{parskip}
\usepackage[fleqn]{amsmath}
\usepackage{mathtools}
\usepackage{amsfonts}
\usepackage{amssymb}
\usepackage{graphicx}
\usepackage{hyperref}
\usepackage{bookmark}
\usepackage{textcomp}

\setlength{\parskip}{0.03\textheight}

\graphicspath{{images/}}

\hypersetup{
    colorlinks,
    citecolor=black,
    filecolor=black,
    linkcolor=black,
    urlcolor=black
}

\newcommand{\meta}[1]{\text{<}#1\text{>}}
\newcommand{\sequence}[1]{\left\{#1\right\}}
\newcommand{\set}[1]{\left\{#1\right\}}
\newcommand{\simsub}{\stackrel{<}{\thicksim}}

\title{Логика}
\date{\today}
\author{Мы}

\begin{document}
	\maketitle

    \tableofcontents
    
    \chapter{}
    \textbf{Автонимный способ обозначения} - это
    способ обозначения,
    при котором формальные выражения обозначаются так же,
    как и их значения.

    \textbf{Высказывательная форма}.

    \textbf{Именная форма} - это
    выражение с переменной.

    \textbf{Связанные переменные} - это
    переменные, вместо которых
    нельзя подставить значение.

    \textbf{Основания математики} - это
    раздел (в книге сказано "аспект")
    математической логики,
    изучающий объекты математики,
    истинные свойства этих объектов,
    на основании которых можно вести рассуждения,
    а также "сохраняющие истину" способы рассуждений.
\end{document}
