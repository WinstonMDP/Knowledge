\documentclass[oneside]{book}

\usepackage[utf8]{inputenc}
\usepackage[T2A]{fontenc}
\usepackage[russian]{babel}
\usepackage[left = 0.3\textwidth, right = 0.3\textwidth]{geometry}
\usepackage{parskip}
\usepackage[fleqn]{amsmath}
\usepackage{mathtools}
\usepackage{amsfonts}
\usepackage{amssymb}
\usepackage{graphicx}
\usepackage{hyperref}
\usepackage{bookmark}
\usepackage{textcomp}

\setlength{\parskip}{0.03\textheight}

\graphicspath{{images/}}

\hypersetup{
    colorlinks,
    citecolor=black,
    filecolor=black,
    linkcolor=black,
    urlcolor=black
}

\newcommand{\meta}[1]{\text{<}#1\text{>}}
\newcommand{\sequence}[1]{\left\{#1\right\}}
\newcommand{\set}[1]{\left\{#1\right\}}
\newcommand{\simsub}{\stackrel{<}{\thicksim}}

\title{Логика}
\date{\today}
\author{Мы}

\begin{document}
	\maketitle

    \tableofcontents
    
    \chapter{Аксиоматический метод}
    Базовое понятие - это неопределяемое понятие.

    \section{}
    Зафиксировать базовые понятия.

    \section{}
    Зафиксировать аксиомы, связывающие понятия.

    \section{}
    Выводить следствия по правилам логики.

    \chapter{Теория множеств}
    \section{Базовые понятия}
    Принадлежность ($ x \in y $).

    \section{Аксиомы}
    \subsection{Равенства}
    \begin{flalign*}
        x = y \longrightarrow \forall z \ \left(x \in z \longrightarrow y \in z\right)
    \end{flalign*}

    \subsection{Пары}
    \begin{flalign*}
        \exists z \ \forall u \
        \left(
        u \in z
        \Leftrightarrow
        \left[
        \begin{aligned}
            &u = x \\
            &u = y
        \end{aligned}
        \right.
        \right)
    \end{flalign*}

    \subsection{Объединения}
    \begin{flalign*}
        \exists y \ \forall u \
        \left(
        u \in y
        \Leftrightarrow
        \exists z \
        \left\{
        \begin{aligned}
            &u \in z \\
            &z \in x
        \end{aligned}
        \right.
        \right)
    \end{flalign*}

    \subsection{Степени}
    \begin{flalign*}
        \exists y \ \forall u \ \left(u \in y \Leftrightarrow u \subseteq x\right)
    \end{flalign*}

    \subsection{Выделения}
    \begin{flalign*}
        \set{\left.x \in A \ \right| \ \varphi(x)}
    \end{flalign*}

    \section{Определения}
    Класс $ = \set{x \ \left| \ \varphi(x)\right.} $.
    Не все классы являются множествами. Все множества являются классами.
    \begin{flalign*}
        x = y \Leftrightarrow \left(z \in x \Leftrightarrow z \in y\right)
    \end{flalign*}
    \begin{flalign*}
        x \subseteq y \Leftrightarrow \forall z \ \left(z \in x \longrightarrow z \in y\right)
    \end{flalign*}

    \chapter{Не обработанное}
    \textbf{Автонимный способ обозначения} - это
    способ обозначения,
    при котором формальные выражения обозначаются так же,
    как и их значения.

    \textbf{Высказывательная форма}.

    \textbf{Именная форма} - это
    выражение с переменной.

    \textbf{Связанные переменные} - это
    переменные, вместо которых
    нельзя подставить значение.

    \textbf{Основания математики} - это
    раздел (в книге сказано "аспект")
    математической логики,
    изучающий объекты математики,
    истинные свойства этих объектов,
    на основании которых можно вести рассуждения,
    а также "сохраняющие истину" способы рассуждений.
\end{document}
