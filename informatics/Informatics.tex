\documentclass[oneside]{book}

\usepackage[utf8]{inputenc}
\usepackage[T2A]{fontenc}
\usepackage[russian]{babel}
\usepackage[left = 0.3\textwidth, right = 0.3\textwidth]{geometry}
\usepackage{parskip}
\usepackage[fleqn]{amsmath}
\usepackage{multirow}
\usepackage{textcomp}
\usepackage{hyperref}

\title{Информатика}
\date{\today}
\author{Мы}

\hypersetup{
    colorlinks,
    citecolor=black,
    filecolor=black,
    linkcolor=black,
    urlcolor=black
}

\begin{document}
	\maketitle
    \tableofcontents
	\chapter{Программная инженерия}
	Любую программу можно написать на низкоуровневом
	языке - языке, наиболее приближённом к
	устройству компьютера. Однако, тем не менее,
	разрабатывают всё новые языки программирования.
	Далее рассматриваются причины, по которым
	возникает нужда в высокоуровневых языках.

	\section{Изменяемость}
	Программистам в течение жизненного цикла
	разработки ПО приходится изменять программу.
	По причине изменений требований к продукту
	или для устранения ошибки.

	Чтобы что-то изменить,
	нужно найти всю имплементацию этого чего-то
	в программе. C этим возникают две
	трудности: имплементация этого чего-то
	простирается в большой части кода -
	слишком много приходится править, чтобы
	внести нужное изменение -
	и код трудно читаем - сложно понять,
	где то, что нам нужно.

	Языки программирования создают такими,
	чтобы они как можно более полно
	решали данные проблемы.

	Часто добавляют "мультипарадигменные" \
	конструкции, которые должны матчаться в нашем мозгу
	с устоявшимися паттернами. Однако всевозможных
	паттернов настолько много, что данные вводящиеся
	конструкции только капля в море.

	Вторым способом бороться с данными проблемами,
	который работает всегда, является продолжающийся
	рефакторинг согласно сложности Джона.

    \chapter*{Трансляция кода}
    \textbf{Компилятор} - это программа,
    переводящая текст программы
    с одного языка на другой.

    \textbf{Интерпретатор} - это программа,
    выполняющая код программы,
    не переводя её на другой язык.

    \textbf{Компоновщик (линкер)} - это программа,
    выполняющая разрешение внешних
    адресов памяти, по которым код из одного
    файла может обращаться к информации из другого файла.

    \textbf{Загрузчик} - это программа,
    которая помещает все выполнимые
    объектные файлы в память для
    выполнения.

    Компиляция состоит из анализа и синтеза.

    В течение компиляции код может
    переводиться по цепочке в несколько
    промежуточных представлений.
\end{document}
