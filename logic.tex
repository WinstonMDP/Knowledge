\documentclass[oneside]{book}

\usepackage[utf8]{inputenc}
\usepackage[T2A]{fontenc}
\usepackage[russian]{babel}
\usepackage[left = 0.3\textwidth, right = 0.3\textwidth]{geometry}
\usepackage{parskip}
\usepackage[fleqn]{amsmath}
\usepackage{amssymb}
\usepackage{graphicx}
\usepackage{bookmark}
\usepackage{cancel}
\usepackage{mathtools}

\setlength{\parskip}{0.03\textheight}

\hypersetup{
    colorlinks,
    citecolor=black,
    filecolor=black,
    linkcolor=black,
    urlcolor=black
}

\newcommand{\bb}[1]{\mathbb{#1}}
\newcommand{\gr}[4]{#1 \ \left\{#2 \mid #3\right\} \ #4}
\newcommand{\greq}[3]{#1 \ \left\{#2\right\} \ #3}
\newcommand{\dfto}{\text{def-to}}
\newcommand{\ax}[1]{\text{ax} \ #1}
\newcommand{\type}[2]{#1 : #2}
\newcommand{\argtype}[2]{\left(\type{#1}{#2}\right)}
\newcommand{\exel}{\exists \text{-el} \ }
\newcommand{\cstr}{\text{cstr}}
\newcommand{\fnd}[1]{\left[#1\right]}
\newcommand{\set}[1]{\left\{#1\right\}}
\newcommand{\tot}{\leftrightarrow}
\newcommand{\tx}[1]{\text{#1}}
\newcommand{\ntnt}[2]{\tx{"$ #1 $"} - #2}
\newcommand{\lend}{\rule{\textwidth}{0.4pt}}
\newcommand{\notis}{\ \cancel{-} \ }
\newcommand{\andc}{\tx{-}}
\newcommand{\abin}{\smile}
\newcommand{\abind}{\ni}
\newcommand{\concat}{\ {++} \ }


\title{Логика}
\date{\today}
\author{WinstonMDP}

\begin{document}
\maketitle

\tableofcontents

\chapter{Теория множеств (ZFC)}
\section{Основание}
Три возможные формы с кванторами
\begin{flalign*}
    &\forall (y \ x) \ (z \ x) \mid \tx{в $ y \ x $ стоит первым символом} \\
    &\exists x \\
    &\exists x \ (y \ x) \\
    &\exists (y \ x) \ (z \ x) \mid \tx{в $ y \ x $ стоит первым символом}
\end{flalign*}
\begin{flalign*}
    &\gr{r}{}{}{\_\to\_} \\
    &\gr{n}{}{\_\to\_}{\_=\_} \\
    &\gr{n}{\_=\_}{\_\tot\_}{\exists\_\_} \\
    &\greq{n}{\exists\_\_}{\forall\_\_} \\
    &=\bot \\
    &\argtype{y}{\bb{T}_x} \to \left(y \to \bot = \overline{y}\right) \\
    &\ax{\left(x \tot y\right) \to \left(z \ x \leftrightarrow z \ y\right)}
\end{flalign*}

\section{Аксиомы}
\subsection{Равенства}
\begin{flalign*}
    \ax{x \equiv y \to \left(x \in z \tot y \in z\right)}
\end{flalign*}

\subsection{Пары}
\begin{flalign*}
    \ax{\exists \set{x, y}}
\end{flalign*}

\subsection{Объединения}
\begin{flalign*}
    \ax{\exists \left(\cup x\right)}
\end{flalign*}

\subsection{Степени}
\begin{flalign*}
    \ax{\exists \left(\mathcal{P} \ x\right)}
\end{flalign*}

\subsection{Выделения}
\begin{flalign*}
    \exists \set{\alpha \in x \mid y \ \alpha}
\end{flalign*}
Выводится из аксиомы подстановки.

\subsection{Бесконечности}
\begin{flalign*}
    \ax{\exists_{\bb{S}} \ x - \tx{индуктивное}}
\end{flalign*}

\subsection{Выбора}
?

\subsection{Регулярности (фундированности)}
\begin{flalign*}
    \ax{\exists y \in x \ \forall z \in x \ z \not\in y}
\end{flalign*}

\subsection{Подстановки}
\begin{flalign*}
    \ax {
    \argtype{y}{\bb{S}}
    \to
    \left(
    z
    \to
    \left[
    \begin{aligned}
        &\exists! w \ (x \ z \ w) \\
        &\nexists w \ (x \ z \ w)
    \end{aligned}
    \right.
    \right)
    \to
    \exists z \ w
    \to
    \left(w \in z \tot \exists i \in y \ x \ i \ w\right)
    }
\end{flalign*}

\section{Определения}
Класс - это $ \set{x \ \left| \ \varphi(x)\right.} $.
Не все классы являются множествами. Все множества являются классами.
\begin{flalign*}
    &\ax{\type{\bb{S}}{\bb{T}}} \\
    &\ax{\type{\_\in\_}{\bb{S} \to \bb{S} \to \bb{T}}} \\
    &\gr{t}{}{}{\_\in\_}
\end{flalign*}
\begin{flalign*}
    &x \tot y
    =
    \begin{cases}
        x \to y \\
        y \to x
    \end{cases} \\
    &\gr{r}{}{\_\to\_}{\_\tot\_} \\
    &\tx{$ x $ тогда и только тогда, когда $ y $}
\end{flalign*}
\begin{flalign*}
    &x \equiv y = z \to \left(z \in x \tot z \in y\right) \\
    &\gr{t}{}{}{\_\equiv\_}
\end{flalign*}
\begin{flalign*}
    &\exists! x \ y \ x
    =
    \begin{cases}
        \exists x \ y \ x \\
        z \to y \ z \to z \equiv \exel\fnd{\exists x \ y \ x}
    \end{cases} \\
    &\tx{Существует единственный $ x $ удовлетворяющий $ y \ x $}
\end{flalign*}
\begin{flalign*}
    \ax{x \equiv y \to z \ x \equiv z \ y}
\end{flalign*}
\begin{flalign*}
    &x \in \set{y, z, \ldots}
    =
    \left[
    \begin{aligned}
        &x \equiv y \\
        &x \equiv z \\
        &\ldots
    \end{aligned}
    \right. \\
    &\ntnt{\set{x, y, \ldots}}{\tx{множество из $ x, y, \ldots $}}
\end{flalign*}
\begin{flalign*}
    &x \subseteq y = \forall z \in x \ z \in y \\
    &\tx{$ x $ вложено в $ y $; $ x $ - подмножество $ y $}
\end{flalign*}
\begin{flalign*}
    &x \in \cup \ y
    =
    \exists z
    \begin{cases}
        x \in z \\
        z \in y
    \end{cases} \\
    &\ntnt{\cup x}{\tx{объединение $ x $}} \\
\end{flalign*}
\begin{flalign*}
    &x \in \mathcal{P} \ y = x \subseteq y \\
    &\ntnt{\mathcal{P} \ x}{\tx{множество всех подмножеств $ x $}}
\end{flalign*}
\begin{flalign*}
    &\set{\alpha \in x \mid z \ \alpha} \equiv y
    =
    w
    \to
    \left(
    w \in y
    \tot
    \begin{cases}
        w \in x \\
        z \ w
    \end{cases}
    \right) \\
    &x - \tx{пустое} = y \to y \not\in x\\
    &x - \tx{индуктивное}
    =
    \begin{cases}
        \forall y - \tx{пустое} \ y \in x \\
        \forall z \in x \ z \cup \set{z} \in x
    \end{cases} \\
    &\varnothing
    =
    \set{\alpha \in \exel\fnd{\exists_{\bb{S}} \ x - \tx{индуктивное}} \mid \bot} \\
    &x - \tx{транзитивное} = \cup x \subseteq x
\end{flalign*}
\begin{flalign*}
    &\cap x = \set{y \in \cup x \mid \forall z \in x \ y \in z} \\
    &\tx{Пересечение x}
\end{flalign*}
\begin{flalign*}
    &x \cup y = \cup\set{x, y} \\
    &\tx{Объединение $ x $ и $ y $}
\end{flalign*}
\begin{flalign*}
    &x \cap y = \cap\set{x, y} \\
    &\tx{Пересечение $ x $ и $ y $}
\end{flalign*}
\begin{flalign*}
    &x \setminus y = \set{\alpha \in x \mid \alpha \not\in y} \\
    &\tx{Разность $ x $ и $ y $}
\end{flalign*}
\begin{flalign*}
    &x \triangle y = (x \setminus y) \cup (y \setminus x) \\
    &\tx{Симметрическая разность $ x $ и $ y $}
\end{flalign*}
\begin{flalign*}
    &(x, y)_z \equiv w
    =
    \begin{cases}
        i \to j \to k \to t \to
        \left(
        \begin{cases}
            i \equiv k \\
            j \equiv t
        \end{cases}
        \tot
        z \ i \ j \equiv z \ k \ t
        \right) \\
        w \equiv z \ x \ y
    \end{cases} \\
    &\ntnt{(x, y)_z}{\tx{упорядоченная пара $ x $ и $ y $ с реализацией $ z $}}
\end{flalign*}
\begin{flalign*}
    &\tx{kur} \ x \ y = \set{\set{x}, \set{x, y}} \\
    &\tx{Упорядоченная пара Куратовского}
\end{flalign*}
\begin{flalign*}
    &x \times_y z \equiv w
    =
    \forall i \in w \ \exists j, k
    \begin{cases}
        j \in x \\
        k \in z \\
        i \equiv (j, k)_y
    \end{cases} \\
    &\ntnt{x \times_y z} {
    \tx{прямое (декартово) произведение $ x $ и $ w $ с реализацией} \\
    &\tx{упорядоченной пары $ y $}
    } \\
\end{flalign*}
\begin{flalign*}
    &x - \tx{бинарное отношение с реализацией упорядоченной пары $ y $}
    = \\
    &\exists z, w \ x \subseteq z \times_y w
\end{flalign*}
\begin{flalign*}
    &\tx{dom}_x \ y \equiv z
    =
    w \to i \to (w, i)_x \in y \to w \in z \\
    &\ntnt{\tx{dom}_x \ y} {
    \tx{область определения $ y $ с реализацией упорядоченной пары $ x $}
    }
\end{flalign*}
\begin{flalign*}
    &\tx{rng}_x \ y \equiv z = w \to i \to (w, i)_x \in y \to i \in z \\
    &\ntnt{\tx{rng}_x \ y} {
    \tx{область значений $ y $ с реализацией упорядоченной пары $ x $}
    }
\end{flalign*}


\section{Теоремы}
\begin{flalign*}
    &\exists! x \ x - \tx{пустое} \\
    &x \times_{\tx{kur}} y
    \equiv
    \set {
    \mathcal{P} \left(\mathcal{P} \left(x \cup y\right)\right) \
    \left| \
    \exists z, w, i
    \begin{cases}
        z \in x \\
        i \in y \\
        z \equiv (w, i)_{kur}
    \end{cases}
    \right.
    } \\
    &\begin{cases}
        x \equiv y \\
        z \equiv w
    \end{cases}
    \tot
    x \times z \equiv y \times w \\
    &x \not\equiv \varnothing
    \to
    y \not\equiv \varnothing
    \to
    \left(x \times y\right) \cup \left(y \times x\right) \equiv z \times w
    \to
    x \equiv y \equiv z \equiv w \\
    &\tx{dom}_{\tx{kur}} \ x
    =
    \set{\alpha \in \cup\cup x \mid \exists y \ (\alpha, y) \in x} \\
    &\tx{rng}_{\tx{kur}} \ x
    =
    \set{\alpha \in \cup\cup x \mid \exists y \ (y, \alpha) \in x}
\end{flalign*}

\chapter{Формальные языки}
\begin{flalign*}
    &a \neq \varnothing \tot a - \tx{алфавит} \\
    &a \in A \tot a - \tx{символ (буква)} \\
    &f: \underline{n} \rightarrow A \tot f - \tx{слово} \\
    &\varepsilon - \tx{пустое слово} \\
    &\varepsilon = \varnothing
\end{flalign*}

\chapter{Необработанное}
\textbf{Автонимный способ обозначения} - это
способ обозначения,
при котором формальные выражения обозначаются так же,
как и их значения.

\textbf{Высказывательная форма}.

\textbf{Именная форма} - это
выражение с переменной.

\textbf{Связанные переменные} - это
переменные, вместо которых
нельзя подставить значение.

\textbf{Основания математики} - это
раздел (в книге сказано "аспект")
математической логики,
изучающий объекты математики,
истинные свойства этих объектов,
на основании которых можно вести рассуждения,
а также "сохраняющие истину"{ }способы рассуждений.
\end{document}
