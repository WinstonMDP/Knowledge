\documentclass[oneside]{book}

\usepackage[utf8]{inputenc}
\usepackage[T2A]{fontenc}
\usepackage[russian]{babel}
\usepackage[left = 0.3\textwidth, right = 0.3\textwidth]{geometry}
\usepackage{parskip}
\usepackage[fleqn]{amsmath}
\usepackage{mathtools}
\usepackage{amsfonts}
\usepackage{amssymb}
\usepackage{graphicx}
\usepackage{hyperref}
\usepackage{bookmark}
\usepackage{textcomp}

\documentclass[oneside]{book}

\usepackage[utf8]{inputenc}
\usepackage[T2A]{fontenc}
\usepackage[russian]{babel}
\usepackage[left = 0.3\textwidth, right = 0.3\textwidth]{geometry}
\usepackage{parskip}
\usepackage[fleqn]{amsmath}
\usepackage{amssymb}
\usepackage{graphicx}
\usepackage{bookmark}
\usepackage{cancel}
\usepackage{mathtools}

\setlength{\parskip}{0.03\textheight}

\hypersetup{
    colorlinks,
    citecolor=black,
    filecolor=black,
    linkcolor=black,
    urlcolor=black
}

\newcommand{\bb}[1]{\mathbb{#1}}
\newcommand{\gr}[4]{#1 \ \left\{#2 \mid #3\right\} \ #4}
\newcommand{\greq}[3]{#1 \ \left\{#2\right\} \ #3}
\newcommand{\dfto}{\text{def-to}}
\newcommand{\ax}[1]{\text{ax} \ #1}
\newcommand{\type}[2]{#1 : #2}
\newcommand{\argtype}[2]{\left(\type{#1}{#2}\right)}
\newcommand{\exel}{\exists \text{-el} \ }
\newcommand{\cstr}{\text{cstr}}
\newcommand{\fnd}[1]{\left[#1\right]}
\newcommand{\set}[1]{\left\{#1\right\}}
\newcommand{\tot}{\leftrightarrow}
\newcommand{\tx}[1]{\text{#1}}
\newcommand{\ntnt}[2]{\tx{"$ #1 $"} - #2}
\newcommand{\lend}{\rule{\textwidth}{0.4pt}}
\newcommand{\notis}{\ \cancel{-} \ }
\newcommand{\andc}{\tx{-}}
\newcommand{\abin}{\smile}
\newcommand{\abind}{\ni}
\newcommand{\concat}{\ {++} \ }


\setlength{\parskip}{0.03\textheight}

\graphicspath{{images/}}

\hypersetup{
    colorlinks,
    citecolor=black,
    filecolor=black,
    linkcolor=black,
    urlcolor=black
}

\title{Логика}
\date{\today}
\author{WinstonMDP}

\begin{document}
\maketitle

\tableofcontents

\chapter{Теория множеств (ZFC)}
\section{Основание}
Три возможные формы с кванторами
\begin{flalign*}
    &\forall (y \ x) \ (z \ x) \mid \text{в $ y \ x $ стоит первым символом} \\
    &\exists x \\
    &\exists x \ (y \ x) \\
    &\exists (y \ x) \ (z \ x) \mid \text{в $ y \ x $ стоит первым символом}
\end{flalign*}
\begin{flalign*}
    &\gr{r}{}{}{\_\to\_} \\
    &\gr{n}{}{\_\to\_}{\df{\_}{\_}} \\
    &\gr{n}{\df{\_}{\_}}{\_\iff\_}{\exists\_\_} \\
    &\greq{n}{\exists\_\_}{\forall\_\_} \\
    &\df{}{\bot} \\
    &\argtype{y}{\bb{T}_x} \to \left(\df{y \to \bot}{\overline{y}}\right)
\end{flalign*}

\section{Аксиомы}
\subsection{Равенства}
\begin{flalign*}
    \ax{x \equiv y \to \left(x \in z \iff y \in z\right)}
\end{flalign*}

\subsection{Пары}
\begin{flalign*}
    \ax{\exists \set{x, y}}
\end{flalign*}

\subsection{Объединения}
\begin{flalign*}
    \ax{\exists \left(\cup x\right)}
\end{flalign*}

\subsection{Степени}
\begin{flalign*}
    \ax{\exists \left(\mathcal{P} \ x\right)}
\end{flalign*}

\subsection{Выделения}
\begin{flalign*}
    \exists \set{\alpha \in x \mid y \ \alpha}
\end{flalign*}
Выводится из аксиомы подстановки.

\subsection{Бесконечности}
\begin{flalign*}
    \ax{\exists_{\bb{S}} \ x - \text{индуктивное}}
\end{flalign*}

\subsection{Выбора}
?

\subsection{Регулярности (фундированности)}
\begin{flalign*}
    \ax{\exists y \in x \ \forall z \in x \ z \not\in y}
\end{flalign*}

\subsection{Подстановки}
\begin{flalign*}
    \ax {
    \argtype{y}{\bb{S}}
    \to
    \left(
    z
    \to
    \left[
    \begin{aligned}
        &\exists! w \ (x \ z \ w) \\
        &\nexists w \ (x \ z \ w)
    \end{aligned}
    \right.
    \right)
    \to
    \exists z \ w
    \to
    \left(w \in z \iff \exists i \in y \ x \ i \ w\right)
    }
\end{flalign*}

\section{Определения и обозначения}
Класс - это $ \set{x \ \left| \ \varphi(x)\right.} $.
Не все классы являются множествами. Все множества являются классами.
\begin{flalign*}
    &\ax{\type{\bb{S}}{\bb{T}}} \\
    &\ax{\type{\_\in\_}{\bb{S} \ \bb{S} \ \bb{T}}} \\
    &\gr{a}{}{}{\_\in\_} \\
    &\df {
    \begin{cases}
        x \to y \\
        y \to x
    \end{cases}
    }
    {x \iff y} \\
    &\gr{r}{}{\_\to\_}{\_\iff\_} \\
    &\df{z \to \left(z \in x \iff z \in y\right)}{x \equiv y} \\
    &\df {
    \begin{cases}
        \exists x \ y \ x \\
        z \to y \ z \to z \equiv \exel\fnd{\exists x \ y \ x}
    \end{cases}
    }
    {\exists! x \ y \ x} \\
    &\df {
    \left[
    \begin{aligned}
        &x \equiv y \\
        &x \equiv z \\
        &\ldots
    \end{aligned}
    \right.
    }
    {x \in \set{y, z, \ldots}} \\
    &\df{\forall z \in x \ z \in y}{x \subseteq y} \\
    &\df {
    \exists z
    \begin{cases}
        x \in z \\
        z \in y
    \end{cases}
    }
    {x \in \cup \ y} \\
    &\df{x \subseteq y}{x \in \mathcal{P} \ y} \\
    &\df {
    w
    \to
    \left(
    w \in y
    \iff
    \begin{cases}
        w \in x \\
        z \ w
    \end{cases}
    \right)
    }
    {y \equiv \set{\alpha \in x \mid z \ \alpha}} \\
    &\df{y \to y \not\in x}{x - \text{пустое}} \\
    &\df {
    \begin{cases}
        \forall y - \text{пустое;} \ y \in x \\
        \forall z \in x \ z \cup \set{z} \in x
    \end{cases}
    }
    {x - \text{индуктивное}} \\
    &\varnothing
    =
    \set{\alpha \in \exel\fnd{\exists_{\bb{S}} \ x - \text{индуктивное}} \mid \bot} \\
    &\df{\cup x \subseteq x}{x - \text{транзитивное}} \\
    &\cap x = \set{y \in \cup x \mid \forall z \in x \ y \in z} \\
    &x \cup y = \cup\set{x, y} \\
    &x \cap y = \cap\set{x, y} \\
    &x \setminus y = \set{\alpha \in x \mid \alpha \not\in y} \\
    &a \triangle b = (a \setminus b) \cup (b \setminus a) \\
    &\df {
    x \to y \to z \to w \to
    \left(
    \begin{cases}
        x \equiv y \\
        z \equiv w
    \end{cases}
    \iff
    i \ x \ z \equiv i \ y \ w
    \right)
    }
    {i - \text{упорядоченная пара}} \\
    &(x, y) = \set{\set{x}, \set{x, y}} \\
    &(\_, \_) - \text{упорядоченная пара} \\
    &x \times y
    =
    \set {
    \mathcal{P} \left(\mathcal{P} \left(x \cup y\right)\right) \
    \left| \
    \exists z, w, i
    \begin{cases}
        z \in x \\
        i \in y \\
        z \equiv (w, i)
    \end{cases}
    \right.
    }
\end{flalign*}

\section{Теоремы}

\chapter{Формальные языки}
\begin{flalign*}
    &a \neq \varnothing \iff a - \text{алфавит} \\
    &a \in A \iff a - \text{символ (буква)} \\
    &f: \underline{n} \rightarrow A \iff f - \text{слово} \\
    &\varepsilon - \text{пустое слово} \\
    &\varepsilon = \varnothing
\end{flalign*}

\chapter{Необработанное}
\textbf{Автонимный способ обозначения} - это
способ обозначения,
при котором формальные выражения обозначаются так же,
как и их значения.

\textbf{Высказывательная форма}.

\textbf{Именная форма} - это
выражение с переменной.

\textbf{Связанные переменные} - это
переменные, вместо которых
нельзя подставить значение.

\textbf{Основания математики} - это
раздел (в книге сказано "аспект")
математической логики,
изучающий объекты математики,
истинные свойства этих объектов,
на основании которых можно вести рассуждения,
а также "сохраняющие истину"{ }способы рассуждений.
\end{document}
