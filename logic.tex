\documentclass[oneside]{book}

\usepackage[utf8]{inputenc}
\usepackage[T2A]{fontenc}
\usepackage[russian]{babel}
\usepackage[left = 0.3\textwidth, right = 0.3\textwidth]{geometry}
\usepackage{parskip}
\usepackage[fleqn]{amsmath}
\usepackage{mathtools}
\usepackage{amsfonts}
\usepackage{amssymb}
\usepackage{graphicx}
\usepackage{hyperref}
\usepackage{bookmark}
\usepackage{textcomp}

\documentclass[oneside]{book}

\usepackage[utf8]{inputenc}
\usepackage[T2A]{fontenc}
\usepackage[russian]{babel}
\usepackage[left = 0.3\textwidth, right = 0.3\textwidth]{geometry}
\usepackage{parskip}
\usepackage[fleqn]{amsmath}
\usepackage{amssymb}
\usepackage{graphicx}
\usepackage{bookmark}
\usepackage{cancel}
\usepackage{mathtools}

\setlength{\parskip}{0.03\textheight}

\hypersetup{
    colorlinks,
    citecolor=black,
    filecolor=black,
    linkcolor=black,
    urlcolor=black
}

\newcommand{\bb}[1]{\mathbb{#1}}
\newcommand{\gr}[4]{#1 \ \left\{#2 \mid #3\right\} \ #4}
\newcommand{\greq}[3]{#1 \ \left\{#2\right\} \ #3}
\newcommand{\dfto}{\text{def-to}}
\newcommand{\ax}[1]{\text{ax} \ #1}
\newcommand{\type}[2]{#1 : #2}
\newcommand{\argtype}[2]{\left(\type{#1}{#2}\right)}
\newcommand{\exel}{\exists \text{-el} \ }
\newcommand{\cstr}{\text{cstr}}
\newcommand{\fnd}[1]{\left[#1\right]}
\newcommand{\set}[1]{\left\{#1\right\}}
\newcommand{\tot}{\leftrightarrow}
\newcommand{\tx}[1]{\text{#1}}
\newcommand{\ntnt}[2]{\tx{"$ #1 $"} - #2}
\newcommand{\lend}{\rule{\textwidth}{0.4pt}}
\newcommand{\notis}{\ \cancel{-} \ }
\newcommand{\andc}{\tx{-}}
\newcommand{\abin}{\smile}
\newcommand{\abind}{\ni}
\newcommand{\concat}{\ {++} \ }


\setlength{\parskip}{0.03\textheight}

\graphicspath{{images/}}

\hypersetup{
    colorlinks,
    citecolor=black,
    filecolor=black,
    linkcolor=black,
    urlcolor=black
}

\title{Логика}
\date{\today}
\author{WinstonMDP}

\begin{document}
    \maketitle

    \tableofcontents

    \chapter{Теория множеств (ZFC)}
    \section{Определения и аксиомы}
    \begin{flalign*}
        &\gr{r}{25}{\_ \ \_} \\
        &\gr{n}{20}{\df{\_}{\_}} \\
        &\gr{l}{50}{\_\gets\_} \\
        &\ax{\type{\bb{S}}{\bb{T}}} \\
        &\ax{\type{\_\in\_}{\bb{S} \ \bb{S} \ \bb{T}}} \\
        &\gr{a}{40}{\_\in\_} \\
        &\df {
        \begin{cases}
            x \ y \\
            y \ x
        \end{cases}
        }
        {x \iff y} \\
        &\gr{r}{23}{\_\iff\_} \\
        &\df{z \ \left(z \in x \iff z \in y\right)}{x == y}
    \end{flalign*}
    \textbf{Аксиома равенства}
    \begin{flalign*}
        \ax{x == y \ z \left(x \in z \iff y \in z\right)}
    \end{flalign*}
    \begin{flalign*}
        &\df {
        \left[
        \begin{aligned}
            &x == y \\
            &x == z \\
            &\ldots
        \end{aligned}
        \right.
        }
        {x \in \set{y, z, \ldots}} \\
        &\df{z \ z \in x \ z \in y}{x \subseteq y}
    \end{flalign*}
    \textbf{Аксиома пары}
    \begin{flalign*}
        \ax{\exists \lmbd{z}{z == \set{x, y}}}
    \end{flalign*}

    \subsection{Объединения}
    \begin{flalign*}
        \exists \cup x
    \end{flalign*}

    \subsection{Степени}
    \begin{flalign*}
        \exists \mathcal{P}(x)
    \end{flalign*}

    \subsection{Выделения}
    \begin{flalign*}
        \exists y \ \forall z \
        \left(
        z \in y
        \iff
        \begin{cases}
            z \in x \\
            \varphi(z)
        \end{cases}
        \right)
    \end{flalign*}

    \subsection{Бесконечности}
    \begin{flalign*}
        \exists S - \text{индуктивное множество}
    \end{flalign*}

    \subsection{Выбора}
    \begin{flalign*}
        \varnothing \not\in S \
        \exists f \
        \begin{cases}
            f: S \rightarrow \cup S \\
            \forall s \in S \ f(s) \in s
        \end{cases}
    \end{flalign*}

    \subsection{Регулярности (фундированности)}
    Необязательная аксиома.
    \begin{flalign*}
        \exists y \in x \ \forall z \in x \ z \not\in y
    \end{flalign*}

    \subsection{Подстановки}
    Необязательная аксиома.
    Аксиома выделения - это часть данной аксиомы.
    \begin{flalign*}
        \left(
        \forall x \
        \left[
        \begin{aligned}
            &\exists! y \ \varphi(x, y) \\
            &\overline{\exists y \ \varphi(x, y)}
        \end{aligned}
        \right.
        \right)
        \forall x \ \exists y \ \forall z \
        \left(z \in y \iff \exists w \in x \ \varphi(w, z)\right)
    \end{flalign*}

    \section{Определения}
    Класс - это $ \set{x \ \left| \ \varphi(x)\right.} $.
    Не все классы являются множествами. Все множества являются классами.
    \begin{flalign*}
        &x = y \iff \left(z \in x \iff z \in y\right) \\
        &y = \set{\ldots, x, \ldots} \iff x \in y \\
        &x \subseteq y \iff \forall z \in x \ z \in y \\
        &x \subsetneq y
        \iff
        \begin{cases}
            x \neq y \\
            x \subseteq y
        \end{cases} \\
        &\varnothing - \text{пустое множество} \\
        &x \not\in \varnothing \\
        &x \in \mathcal{P}(y) \iff x \subseteq y \\
        &y \in \cup x
        \iff
        \exists z
        \begin{cases}
            z \in x \\
            y \in z
        \end{cases} \\
        &x - \text{транзитивное множество} \iff \cup x \subseteq x \\
        &\cap x = \set{y \in \cup x \mid z \in x \ y \in z} \\
        &a \cup b = \cup\set{a, b} \\
        &a \cap b = \set{x \in a \mid x \in b} \\
        &a \setminus b = \set{x \in a \mid x \not\in b} \\
        &a \triangle b = (a \setminus b) \cup (b \setminus a) \\
        &S - \text{индуктивное множество}
        \iff
        \begin{cases}
            \varnothing \in S \\
            \forall s \in S \ s \cup \set{s} \in S
        \end{cases} \\
        &(x, y) = \set{x, \set{x, y}} \\
        &x \times y - \text{декартово (прямое) произведение множеств}\ x \ \text{и} \ y \\
        &z \in x \times y
        \iff
        \exists w, i
        \begin{cases}
            w \in x \\
            i \in y \\
            z = (w, i)
        \end{cases}
    \end{flalign*}

    \section{Теоремы}

    \chapter{Формальные языки}
    \begin{flalign*}
        &a \neq \varnothing \iff a - \text{алфавит} \\
        &a \in A \iff a - \text{символ (буква)} \\
        &f: \underline{n} \rightarrow A \iff f - \text{слово} \\
        &\varepsilon - \text{пустое слово} \\
        &\varepsilon = \varnothing
    \end{flalign*}

    \chapter{Необработанное}
    \textbf{Автонимный способ обозначения} - это
    способ обозначения,
    при котором формальные выражения обозначаются так же,
    как и их значения.

    \textbf{Высказывательная форма}.

    \textbf{Именная форма} - это
    выражение с переменной.

    \textbf{Связанные переменные} - это
    переменные, вместо которых
    нельзя подставить значение.

    \textbf{Основания математики} - это
    раздел (в книге сказано "аспект")
    математической логики,
    изучающий объекты математики,
    истинные свойства этих объектов,
    на основании которых можно вести рассуждения,
    а также "сохраняющие истину"{ }способы рассуждений.
\end{document}
