\documentclass[oneside]{book}

\usepackage[utf8]{inputenc}
\usepackage[T2A]{fontenc}
\usepackage[russian]{babel}
\usepackage[left = 0.3\textwidth, right = 0.3\textwidth]{geometry}
\usepackage{parskip}
\usepackage[fleqn]{amsmath}
\usepackage{amssymb}
\usepackage{graphicx}
\usepackage{bookmark}
\usepackage{cancel}
\usepackage{mathtools}

\setlength{\parskip}{0.03\textheight}

\hypersetup{
    colorlinks,
    citecolor=black,
    filecolor=black,
    linkcolor=black,
    urlcolor=black
}

\newcommand{\bb}[1]{\mathbb{#1}}
\newcommand{\gr}[4]{#1 \ \left\{#2 \mid #3\right\} \ #4}
\newcommand{\greq}[3]{#1 \ \left\{#2\right\} \ #3}
\newcommand{\dfto}{\text{def-to}}
\newcommand{\ax}[1]{\text{ax} \ #1}
\newcommand{\type}[2]{#1 : #2}
\newcommand{\argtype}[2]{\left(\type{#1}{#2}\right)}
\newcommand{\exel}{\exists \text{-el} \ }
\newcommand{\cstr}{\text{cstr}}
\newcommand{\fnd}[1]{\left[#1\right]}
\newcommand{\set}[1]{\left\{#1\right\}}
\newcommand{\tot}{\leftrightarrow}
\newcommand{\tx}[1]{\text{#1}}
\newcommand{\ntnt}[2]{\tx{"$ #1 $"} - #2}
\newcommand{\lend}{\rule{\textwidth}{0.4pt}}
\newcommand{\notis}{\ \cancel{-} \ }
\newcommand{\andc}{\tx{-}}
\newcommand{\abin}{\smile}
\newcommand{\abind}{\ni}
\newcommand{\concat}{\ {++} \ }


\title{Логика}
\date{\today}
\author{WinstonMDP}

\begin{document}
\maketitle

\tableofcontents

\chapter{Основание}
Возможные формы с кванторами
\begin{flalign*}
    &\forall (y \ x) \ (z \ x) \mid \tx{в $ y \ x $ стоит первым символом} \\
    &\exists x \ (y \ x) \\
    &\exists (y \ x) \mid \tx{в $ y \ x $ стоит первым символом} \\
    &\exists (y \ x) \ (z \ x) \mid \tx{в $ y \ x $ стоит первым символом}
\end{flalign*}
\begin{flalign*}
    &\gr{r}{}{}{\_\to\_} \\
    &\gr{n}{}{\_\to\_}{\_=\_} \\
    &\greq{n}{\_=\_}{\_\mapsto\_} \\
    &\gr{n}{\_=\_}{\_\tot\_}{\exists\_\_} \\
    &\greq{n}{\exists\_\_}{\forall\_\_} \\
    &=\bot \\
    &\overline{y} = y \to \bot \\
    &\ax{\left(x \tot y\right) \to \left(z \ x \leftrightarrow z \ y\right)}
\end{flalign*}

\chapter{Аксиомы}
\section{Равенства}
\begin{flalign*}
    \ax{x \equiv y \to \left(x \in z \tot y \in z\right)}
\end{flalign*}

\section{Пары}
\begin{flalign*}
    &\ax {
    x \in \set{y, z}
    =
    \left[
    \begin{aligned}
        &x \equiv y \\
        &x \equiv z
    \end{aligned}
    \right.
    } \\
    &\ntnt{\set{x, y}}{\tx{множество из $ x, y $}}
\end{flalign*}

\section{Объединения}
\begin{flalign*}
    &\ax {
    x \in \cup \ y
    =
    \exists z \in y \ x \in z
    } \\
    &\ntnt{\cup x}{\tx{объединение $ x $}}
\end{flalign*}

\section{Степени}
\begin{flalign*}
    &\ax{x \in \mathcal{P} \ y = x \subseteq y} \\
    &\ntnt{\mathcal{P} \ x}{\tx{множество всех подмножеств $ x $}}
\end{flalign*}

\section{Выделения}
\begin{flalign*}
    &\set{\alpha \in x \mid y \ \alpha}
    = \\
    &\exel
    \left(
    \fnd {
    \argtype{y}{\bb{S}}
    \to
    \left(
    z
    \to
    \left[
    \begin{aligned}
        &\exists! (w \mapsto x \ z \ w) \\
        &\nexists w \ x \ z \ w
    \end{aligned}
    \right.
    \right)
    \to
    \exists z \
    w
    \to
    \left(w \in z \tot \exists i \in y \ x \ i \ w\right)
    }
    \right. \\
    &\left.
    \left(
    z, w \mapsto
    \begin{cases}
        z \equiv w \\
        y \ w
    \end{cases}
    \right) \
    x \
    \fnd {
    z
    \to
    \left[
    \begin{aligned}
        &\exists! w
        \begin{cases}
            z \equiv w \\
            y \ w
        \end{cases} \\
        &\nexists w
        \begin{cases}
            z \equiv w \\
            y \ w
        \end{cases}
    \end{aligned}
    \right.
    }
    \right)
\end{flalign*}
Выводится из аксиомы подстановки.

\section{Бесконечности}
\begin{flalign*}
    \ax{\exists x - \tx{индуктивное}}
\end{flalign*}

\section{Выбора}
В этой системе, которая вроде является системой Мартина-Лёфа Пера, вроде доказуема.
\begin{flalign*}
    \varnothing \not\in x
    \to
    \exists y
    \begin{cases}
        y - \tx{функция из $ x $} \\
        y \subseteq x \times \cup x \\
        \forall z \in x \ y_z \in z
    \end{cases}
\end{flalign*}

\section{Регулярности (фундированности)}
\begin{flalign*}
    \ax{\exists y \in x \ \forall z \in x \ z \not\in y}
\end{flalign*}

\section{Подстановки}
\begin{flalign*}
    \ax {
    \argtype{y}{\bb{S}}
    \to
    \left(
    z
    \to
    \left[
    \begin{aligned}
        &\exists! w \mapsto x \ z \ w \\
        &\nexists w \ x \ z \ w
    \end{aligned}
    \right.
    \right)
    \to
    \exists z \ w
    \to
    \left(w \in z \tot \exists i \in y \ x \ i \ w\right)
    }
\end{flalign*}

\chapter{Операции над множествами}
\section{Определения}
Класс - это $ \set{x \ \left| \ \varphi(x)\right.} $.
Не все классы являются множествами. Все множества являются классами.
\begin{flalign*}
    &\ax{\type{\bb{S}}{\bb{T}}} \\
    &\ax{\type{\_\in\_}{\bb{S} \to \bb{S} \to \bb{T}}} \\
    &\gr{t}{}{}{\_\in\_}
\end{flalign*}
\begin{flalign*}
    &x \tot y
    =
    \begin{cases}
        x \to y \\
        y \to x
    \end{cases} \\
    &\gr{r}{}{\_\to\_}{\_\tot\_} \\
    &\tx{$ x $ тогда и только тогда, когда $ y $}
\end{flalign*}
\begin{flalign*}
    &x \equiv y = z \in x \tot z \in y \\
    &\gr{t}{}{}{\_\equiv\_}
\end{flalign*}
\begin{flalign*}
    &\exists! x
    =
    \begin{cases}
        \exists y \ x \ y \\
        x \ z \to z \equiv \exel\fnd{\exists y \ x \ y}
    \end{cases} \\
    &\tx{Существует единственное, удовлетворяющее $ x $}
\end{flalign*}
\begin{flalign*}
    &x \subseteq y = z \in x \to z \in y \\
    &\tx{$ x $ - подмножество $ y $}
\end{flalign*}
\begin{flalign*}
    &\set{x, y, \ldots} = \set{x, \cup\set{y, \cup\set{\ldots}}} \\
    &\tx{Множество из $ x, y, \ldots $}
\end{flalign*}
\begin{flalign*}
    &x - \tx{пустое} = y \to y \not\in x \\
    &x - \tx{индуктивное}
    =
    \begin{cases}
        y - \tx{пустое} \to y \in x \\
        z \in x \to z \cup \set{z} \in x
    \end{cases} \\
    &\varnothing
    =
    \set{\alpha \in \exel\fnd{\exists_{\bb{S}} \ x - \tx{индуктивное}} \mid \bot} \\
    &x - \tx{транзитивное} = \cup x \subseteq x
\end{flalign*}
\begin{flalign*}
    &\cap x = \set{y \in \cup x \mid \forall z \in x \ y \in z} \\
    &\tx{Пересечение x}
\end{flalign*}
\begin{flalign*}
    &x \cup y = \cup\set{x, y} \\
    &\tx{Объединение $ x $ и $ y $}
\end{flalign*}
\begin{flalign*}
    &x \cap y = \cap\set{x, y} \\
    &\tx{Пересечение $ x $ и $ y $}
\end{flalign*}
\begin{flalign*}
    &x \setminus y = \set{\alpha \in x \mid \alpha \not\in y} \\
    &\tx{Разность $ x $ и $ y $}
\end{flalign*}
\begin{flalign*}
    &x \triangle y = (x \setminus y) \cup (y \setminus x) \\
    &\tx{Симметрическая разность $ x $ и $ y $}
\end{flalign*}

\section{Теоремы}
\begin{flalign*}
    \exists! (x \mapsto x - \tx{пустое})
\end{flalign*}

\chapter{Бинарные отношения}
\section{Определения}
\begin{flalign*}
    &(x, y) = \set{\set{x}, \set{x, y}} \\
    &\tx{Упорядоченная пара Куратовского $ x $ и $ y $}
\end{flalign*}
\begin{flalign*}
    &x \times y
    =
    \set {
    (\alpha, \beta) \in \mathcal{P}\left(\mathcal{P} \left(x \cup y\right)\right) \
    \left| \
    \begin{cases}
        \alpha \in x \\
        \beta \in y
    \end{cases}
    \right.
    } \\
    &\tx{Прямое (декартово) произведение $ x $ и $ y $}
\end{flalign*}
\begin{flalign*}
    x - \tx{бинарное отношение} = \exists y, z \ x \subseteq y \times z
\end{flalign*}
\begin{flalign*}
    &\tx{dom} \ x \ \fnd{x - \tx{бинарное отношение}}
    =
     \set{\alpha \in \cup\cup x \mid \exists y \ (\alpha, y) \in x} \\
    &\tx{Область определения $ x $}
\end{flalign*}
\begin{flalign*}
    &\tx{rng} \ x \ \fnd{x - \tx{бинарное отношение}}
    =
    \set{\alpha \in \cup\cup x \mid \exists y \ (y, \alpha) \in x} \\
    &\tx{Область значений $ x $}
\end{flalign*}
\begin{flalign*}
    x - \tx{поле $ y $}
    =
    \begin{cases}
        x - \tx{бинарное отношение} \\
        x \equiv \cup\cup y
    \end{cases}
\end{flalign*}
\begin{flalign*}
    &\tx{inv} \ x
    =
    \set {
    (\alpha, \beta) \in \tx{rng} \ x \times \tx{dom} \ x
    \mid
    (\beta, \alpha) \in x
    } \\
    &\tx{Обратное к $ x $}
\end{flalign*}
\begin{flalign*}
    &x \circ y
    =
    \set {
    (\alpha, \beta) \in \tx{dom} \ y \times \tx{rng} \ x \
    \left| \
    \exists z
    \begin{cases}
        (\alpha, z) \in y \\
        (z, \beta) \in x
    \end{cases}
    \right.
    } \\
    &\tx{Композиция $ x $ и $ y $}
\end{flalign*}
\begin{flalign*}
    \tx{idr} \ x
    =
    \set{(\alpha, \alpha) \in \tx{zip} \ x^2 \mid \top}
\end{flalign*}
\begin{flalign*}
    &x \upharpoonright y \ \fnd{x - \tx{бинарное отношение}}
    =
    \set{(\alpha, \beta) \in x \mid \alpha \in y} \\
    &\tx{Ограничение $ x $ на $ y $ слева}
\end{flalign*}
\begin{flalign*}
    &x \upharpoonleft y \ \fnd{x - \tx{бинарное отношение}}
    =
    \set{(\alpha, \beta) \in x \mid \beta \in y} \\
    &\tx{Ограничение $ x $ на $ y $ справа}
\end{flalign*}
\begin{flalign*}
    &x \uparrow y = x \upharpoonleft y \upharpoonright y \\
    &\tx{Ограничение $ x $ на $ y $}
\end{flalign*}
\begin{flalign*}
    &x - \tx{тотальное на $ y $} = y \subseteq \tx{dom} \ x \\
    &x - \tx{сюръективное на $ y $} = y \subseteq \tx{rng} \ x \\
    &x - \tx{функциональное}
    =
    \begin{cases}
        x - \tx{бинарное отношение} \\
        (y, z) \in x \to (y, w) \in x \to z \equiv w
    \end{cases} \\
    &x - \tx{инъективное}
    =
    \begin{cases}
        x - \tx{бинарное отношение} \\
        (y, z) \in x \to (w, z) \in x \to y \equiv w
    \end{cases}
\end{flalign*}
\begin{flalign*}
    \begin{aligned}
        x{ }-{ }&\tx{функция (индексированное семейство} \\
                &\tx{множеств с индексами) из $ y $}
    \end{aligned}
    =
    \begin{cases}
        \exists z \ x \subseteq y \times z \\
        x - \tx{тотальное на $ y $} \\
        x - \tx{функциональное}
    \end{cases}
\end{flalign*}
\begin{flalign*}
    x_y \
    \fnd{
    \exists z
    \begin{cases}
        x - \tx{функция из $ z $} \\
        y \in z
    \end{cases}
    }
    =
    \exel\fnd{\exists w \ (y, w) \in x}
\end{flalign*}
\begin{flalign*}
    x^y
    =
    \set{\alpha \in \mathcal{P}\left(y \times x\right) \mid \alpha - \tx{функция из $ y $}}
\end{flalign*}
\begin{flalign*}
    &\prod x
    =
    \set {
    \alpha \in (\cup\tx{rng} \ x)^{\tx{dom} \ x}
    \mid
    \forall y \in \tx{dom} \ x \ \alpha_y \in x_y} \\
    &\tx{Прямое (декартово) произведение $ x $}
\end{flalign*}
\begin{flalign*}
    \bigsqcup x
    =
    \set{(\alpha, \beta) \in \cup\tx{rng} \ x \times \tx{dom} \ x \mid \alpha \in x_{\beta}}
\end{flalign*}
\begin{flalign*}
    \tx{zip} \ x^2 = x \times x
\end{flalign*}
\begin{flalign*}
    &x - \tx{инъекция из $ y $}
    =
    \begin{cases}
        x - \tx{функция из $ y $} \\
        x - \tx{инъективное}
    \end{cases} \\
    &x - \tx{сюръекция из $ y $ в $ z $}
    =
    \begin{cases}
        x - \tx{функция из $ y $} \\
        x - \tx{суръективно на $ z $}
    \end{cases} \\
    &x - \tx{биекция из $ y $ в $ z $}
    =
    \begin{cases}
        x - \tx{инъекция из $ y $} \\
        x - \tx{сюръекция из $ y $ в $ z $}
    \end{cases}
\end{flalign*}
\begin{flalign*}
    &x \lesssim y
    =
    \exists z
    \begin{cases}
        z \subseteq x \times y \\
        z - \tx{инъекция из $ x $}
    \end{cases} \\
    &\tx{$ x $ вложен в $ y $}
\end{flalign*}
\begin{flalign*}
    &x \sim y = \exists z - \tx{биекция из $ x $ в $ y $} \\
    &\tx{$ x $ равномощен $ y $}
\end{flalign*}
\begin{flalign*}
    &x - \tx{рефлексивное} = \forall y \in \tx{dom} \ x \ (y, y) \in x \\
    &x - \tx{иррефлексивное}
    =
    \begin{cases}
        x - \tx{бинарное отношение} \\
        y \to (y, y) \not\in x
    \end{cases} \\
    &x - \tx{симметричное}
    =
    \begin{cases}
        x - \tx{бинарное отношение} \\
        (y, z) \in x \to (z, y) \in x
    \end{cases} \\
    &x - \tx{антисимметричное}
    =
    \begin{cases}
        x - \tx{бинарное отношение} \\
        (y, z) \in x \to (z, y) \in x \to y \equiv z
    \end{cases} \\
    &x - \tx{транзитивное}
    =
    \begin{cases}
        x - \tx{бинарное отношение} \\
        (y, z) \in x \to (z, w) \in x \to (y, w) \in x
    \end{cases}
\end{flalign*}
\begin{flalign*}
    &x - \tx{$ y $-минимальный}
    =
    \begin{cases}
        y - \tx{иррефлексивное} \\
        \forall z \in \tx{dom} \ y \ (z, x) \not\in y
    \end{cases} \\
    &x - \tx{$ y $-максимальный}
    =
    \begin{cases}
        y - \tx{иррефлексивное} \\
        \forall z \in \tx{rng} \ y \ (x, z) \not\in y
    \end{cases} \\
    &\tx{min} \ x \ y \ \fnd{y - \tx{иррефлексивное}}
    =
    \set{\alpha \in x \mid \alpha - \tx{$ y $-минимальный}} \\
    &\tx{max} \ x \ y \ \fnd{y - \tx{иррефлексивное}}
    =
    \set{\alpha \in x \mid \alpha - \tx{$ y $-максимальный}}
\end{flalign*}

\section{Теоремы}
\begin{flalign*}
    &\begin{cases}
        x \equiv y \\
        z \equiv w
    \end{cases}
    \tot
    (x, z) \equiv (y, w) \\
    &\begin{cases}
        x \equiv y \\
        z \equiv w
    \end{cases}
    \tot
    x \times z \equiv y \times w \\
    &x \not\equiv \varnothing
    \to
    y \not\equiv \varnothing
    \to
    \left(x \times y\right) \cup \left(y \times x\right) \equiv z \times w
    \to
    x \equiv y \equiv z \equiv w \\
    &(y, z) \in x \to y \in \tx{dom} \ x \\
    &(y, z) \in x \to z \in \tx{rng} \ x \\
    &\cup\cup x \equiv \tx{dom} \ x \cup \tx{rng} \ x \\
    &\tx{inv} \ x \subseteq x \tot x \equiv \tx{inv} \ x \\
    &\begin{cases}
        \tx{inv} \ x \circ x \equiv \tx{idr} \ y \\
        x \circ \tx{inv} \ x \equiv \tx{idr} \ z
    \end{cases}
    \tot
    x - \tx{биекция из $ y $ в $ z $} \\
    &x \lesssim y \tot \exists z \subseteq y \ x \sim z
\end{flalign*}

\subsection{Кантора-Шрёдера-Бернштейна}
\begin{flalign*}
    x \lesssim y \to y \lesssim x \to x \sim y
\end{flalign*}

\chapter{Порядки}
\section{Определения}
\begin{flalign*}
    &x - \tx{предпорядок}
    =
    \begin{cases}
        x - \tx{рефлексивное} \\
        x - \tx{транзитивное}
    \end{cases} \\
    &x - \tx{строгий порядок}
    =
    \begin{cases}
        x - \tx{иррефлексивное} \\
        x - \tx{транзитивное}
    \end{cases} \\
    &x - \tx{нестрогий порядок}
    =
    \begin{cases}
        x - \tx{предпорядок} \\
        x - \tx{антисимметричное}
    \end{cases}
\end{flalign*}
\begin{flalign*}
    &x - \tx{$ y $-нижняя грань $ z $}
    =
    \begin{cases}
        y - \tx{нестрогий порядок} \\
        \forall w \in z \ (x, w) \in y
    \end{cases} \\
    &x - \tx{$ y $-верхняя грань $ z $}
    =
    \begin{cases}
        y - \tx{нестрогий порядок} \\
        \forall w \in z \ (w, x) \in y
    \end{cases}
\end{flalign*}
\begin{flalign*}
    &x - \tx{$ y $-наименьший}
    =
    x - \tx{$ y $-нижняя грань $ \tx{rng} \ y $} \\
    &x - \tx{$ y $-наибольший}
    =
    x - \tx{$ y $-верхняя грань $ \tx{rng} \ y $}
\end{flalign*}
\begin{flalign*}
    &x - \tx{$ y $-инфимум ($ y $-точная нижняя грань) $ z $}
    = \\
    &x
    -
    \tx {$
    \left(y \uparrow \set{\alpha \in z \mid \alpha - \tx{нижняя грань $ z $}}\right)
    $-наибольший
    } \\
    &x - \tx{$ y $-супремум ($ y $-точная верхняя грань) $ z $}
    = \\
    &x
    -
    \tx {$
    \left(y \uparrow \set{\alpha \in z \mid \alpha - \tx{верхняя грань $ z $}}\right)
    $-наименьший
    } \\
    &\tx{inf} \ x \ y \ \fnd{\exists z - \tx{$ y $-инфимум $ x $}}
    =
    \cup\set{\alpha \in x \mid \alpha - \tx{$ y $-инфимум $ x $}} \\
    &\tx{sup} \ x \ y \ \fnd{\exists z - \tx{$ y $-супремум $ x $}}
    =
    \cup\set{\alpha \in x \mid \alpha - \tx{$ y $-супремум $ x $}}
\end{flalign*}
\begin{flalign*}
    &x - \tx{$ y $-цепь}
    =
    \begin{cases}
        y - \tx{нестрогий порядок} \\
        \forall z \andc w \in x
        \left[
        \begin{aligned}
            (z, w) \in y \\
            (w, z) \in y
        \end{aligned}
        \right.
    \end{cases} \\
    &x - \tx{$ y $-антицепь}
    =
    \begin{cases}
        y - \tx{нестрогий порядок} \\
        \forall z \andc w \in x \ (z, w) \in y \to z \equiv w
    \end{cases}
\end{flalign*}
\begin{flalign*}
    (x, y) - \tx{частично упорядоченное множество (ч.у.м.)}
    =
    \begin{cases}
        y - \tx{нестрогий порядок} \\
        \tx{dom} \ y \equiv x
    \end{cases}
\end{flalign*}
\begin{flalign*}
    (x, y) - \tx{решётка}
    =
    \begin{cases}
        (x, y) - \tx{ч.у.м.} \\
        \forall z \in x, w \in x
        \begin{cases}
            \exists i - \tx{$ y $-инфимум $ \set{z, w} $} \\
            \exists i - \tx{$ y $-супремум $ \set{z, w} $}
        \end{cases}
    \end{cases}
\end{flalign*}
\begin{flalign*}
    (x, y) - \tx{полная решётка}
    =
    \begin{cases}
        (x, y) - \tx{ч.у.м.} \\
        \forall z \subseteq x
        \begin{cases}
            \exists w - \tx{$ y $-инфимум $ z $} \\
            \exists w - \tx{$ y $-супремум $ z $}
        \end{cases}
    \end{cases}
\end{flalign*}
\begin{flalign*}
    (x, y) - \tx{линейно упорядоченное множество (л.у.м.)}
    =
    \begin{cases}
        (x, y) - \tx{ч.у.м.} \\
        x - \tx{$ y $-цепь}
    \end{cases}
\end{flalign*}
\begin{flalign*}
    &\begin{aligned}
        (x, y){ }-{ }&\tx{соответствие Галуа между} \\
                     &\tx{$ (z, w) $ и $ (i, j) $}
    \end{aligned}
    =
    \begin{cases}
        (z, w) \andc (i, j) - \tx{ч.у.м} \\
        x \subseteq z \times i \\
        x - \tx{функция из $ z $} \\
        y \subseteq i \times z \\
        y - \tx{функция из $ i $} \\
        \forall k \in z, t \in i \ (k, y_t) \in w \tot (x_k, t) \in j
    \end{cases}
\end{flalign*}
\begin{flalign*}
    x - \tx{эквивалентность}
    =
    \begin{cases}
        x - \tx{предпорядок} \\
        x - \tx{симметричное}
    \end{cases}
\end{flalign*}
\begin{flalign*}
    &\tx{ker} \ x \ \fnd{\exists y \ x - \tx{функция из $ y $}}
    =
    \set {
    (\alpha, \beta) \in \tx{zip} \left(\tx{dom} \ x\right)^2
    \mid
    x_{\alpha} \equiv x_{\beta}
    } \\
    &\tx{Ядерная эквивалентность $ x $}
\end{flalign*}
\begin{flalign*}
    &\tx{class} \ x \ y \ \fnd{y - \tx{эквивалентность}}
    =
    \set{\alpha \in \tx{dom} \ y \mid \exists z \ (x, z) \in y} \\
    &\tx{Класс эквивалентности $ x $ по $ y $}
\end{flalign*}
\begin{flalign*}
    &x / y = \set{\tx{class} \ \alpha \ y \in \mathcal{P} \ x \mid \alpha \in x} \\
    &\tx{Фактор-множество $ x $ по $ y $}
\end{flalign*}
\begin{flalign*}
    x - \tx{разбиение $ y $}
    =
    \begin{cases}
        x \subseteq \mathcal{P} \ y \\
        \cup x \equiv y \\
        \varnothing \not\in x \\
        \forall z \andc w \in x \ z \cap w \not\equiv \varnothing \to z \equiv w
    \end{cases}
\end{flalign*}

\section{Теоремы}
\begin{flalign*}
    &\set {
    \alpha \in \mathcal{P}\left(\tx{zip} \ x^2\right) \mid \alpha - \tx{строгий порядок}
    }
    \sim
    \set {
    \alpha \in \mathcal{P}\left(\tx{zip} \ x^2\right)
    \mid
    \alpha - \tx{нестрогий порядок}
    } \\
    &(x, y) - \tx{ч.у.м.}
    \to
    \left(
    \left(
    \forall z \subseteq x \
    \exists w - \tx{$ y $-инфимум $ z $}
    \right)
    \tot
    \left(
    \forall z \subseteq x \
    \exists w - \tx{$ y $-супремум $ z $}
    \right)
    \right) \\
    &(x, y) - \tx{ч.у.м.}
    \to
    \exists z \subseteq \mathcal{P}\ x \
    (x, y)
    \cong
    \left(
    z,
    \set {
    (\alpha, \beta) \in \tx{zip}\left(\mathcal{P} \ x\right)^2 \mid \alpha \subseteq \beta
    }
    \right) \\
    &\set {
    \alpha \in x^z
    \mid
    \exists i \ (\alpha, i) - \tx{соответствие Галуа между $ (z, y) $ и $ (x, w) $}
    }
    \sim \\
    &\set {
    \alpha \in z^x
    \mid
    \exists i \ (i, \alpha) - \tx{соответствие Галуа между $ (z, y) $ и $ (x, w) $}
    } \\
    &\set {
    \alpha \in \mathcal{P}\left(\tx{zip} \ x^2\right)
    \mid
    \alpha - \tx{эквивалентность}
    }
    \sim
    \set{\alpha \in \mathcal{P} \ \mathcal{P} \ x \mid \alpha - \tx{разбиение $ x $}} \\
    &\left(
    \set {
    \alpha \in \mathcal{P}\left(\tx{zip} \ x^2\right)
    \mid
    \alpha - \tx{эквивалентность}
    },
    \set{(\alpha, \beta) \in \tx{zip} \ x^2 \mid \alpha \subseteq \beta}
    \right)
    - \\
    &\tx{полная решётка}
\end{flalign*}
\end{document}
